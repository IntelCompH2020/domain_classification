%% Generated by Sphinx.
\def\sphinxdocclass{report}
\documentclass[letterpaper,10pt,english]{sphinxmanual}
\ifdefined\pdfpxdimen
   \let\sphinxpxdimen\pdfpxdimen\else\newdimen\sphinxpxdimen
\fi \sphinxpxdimen=.75bp\relax
\ifdefined\pdfimageresolution
    \pdfimageresolution= \numexpr \dimexpr1in\relax/\sphinxpxdimen\relax
\fi
%% let collapsible pdf bookmarks panel have high depth per default
\PassOptionsToPackage{bookmarksdepth=5}{hyperref}

\PassOptionsToPackage{warn}{textcomp}
\usepackage[utf8]{inputenc}
\ifdefined\DeclareUnicodeCharacter
% support both utf8 and utf8x syntaxes
  \ifdefined\DeclareUnicodeCharacterAsOptional
    \def\sphinxDUC#1{\DeclareUnicodeCharacter{"#1}}
  \else
    \let\sphinxDUC\DeclareUnicodeCharacter
  \fi
  \sphinxDUC{00A0}{\nobreakspace}
  \sphinxDUC{2500}{\sphinxunichar{2500}}
  \sphinxDUC{2502}{\sphinxunichar{2502}}
  \sphinxDUC{2514}{\sphinxunichar{2514}}
  \sphinxDUC{251C}{\sphinxunichar{251C}}
  \sphinxDUC{2572}{\textbackslash}
\fi
\usepackage{cmap}
\usepackage[T1]{fontenc}
\usepackage{amsmath,amssymb,amstext}
\usepackage{babel}



\usepackage{tgtermes}
\usepackage{tgheros}
\renewcommand{\ttdefault}{txtt}



\usepackage[Bjarne]{fncychap}
\usepackage{sphinx}

\fvset{fontsize=auto}
\usepackage{geometry}


% Include hyperref last.
\usepackage{hyperref}
% Fix anchor placement for figures with captions.
\usepackage{hypcap}% it must be loaded after hyperref.
% Set up styles of URL: it should be placed after hyperref.
\urlstyle{same}

\addto\captionsenglish{\renewcommand{\contentsname}{Contents:}}

\usepackage{sphinxmessages}
\setcounter{tocdepth}{1}



\title{domain\_classification\_doc}
\date{Apr 04, 2022}
\release{}
\author{Jesús Cid, Manuel A.\@{} Vázquez}
\newcommand{\sphinxlogo}{\vbox{}}
\renewcommand{\releasename}{}
\makeindex
\begin{document}

\pagestyle{empty}
\sphinxmaketitle
\pagestyle{plain}
\sphinxtableofcontents
\pagestyle{normal}
\phantomsection\label{\detokenize{index::doc}}



\chapter{Menu Navigator}
\label{\detokenize{mn_menu_navigator:menu-navigator}}\label{\detokenize{mn_menu_navigator::doc}}\phantomsection\label{\detokenize{mn_menu_navigator:module-src.menu_navigator.menu_navigator}}\index{module@\spxentry{module}!src.menu\_navigator.menu\_navigator@\spxentry{src.menu\_navigator.menu\_navigator}}\index{src.menu\_navigator.menu\_navigator@\spxentry{src.menu\_navigator.menu\_navigator}!module@\spxentry{module}}
\sphinxAtStartPar
A generic class to manage the user navigation through a multilevel options
menu.

\sphinxAtStartPar
Created on March. 04, 2019
\begin{description}
\item[{@author: Jesús Cid Sueiro}] \leavevmode
\sphinxAtStartPar
Based on former menu manager scripts by Jerónimo Arenas.

\end{description}
\index{MenuNavigator (class in src.menu\_navigator.menu\_navigator)@\spxentry{MenuNavigator}\spxextra{class in src.menu\_navigator.menu\_navigator}}

\begin{fulllineitems}
\phantomsection\label{\detokenize{mn_menu_navigator:src.menu_navigator.menu_navigator.MenuNavigator}}\pysiglinewithargsret{\sphinxbfcode{\sphinxupquote{class\DUrole{w}{  }}}\sphinxcode{\sphinxupquote{src.menu\_navigator.menu\_navigator.}}\sphinxbfcode{\sphinxupquote{MenuNavigator}}}{\emph{\DUrole{n}{tm}}, \emph{\DUrole{n}{path2menu}}, \emph{\DUrole{n}{paths2data}\DUrole{o}{=}\DUrole{default_value}{None}}}{}
\sphinxAtStartPar
Bases: \sphinxcode{\sphinxupquote{object}}

\sphinxAtStartPar
A class to manage the user navigation through a multilevel options menu.

\sphinxAtStartPar
The structure of multilevel menu options with their associated
actions and parameters should be defined in a yaml file.

\sphinxAtStartPar
A taskmanager class is required to take the selected actions with the
given parameters.
\index{\_\_init\_\_() (src.menu\_navigator.menu\_navigator.MenuNavigator method)@\spxentry{\_\_init\_\_()}\spxextra{src.menu\_navigator.menu\_navigator.MenuNavigator method}}

\begin{fulllineitems}
\phantomsection\label{\detokenize{mn_menu_navigator:src.menu_navigator.menu_navigator.MenuNavigator.__init__}}\pysiglinewithargsret{\sphinxbfcode{\sphinxupquote{\_\_init\_\_}}}{\emph{\DUrole{n}{tm}}, \emph{\DUrole{n}{path2menu}}, \emph{\DUrole{n}{paths2data}\DUrole{o}{=}\DUrole{default_value}{None}}}{}
\sphinxAtStartPar
Initializes a menu navigator.
\begin{quote}\begin{description}
\item[{Parameters}] \leavevmode\begin{itemize}
\item {} 
\sphinxAtStartPar
\sphinxstylestrong{tm} (\sphinxstyleemphasis{object}) \textendash{} A task manager object, that will be in charge of executing all
actions selected by the user through the menu interaction. Thus, it
must contain:
(1) One action method per method specified in the menu structure
(2) Data collection methods, required for some menus with dynamic
options.

\item {} 
\sphinxAtStartPar
\sphinxstylestrong{path2menu} (\sphinxstyleemphasis{str}) \textendash{} The route to the yaml file containing the menu structure

\item {} 
\sphinxAtStartPar
\sphinxstylestrong{paths2data} (\sphinxstyleemphasis{dict or None, optional (default=None)}) \textendash{} A dictionary of paths to data repositories. The key is a name of
the path, and the value is the path.

\end{itemize}

\end{description}\end{quote}

\end{fulllineitems}

\index{\_\_weakref\_\_ (src.menu\_navigator.menu\_navigator.MenuNavigator attribute)@\spxentry{\_\_weakref\_\_}\spxextra{src.menu\_navigator.menu\_navigator.MenuNavigator attribute}}

\begin{fulllineitems}
\phantomsection\label{\detokenize{mn_menu_navigator:src.menu_navigator.menu_navigator.MenuNavigator.__weakref__}}\pysigline{\sphinxbfcode{\sphinxupquote{\_\_weakref\_\_}}}
\sphinxAtStartPar
list of weak references to the object (if defined)

\end{fulllineitems}

\index{clear() (src.menu\_navigator.menu\_navigator.MenuNavigator method)@\spxentry{clear()}\spxextra{src.menu\_navigator.menu\_navigator.MenuNavigator method}}

\begin{fulllineitems}
\phantomsection\label{\detokenize{mn_menu_navigator:src.menu_navigator.menu_navigator.MenuNavigator.clear}}\pysiglinewithargsret{\sphinxbfcode{\sphinxupquote{clear}}}{}{}
\sphinxAtStartPar
Cleans terminal window

\end{fulllineitems}

\index{front\_page() (src.menu\_navigator.menu\_navigator.MenuNavigator method)@\spxentry{front\_page()}\spxextra{src.menu\_navigator.menu\_navigator.MenuNavigator method}}

\begin{fulllineitems}
\phantomsection\label{\detokenize{mn_menu_navigator:src.menu_navigator.menu_navigator.MenuNavigator.front_page}}\pysiglinewithargsret{\sphinxbfcode{\sphinxupquote{front\_page}}}{\emph{\DUrole{n}{title}}}{}
\sphinxAtStartPar
Prints a simple title heading the application user screen
\begin{quote}\begin{description}
\item[{Parameters}] \leavevmode
\sphinxAtStartPar
\sphinxstylestrong{title} (\sphinxstyleemphasis{str}) \textendash{} Title message to be printed

\end{description}\end{quote}

\end{fulllineitems}

\index{navigate() (src.menu\_navigator.menu\_navigator.MenuNavigator method)@\spxentry{navigate()}\spxextra{src.menu\_navigator.menu\_navigator.MenuNavigator method}}

\begin{fulllineitems}
\phantomsection\label{\detokenize{mn_menu_navigator:src.menu_navigator.menu_navigator.MenuNavigator.navigate}}\pysiglinewithargsret{\sphinxbfcode{\sphinxupquote{navigate}}}{\emph{\DUrole{n}{option}\DUrole{o}{=}\DUrole{default_value}{None}}, \emph{\DUrole{n}{active\_options}\DUrole{o}{=}\DUrole{default_value}{None}}}{}
\sphinxAtStartPar
Manages the menu navigation loop
\begin{quote}\begin{description}
\item[{Parameters}] \leavevmode\begin{itemize}
\item {} 
\sphinxAtStartPar
\sphinxstylestrong{options} (\sphinxstyleemphasis{dict}) \textendash{} A dictionary of options

\item {} 
\sphinxAtStartPar
\sphinxstylestrong{active\_options} (\sphinxstyleemphasis{list or None, optional (default=None)}) \textendash{} List of option keys indicating the available options to print.
If None, all options are shown.

\end{itemize}

\end{description}\end{quote}

\end{fulllineitems}

\index{query\_options() (src.menu\_navigator.menu\_navigator.MenuNavigator method)@\spxentry{query\_options()}\spxextra{src.menu\_navigator.menu\_navigator.MenuNavigator method}}

\begin{fulllineitems}
\phantomsection\label{\detokenize{mn_menu_navigator:src.menu_navigator.menu_navigator.MenuNavigator.query_options}}\pysiglinewithargsret{\sphinxbfcode{\sphinxupquote{query\_options}}}{\emph{\DUrole{n}{options}}, \emph{\DUrole{n}{active\_options}\DUrole{o}{=}\DUrole{default_value}{None}}, \emph{\DUrole{n}{msg}\DUrole{o}{=}\DUrole{default_value}{None}}, \emph{\DUrole{n}{zero\_option}\DUrole{o}{=}\DUrole{default_value}{\textquotesingle{}exit\textquotesingle{}}}}{}
\sphinxAtStartPar
Prints a heading mnd the subset of options indicated in the list of
active\_options, and returns the one selected by the used
\begin{quote}\begin{description}
\item[{Parameters}] \leavevmode\begin{itemize}
\item {} 
\sphinxAtStartPar
\sphinxstylestrong{options} (\sphinxstyleemphasis{dict}) \textendash{} A dictionary of options

\item {} 
\sphinxAtStartPar
\sphinxstylestrong{active\_options} (\sphinxstyleemphasis{list or None, optional (default=None)}) \textendash{} List of option keys indicating the available options to print.
If None, all options are shown.

\item {} 
\sphinxAtStartPar
\sphinxstylestrong{msg} (\sphinxstyleemphasis{str or None, optional (default=None)}) \textendash{} Heading message to be printed before the list of available options

\item {} 
\sphinxAtStartPar
\sphinxstylestrong{zero\_option} (\sphinxstyleemphasis{str \{‘exit’, ‘up’\}, optional (default=’exit’)}) \textendash{} If ‘exit’, an exit option is shown
If ‘up’, an option to go back to the main menu

\end{itemize}

\item[{Returns}] \leavevmode
\sphinxAtStartPar
\sphinxstylestrong{option} \textendash{} Selected option

\item[{Return type}] \leavevmode
\sphinxAtStartPar
str

\end{description}\end{quote}

\end{fulllineitems}

\index{request\_confirmation() (src.menu\_navigator.menu\_navigator.MenuNavigator method)@\spxentry{request\_confirmation()}\spxextra{src.menu\_navigator.menu\_navigator.MenuNavigator method}}

\begin{fulllineitems}
\phantomsection\label{\detokenize{mn_menu_navigator:src.menu_navigator.menu_navigator.MenuNavigator.request_confirmation}}\pysiglinewithargsret{\sphinxbfcode{\sphinxupquote{request\_confirmation}}}{\emph{\DUrole{n}{msg}\DUrole{o}{=}\DUrole{default_value}{\textquotesingle{}     Are you sure?\textquotesingle{}}}}{}
\sphinxAtStartPar
Request confirmation from user
\begin{quote}\begin{description}
\item[{Parameters}] \leavevmode
\sphinxAtStartPar
\sphinxstylestrong{msg} (\sphinxstyleemphasis{str, optional (default=”    Are you sure?”)}) \textendash{} Message printed to request confirmation

\item[{Returns}] \leavevmode
\sphinxAtStartPar
\sphinxstylestrong{r} \textendash{} User respones

\item[{Return type}] \leavevmode
\sphinxAtStartPar
str \{‘yes’, ‘no’\}

\end{description}\end{quote}

\end{fulllineitems}


\end{fulllineitems}



\chapter{Base Task Manager}
\label{\detokenize{dc_base_taskmanager:base-task-manager}}\label{\detokenize{dc_base_taskmanager::doc}}\phantomsection\label{\detokenize{dc_base_taskmanager:module-src.base_taskmanager}}\index{module@\spxentry{module}!src.base\_taskmanager@\spxentry{src.base\_taskmanager}}\index{src.base\_taskmanager@\spxentry{src.base\_taskmanager}!module@\spxentry{module}}\index{baseTaskManager (class in src.base\_taskmanager)@\spxentry{baseTaskManager}\spxextra{class in src.base\_taskmanager}}

\begin{fulllineitems}
\phantomsection\label{\detokenize{dc_base_taskmanager:src.base_taskmanager.baseTaskManager}}\pysiglinewithargsret{\sphinxbfcode{\sphinxupquote{class\DUrole{w}{  }}}\sphinxcode{\sphinxupquote{src.base\_taskmanager.}}\sphinxbfcode{\sphinxupquote{baseTaskManager}}}{\emph{\DUrole{n}{path2project}}, \emph{\DUrole{n}{path2source}\DUrole{o}{=}\DUrole{default_value}{None}}, \emph{\DUrole{n}{config\_fname}\DUrole{o}{=}\DUrole{default_value}{\textquotesingle{}parameters.yaml\textquotesingle{}}}, \emph{\DUrole{n}{metadata\_fname}\DUrole{o}{=}\DUrole{default_value}{\textquotesingle{}metadata.yaml\textquotesingle{}}}, \emph{\DUrole{n}{set\_logs}\DUrole{o}{=}\DUrole{default_value}{True}}}{}
\sphinxAtStartPar
Bases: \sphinxcode{\sphinxupquote{object}}

\sphinxAtStartPar
Base Task Manager class.

\sphinxAtStartPar
This class provides the basic functionality to create, load and setup an
execution project from the main application

\sphinxAtStartPar
The behavior of this class might depend on the state of the project, which
is stored in dictionary self.state, with the followin entries:
\begin{itemize}
\item {} 
\sphinxAtStartPar
‘isProject’   : If True, project created. Metadata variables loaded

\item {} 
\sphinxAtStartPar
‘configReady’ : If True, config file succesfully loaded and processed

\end{itemize}
\index{\_\_init\_\_() (src.base\_taskmanager.baseTaskManager method)@\spxentry{\_\_init\_\_()}\spxextra{src.base\_taskmanager.baseTaskManager method}}

\begin{fulllineitems}
\phantomsection\label{\detokenize{dc_base_taskmanager:src.base_taskmanager.baseTaskManager.__init__}}\pysiglinewithargsret{\sphinxbfcode{\sphinxupquote{\_\_init\_\_}}}{\emph{\DUrole{n}{path2project}}, \emph{\DUrole{n}{path2source}\DUrole{o}{=}\DUrole{default_value}{None}}, \emph{\DUrole{n}{config\_fname}\DUrole{o}{=}\DUrole{default_value}{\textquotesingle{}parameters.yaml\textquotesingle{}}}, \emph{\DUrole{n}{metadata\_fname}\DUrole{o}{=}\DUrole{default_value}{\textquotesingle{}metadata.yaml\textquotesingle{}}}, \emph{\DUrole{n}{set\_logs}\DUrole{o}{=}\DUrole{default_value}{True}}}{}
\sphinxAtStartPar
Sets the main attributes to manage tasks over a specific application
project.
\begin{quote}\begin{description}
\item[{Parameters}] \leavevmode\begin{itemize}
\item {} 
\sphinxAtStartPar
\sphinxstylestrong{path2project} (\sphinxstyleemphasis{str or pathlib.Path}) \textendash{} Path to the application project

\item {} 
\sphinxAtStartPar
\sphinxstylestrong{path2source} (\sphinxstyleemphasis{str or pathlib.Path or None, optional (default=None)}) \textendash{} Paht to the folder containing the data sources for the application.
If none, no source data is used.

\item {} 
\sphinxAtStartPar
\sphinxstylestrong{config\_fname} (\sphinxstyleemphasis{str, optional (default=’parameters.yaml’)}) \textendash{} Name of the configuration file

\item {} 
\sphinxAtStartPar
\sphinxstylestrong{metadata\_fname} (\sphinxstyleemphasis{str or None, optional (default=’metadata.yaml’)}) \textendash{} Name of the project metadata file.

\item {} 
\sphinxAtStartPar
\sphinxstylestrong{set\_logs} (\sphinxstyleemphasis{bool, optional (default=True)}) \textendash{} If True logger objects are created according to the parameters
specified in the configuration file

\end{itemize}

\end{description}\end{quote}

\end{fulllineitems}

\index{\_\_weakref\_\_ (src.base\_taskmanager.baseTaskManager attribute)@\spxentry{\_\_weakref\_\_}\spxextra{src.base\_taskmanager.baseTaskManager attribute}}

\begin{fulllineitems}
\phantomsection\label{\detokenize{dc_base_taskmanager:src.base_taskmanager.baseTaskManager.__weakref__}}\pysigline{\sphinxbfcode{\sphinxupquote{\_\_weakref\_\_}}}
\sphinxAtStartPar
list of weak references to the object (if defined)

\end{fulllineitems}

\index{create() (src.base\_taskmanager.baseTaskManager method)@\spxentry{create()}\spxextra{src.base\_taskmanager.baseTaskManager method}}

\begin{fulllineitems}
\phantomsection\label{\detokenize{dc_base_taskmanager:src.base_taskmanager.baseTaskManager.create}}\pysiglinewithargsret{\sphinxbfcode{\sphinxupquote{create}}}{}{}
\sphinxAtStartPar
Creates an application project.
To do so, it defines the main folder structure, and creates (or cleans)
the project folder, specified in self.path2project

\end{fulllineitems}

\index{load() (src.base\_taskmanager.baseTaskManager method)@\spxentry{load()}\spxextra{src.base\_taskmanager.baseTaskManager method}}

\begin{fulllineitems}
\phantomsection\label{\detokenize{dc_base_taskmanager:src.base_taskmanager.baseTaskManager.load}}\pysiglinewithargsret{\sphinxbfcode{\sphinxupquote{load}}}{}{}
\sphinxAtStartPar
Loads an existing project, by reading the metadata file in the project
folder.

\sphinxAtStartPar
It can be used to modify file or folder names, or paths, by specifying
the new names/paths in the f\_struct dictionary.

\end{fulllineitems}

\index{setup() (src.base\_taskmanager.baseTaskManager method)@\spxentry{setup()}\spxextra{src.base\_taskmanager.baseTaskManager method}}

\begin{fulllineitems}
\phantomsection\label{\detokenize{dc_base_taskmanager:src.base_taskmanager.baseTaskManager.setup}}\pysiglinewithargsret{\sphinxbfcode{\sphinxupquote{setup}}}{}{}
\sphinxAtStartPar
Sets up the application projetc. To do so, it loads the configuration
file and activates the logger objects.

\end{fulllineitems}


\end{fulllineitems}



\chapter{Task Manager}
\label{\detokenize{dc_task_manager:task-manager}}\label{\detokenize{dc_task_manager::doc}}\phantomsection\label{\detokenize{dc_task_manager:module-src.task_manager}}\index{module@\spxentry{module}!src.task\_manager@\spxentry{src.task\_manager}}\index{src.task\_manager@\spxentry{src.task\_manager}!module@\spxentry{module}}\index{TaskManager (class in src.task\_manager)@\spxentry{TaskManager}\spxextra{class in src.task\_manager}}

\begin{fulllineitems}
\phantomsection\label{\detokenize{dc_task_manager:src.task_manager.TaskManager}}\pysiglinewithargsret{\sphinxbfcode{\sphinxupquote{class\DUrole{w}{  }}}\sphinxcode{\sphinxupquote{src.task\_manager.}}\sphinxbfcode{\sphinxupquote{TaskManager}}}{\emph{\DUrole{n}{path2project}}, \emph{\DUrole{n}{path2source}\DUrole{o}{=}\DUrole{default_value}{None}}, \emph{\DUrole{n}{path2zeroshot}\DUrole{o}{=}\DUrole{default_value}{None}}, \emph{\DUrole{n}{config\_fname}\DUrole{o}{=}\DUrole{default_value}{\textquotesingle{}parameters.yaml\textquotesingle{}}}, \emph{\DUrole{n}{metadata\_fname}\DUrole{o}{=}\DUrole{default_value}{\textquotesingle{}metadata.yaml\textquotesingle{}}}, \emph{\DUrole{n}{set\_logs}\DUrole{o}{=}\DUrole{default_value}{True}}}{}
\sphinxAtStartPar
Bases: {\hyperref[\detokenize{dc_base_taskmanager:src.base_taskmanager.baseTaskManager}]{\sphinxcrossref{\sphinxcode{\sphinxupquote{src.base\_taskmanager.baseTaskManager}}}}}

\sphinxAtStartPar
This class extends the functionality of the baseTaskManager class for a
specific example application

\sphinxAtStartPar
This class inherits from the baseTaskManager class, which provides the
basic method to create, load and setup an application project.

\sphinxAtStartPar
The behavior of this class might depend on the state of the project, in
dictionary self.state, with the following entries:
\begin{itemize}
\item {} 
\sphinxAtStartPar
‘isProject’   : If True, project created. Metadata variables loaded

\item {} \begin{description}
\item[{‘configReady’}] \leavevmode{[}If True, config file succesfully loaded. Datamanager{]}
\sphinxAtStartPar
activated.

\end{description}

\end{itemize}
\index{\_\_init\_\_() (src.task\_manager.TaskManager method)@\spxentry{\_\_init\_\_()}\spxextra{src.task\_manager.TaskManager method}}

\begin{fulllineitems}
\phantomsection\label{\detokenize{dc_task_manager:src.task_manager.TaskManager.__init__}}\pysiglinewithargsret{\sphinxbfcode{\sphinxupquote{\_\_init\_\_}}}{\emph{\DUrole{n}{path2project}}, \emph{\DUrole{n}{path2source}\DUrole{o}{=}\DUrole{default_value}{None}}, \emph{\DUrole{n}{path2zeroshot}\DUrole{o}{=}\DUrole{default_value}{None}}, \emph{\DUrole{n}{config\_fname}\DUrole{o}{=}\DUrole{default_value}{\textquotesingle{}parameters.yaml\textquotesingle{}}}, \emph{\DUrole{n}{metadata\_fname}\DUrole{o}{=}\DUrole{default_value}{\textquotesingle{}metadata.yaml\textquotesingle{}}}, \emph{\DUrole{n}{set\_logs}\DUrole{o}{=}\DUrole{default_value}{True}}}{}
\sphinxAtStartPar
Opens a task manager object.
\begin{quote}\begin{description}
\item[{Parameters}] \leavevmode\begin{itemize}
\item {} 
\sphinxAtStartPar
\sphinxstylestrong{path2project} (\sphinxstyleemphasis{pathlib.Path}) \textendash{} Path to the application project

\item {} 
\sphinxAtStartPar
\sphinxstylestrong{path2source} (\sphinxstyleemphasis{str or pathlib.Path or None (default=None)}) \textendash{} Path to the folder containing the data sources

\item {} 
\sphinxAtStartPar
\sphinxstylestrong{path2zeroshot} (\sphinxstyleemphasis{str or pathlib.Path or None (default=None)}) \textendash{} Path to the folder containing the zero\sphinxhyphen{}shot\sphinxhyphen{}model

\item {} 
\sphinxAtStartPar
\sphinxstylestrong{config\_fname} (\sphinxstyleemphasis{str, optional (default=’parameters.yaml’)}) \textendash{} Name of the configuration file

\item {} 
\sphinxAtStartPar
\sphinxstylestrong{metadata\_fname} (\sphinxstyleemphasis{str or None, optional (default=None)}) \textendash{} Name of the project metadata file.
If None, no metadata file is used.

\item {} 
\sphinxAtStartPar
\sphinxstylestrong{set\_logs} (\sphinxstyleemphasis{bool, optional (default=True)}) \textendash{} If True logger objects are created according to the parameters
specified in the configuration file

\end{itemize}

\end{description}\end{quote}

\end{fulllineitems}

\index{analyze\_keywords() (src.task\_manager.TaskManager method)@\spxentry{analyze\_keywords()}\spxextra{src.task\_manager.TaskManager method}}

\begin{fulllineitems}
\phantomsection\label{\detokenize{dc_task_manager:src.task_manager.TaskManager.analyze_keywords}}\pysiglinewithargsret{\sphinxbfcode{\sphinxupquote{analyze\_keywords}}}{\emph{\DUrole{n}{wt}\DUrole{o}{=}\DUrole{default_value}{2}}}{}
\sphinxAtStartPar
Get a set of positive labels using keyword\sphinxhyphen{}based search
\begin{quote}\begin{description}
\item[{Parameters}] \leavevmode
\sphinxAtStartPar
\sphinxstylestrong{wt} (\sphinxstyleemphasis{float, optional (default=2)}) \textendash{} Weighting factor for the title components. Keyword matches with
title words are weighted by this factor

\end{description}\end{quote}

\end{fulllineitems}

\index{evaluate\_PUmodel() (src.task\_manager.TaskManager method)@\spxentry{evaluate\_PUmodel()}\spxextra{src.task\_manager.TaskManager method}}

\begin{fulllineitems}
\phantomsection\label{\detokenize{dc_task_manager:src.task_manager.TaskManager.evaluate_PUmodel}}\pysiglinewithargsret{\sphinxbfcode{\sphinxupquote{evaluate\_PUmodel}}}{}{}
\sphinxAtStartPar
Evaluate a domain classifiers

\end{fulllineitems}

\index{get\_feedback() (src.task\_manager.TaskManager method)@\spxentry{get\_feedback()}\spxextra{src.task\_manager.TaskManager method}}

\begin{fulllineitems}
\phantomsection\label{\detokenize{dc_task_manager:src.task_manager.TaskManager.get_feedback}}\pysiglinewithargsret{\sphinxbfcode{\sphinxupquote{get\_feedback}}}{}{}
\sphinxAtStartPar
Gets some labels from a user for a selected subset of documents

\end{fulllineitems}

\index{get\_labels\_by\_keywords() (src.task\_manager.TaskManager method)@\spxentry{get\_labels\_by\_keywords()}\spxextra{src.task\_manager.TaskManager method}}

\begin{fulllineitems}
\phantomsection\label{\detokenize{dc_task_manager:src.task_manager.TaskManager.get_labels_by_keywords}}\pysiglinewithargsret{\sphinxbfcode{\sphinxupquote{get\_labels\_by\_keywords}}}{\emph{\DUrole{n}{wt}\DUrole{o}{=}\DUrole{default_value}{2}}, \emph{\DUrole{n}{n\_max}\DUrole{o}{=}\DUrole{default_value}{2000}}, \emph{\DUrole{n}{s\_min}\DUrole{o}{=}\DUrole{default_value}{1}}, \emph{\DUrole{n}{tag}\DUrole{o}{=}\DUrole{default_value}{\textquotesingle{}kwds\textquotesingle{}}}}{}
\sphinxAtStartPar
Get a set of positive labels using keyword\sphinxhyphen{}based search
\begin{quote}\begin{description}
\item[{Parameters}] \leavevmode\begin{itemize}
\item {} 
\sphinxAtStartPar
\sphinxstylestrong{wt} (\sphinxstyleemphasis{float, optional (default=2)}) \textendash{} Weighting factor for the title components. Keyword matches with
title words are weighted by this factor

\item {} 
\sphinxAtStartPar
\sphinxstylestrong{n\_max} (\sphinxstyleemphasis{int or None, optional (defaul=2000)}) \textendash{} Maximum number of elements in the output list. The default is
a huge number that, in practice, means there is no loimit

\item {} 
\sphinxAtStartPar
\sphinxstylestrong{s\_min} (\sphinxstyleemphasis{float, optional (default=1)}) \textendash{} Minimum score. Only elements strictly above s\_min are selected

\item {} 
\sphinxAtStartPar
\sphinxstylestrong{tag} (\sphinxstyleemphasis{str, optional (default=1)}) \textendash{} Name of the output label set.

\end{itemize}

\end{description}\end{quote}

\end{fulllineitems}

\index{get\_labels\_by\_topics() (src.task\_manager.TaskManager method)@\spxentry{get\_labels\_by\_topics()}\spxextra{src.task\_manager.TaskManager method}}

\begin{fulllineitems}
\phantomsection\label{\detokenize{dc_task_manager:src.task_manager.TaskManager.get_labels_by_topics}}\pysiglinewithargsret{\sphinxbfcode{\sphinxupquote{get\_labels\_by\_topics}}}{\emph{\DUrole{n}{topic\_weights}}, \emph{\DUrole{n}{T}}, \emph{\DUrole{n}{df\_metadata}}, \emph{\DUrole{n}{n\_max}\DUrole{o}{=}\DUrole{default_value}{2000}}, \emph{\DUrole{n}{s\_min}\DUrole{o}{=}\DUrole{default_value}{1}}, \emph{\DUrole{n}{tag}\DUrole{o}{=}\DUrole{default_value}{\textquotesingle{}tpcs\textquotesingle{}}}}{}
\sphinxAtStartPar
Get a set of positive labels from a weighted list of topics
\begin{quote}\begin{description}
\item[{Parameters}] \leavevmode\begin{itemize}
\item {} 
\sphinxAtStartPar
\sphinxstylestrong{topic\_weights} (\sphinxstyleemphasis{numpy.array}) \textendash{} Weight of each topic

\item {} 
\sphinxAtStartPar
\sphinxstylestrong{T} (\sphinxstyleemphasis{numpy.ndarray}) \textendash{} Topic matrix

\item {} 
\sphinxAtStartPar
\sphinxstylestrong{df\_metadata} \textendash{} Topic metadata

\item {} 
\sphinxAtStartPar
\sphinxstylestrong{n\_max} (\sphinxstyleemphasis{int or None, optional (defaul=2000)}) \textendash{} Maximum number of elements in the output list. The default is
a huge number that, in practice, means there is no loimit

\item {} 
\sphinxAtStartPar
\sphinxstylestrong{s\_min} (\sphinxstyleemphasis{float, optional (default=1)}) \textendash{} Minimum score. Only elements strictly above s\_min are selected

\item {} 
\sphinxAtStartPar
\sphinxstylestrong{tag} (\sphinxstyleemphasis{str, optional (default=1)}) \textendash{} Name of the output label set.

\end{itemize}

\end{description}\end{quote}

\end{fulllineitems}

\index{get\_labels\_by\_zeroshot() (src.task\_manager.TaskManager method)@\spxentry{get\_labels\_by\_zeroshot()}\spxextra{src.task\_manager.TaskManager method}}

\begin{fulllineitems}
\phantomsection\label{\detokenize{dc_task_manager:src.task_manager.TaskManager.get_labels_by_zeroshot}}\pysiglinewithargsret{\sphinxbfcode{\sphinxupquote{get\_labels\_by\_zeroshot}}}{\emph{\DUrole{n}{n\_max}\DUrole{o}{=}\DUrole{default_value}{2000}}, \emph{\DUrole{n}{s\_min}\DUrole{o}{=}\DUrole{default_value}{0.1}}, \emph{\DUrole{n}{tag}\DUrole{o}{=}\DUrole{default_value}{\textquotesingle{}zeroshot\textquotesingle{}}}}{}
\sphinxAtStartPar
Get a set of positive labels using a zero\sphinxhyphen{}shot classification model
\begin{quote}\begin{description}
\item[{Parameters}] \leavevmode\begin{itemize}
\item {} 
\sphinxAtStartPar
\sphinxstylestrong{n\_max} (\sphinxstyleemphasis{int or None, optional (defaul=2000)}) \textendash{} Maximum number of elements in the output list. The default is
a huge number that, in practice, means there is no loimit

\item {} 
\sphinxAtStartPar
\sphinxstylestrong{s\_min} (\sphinxstyleemphasis{float, optional (default=0.1)}) \textendash{} Minimum score. Only elements strictly above s\_min are selected

\item {} 
\sphinxAtStartPar
\sphinxstylestrong{tag} (\sphinxstyleemphasis{str, optional (default=1)}) \textendash{} Name of the output label set.

\end{itemize}

\end{description}\end{quote}

\end{fulllineitems}

\index{get\_labels\_from\_docs() (src.task\_manager.TaskManager method)@\spxentry{get\_labels\_from\_docs()}\spxextra{src.task\_manager.TaskManager method}}

\begin{fulllineitems}
\phantomsection\label{\detokenize{dc_task_manager:src.task_manager.TaskManager.get_labels_from_docs}}\pysiglinewithargsret{\sphinxbfcode{\sphinxupquote{get\_labels\_from\_docs}}}{\emph{\DUrole{n}{n\_docs}}}{}
\sphinxAtStartPar
Requests feedback about the class of given documents.
\begin{quote}\begin{description}
\item[{Parameters}] \leavevmode
\sphinxAtStartPar
\sphinxstylestrong{selected\_docs} (\sphinxstyleemphasis{pands.DataFrame}) \textendash{} Selected documents

\item[{Returns}] \leavevmode
\sphinxAtStartPar
\sphinxstylestrong{labels} \textendash{} Labels for the given documents, in the same order than the
documents in the input dataframe

\item[{Return type}] \leavevmode
\sphinxAtStartPar
list of boolean

\end{description}\end{quote}

\end{fulllineitems}

\index{import\_labels() (src.task\_manager.TaskManager method)@\spxentry{import\_labels()}\spxextra{src.task\_manager.TaskManager method}}

\begin{fulllineitems}
\phantomsection\label{\detokenize{dc_task_manager:src.task_manager.TaskManager.import_labels}}\pysiglinewithargsret{\sphinxbfcode{\sphinxupquote{import\_labels}}}{}{}
\sphinxAtStartPar
Import labels from file

\end{fulllineitems}

\index{load() (src.task\_manager.TaskManager method)@\spxentry{load()}\spxextra{src.task\_manager.TaskManager method}}

\begin{fulllineitems}
\phantomsection\label{\detokenize{dc_task_manager:src.task_manager.TaskManager.load}}\pysiglinewithargsret{\sphinxbfcode{\sphinxupquote{load}}}{}{}
\sphinxAtStartPar
Extends the load method from the parent class to load the project
corpus and the dataset (if any)

\end{fulllineitems}

\index{load\_corpus() (src.task\_manager.TaskManager method)@\spxentry{load\_corpus()}\spxextra{src.task\_manager.TaskManager method}}

\begin{fulllineitems}
\phantomsection\label{\detokenize{dc_task_manager:src.task_manager.TaskManager.load_corpus}}\pysiglinewithargsret{\sphinxbfcode{\sphinxupquote{load\_corpus}}}{\emph{\DUrole{n}{corpus\_name}}}{}
\sphinxAtStartPar
Loads a dataframe of documents from a given corpus.
\begin{quote}\begin{description}
\item[{Parameters}] \leavevmode
\sphinxAtStartPar
\sphinxstylestrong{corpus\_name} (\sphinxstyleemphasis{str}) \textendash{} Name of the corpus. It should be the name of a folder in
self.path2source

\end{description}\end{quote}

\end{fulllineitems}

\index{load\_labels() (src.task\_manager.TaskManager method)@\spxentry{load\_labels()}\spxextra{src.task\_manager.TaskManager method}}

\begin{fulllineitems}
\phantomsection\label{\detokenize{dc_task_manager:src.task_manager.TaskManager.load_labels}}\pysiglinewithargsret{\sphinxbfcode{\sphinxupquote{load\_labels}}}{\emph{\DUrole{n}{class\_name}}}{}
\sphinxAtStartPar
Load a set of labels and its corresponding dataset (if it exists)
\begin{quote}\begin{description}
\item[{Parameters}] \leavevmode
\sphinxAtStartPar
\sphinxstylestrong{class\_name} (\sphinxstyleemphasis{str}) \textendash{} Name of the target category

\end{description}\end{quote}

\end{fulllineitems}

\index{reevaluate\_model() (src.task\_manager.TaskManager method)@\spxentry{reevaluate\_model()}\spxextra{src.task\_manager.TaskManager method}}

\begin{fulllineitems}
\phantomsection\label{\detokenize{dc_task_manager:src.task_manager.TaskManager.reevaluate_model}}\pysiglinewithargsret{\sphinxbfcode{\sphinxupquote{reevaluate\_model}}}{}{}
\sphinxAtStartPar
Evaluate a domain classifier

\end{fulllineitems}

\index{reset\_labels() (src.task\_manager.TaskManager method)@\spxentry{reset\_labels()}\spxextra{src.task\_manager.TaskManager method}}

\begin{fulllineitems}
\phantomsection\label{\detokenize{dc_task_manager:src.task_manager.TaskManager.reset_labels}}\pysiglinewithargsret{\sphinxbfcode{\sphinxupquote{reset\_labels}}}{\emph{\DUrole{n}{labelset}}}{}
\sphinxAtStartPar
Reset all labels and models associated to a given category
\begin{quote}\begin{description}
\item[{Parameters}] \leavevmode
\sphinxAtStartPar
\sphinxstylestrong{labelset} (\sphinxstyleemphasis{str}) \textendash{} Name of the category to be removed.

\end{description}\end{quote}

\end{fulllineitems}

\index{retrain\_model() (src.task\_manager.TaskManager method)@\spxentry{retrain\_model()}\spxextra{src.task\_manager.TaskManager method}}

\begin{fulllineitems}
\phantomsection\label{\detokenize{dc_task_manager:src.task_manager.TaskManager.retrain_model}}\pysiglinewithargsret{\sphinxbfcode{\sphinxupquote{retrain\_model}}}{}{}
\sphinxAtStartPar
Improves classifier performance using the labels provided by users

\end{fulllineitems}

\index{train\_PUmodel() (src.task\_manager.TaskManager method)@\spxentry{train\_PUmodel()}\spxextra{src.task\_manager.TaskManager method}}

\begin{fulllineitems}
\phantomsection\label{\detokenize{dc_task_manager:src.task_manager.TaskManager.train_PUmodel}}\pysiglinewithargsret{\sphinxbfcode{\sphinxupquote{train\_PUmodel}}}{\emph{\DUrole{n}{max\_imbalance}\DUrole{o}{=}\DUrole{default_value}{3}}, \emph{\DUrole{n}{nmax}\DUrole{o}{=}\DUrole{default_value}{400}}}{}
\sphinxAtStartPar
Train a domain classifiers
\begin{quote}\begin{description}
\item[{Parameters}] \leavevmode\begin{itemize}
\item {} 
\sphinxAtStartPar
\sphinxstylestrong{max\_imbalance} (\sphinxstyleemphasis{int or float or None, optional (default=None)}) \textendash{} Maximum ratio negative vs positive samples. If the ratio in
df\_dataset is higher, the negative class is subsampled.
If None, the original proportions are preserved

\item {} 
\sphinxAtStartPar
\sphinxstylestrong{nmax} (\sphinxstyleemphasis{int or None (defautl=None)}) \textendash{} Maximum size of the whole (train+test) dataset

\end{itemize}

\end{description}\end{quote}

\end{fulllineitems}


\end{fulllineitems}

\index{TaskManagerCMD (class in src.task\_manager)@\spxentry{TaskManagerCMD}\spxextra{class in src.task\_manager}}

\begin{fulllineitems}
\phantomsection\label{\detokenize{dc_task_manager:src.task_manager.TaskManagerCMD}}\pysiglinewithargsret{\sphinxbfcode{\sphinxupquote{class\DUrole{w}{  }}}\sphinxcode{\sphinxupquote{src.task\_manager.}}\sphinxbfcode{\sphinxupquote{TaskManagerCMD}}}{\emph{\DUrole{n}{path2project}}, \emph{\DUrole{n}{path2source}\DUrole{o}{=}\DUrole{default_value}{None}}, \emph{\DUrole{n}{path2zeroshot}\DUrole{o}{=}\DUrole{default_value}{None}}, \emph{\DUrole{n}{config\_fname}\DUrole{o}{=}\DUrole{default_value}{\textquotesingle{}parameters.yaml\textquotesingle{}}}, \emph{\DUrole{n}{metadata\_fname}\DUrole{o}{=}\DUrole{default_value}{\textquotesingle{}metadata.yaml\textquotesingle{}}}, \emph{\DUrole{n}{set\_logs}\DUrole{o}{=}\DUrole{default_value}{True}}}{}
\sphinxAtStartPar
Bases: {\hyperref[\detokenize{dc_task_manager:src.task_manager.TaskManager}]{\sphinxcrossref{\sphinxcode{\sphinxupquote{src.task\_manager.TaskManager}}}}}

\sphinxAtStartPar
Provides extra functionality to the task manager, requesting parameters
from users from a command window.
\index{\_\_init\_\_() (src.task\_manager.TaskManagerCMD method)@\spxentry{\_\_init\_\_()}\spxextra{src.task\_manager.TaskManagerCMD method}}

\begin{fulllineitems}
\phantomsection\label{\detokenize{dc_task_manager:src.task_manager.TaskManagerCMD.__init__}}\pysiglinewithargsret{\sphinxbfcode{\sphinxupquote{\_\_init\_\_}}}{\emph{\DUrole{n}{path2project}}, \emph{\DUrole{n}{path2source}\DUrole{o}{=}\DUrole{default_value}{None}}, \emph{\DUrole{n}{path2zeroshot}\DUrole{o}{=}\DUrole{default_value}{None}}, \emph{\DUrole{n}{config\_fname}\DUrole{o}{=}\DUrole{default_value}{\textquotesingle{}parameters.yaml\textquotesingle{}}}, \emph{\DUrole{n}{metadata\_fname}\DUrole{o}{=}\DUrole{default_value}{\textquotesingle{}metadata.yaml\textquotesingle{}}}, \emph{\DUrole{n}{set\_logs}\DUrole{o}{=}\DUrole{default_value}{True}}}{}
\sphinxAtStartPar
Opens a task manager object.
\begin{quote}\begin{description}
\item[{Parameters}] \leavevmode\begin{itemize}
\item {} 
\sphinxAtStartPar
\sphinxstylestrong{path2project} (\sphinxstyleemphasis{pathlib.Path}) \textendash{} Path to the application project

\item {} 
\sphinxAtStartPar
\sphinxstylestrong{path2source} (\sphinxstyleemphasis{str or pathlib.Path or None (default=None)}) \textendash{} Path to the folder containing the data sources

\item {} 
\sphinxAtStartPar
\sphinxstylestrong{path2zeroshot} (\sphinxstyleemphasis{str or pathlib.Path or None (default=None)}) \textendash{} Path to the folder containing the zero\sphinxhyphen{}shot\sphinxhyphen{}model

\item {} 
\sphinxAtStartPar
\sphinxstylestrong{config\_fname} (\sphinxstyleemphasis{str, optional (default=’parameters.yaml’)}) \textendash{} Name of the configuration file

\item {} 
\sphinxAtStartPar
\sphinxstylestrong{metadata\_fname} (\sphinxstyleemphasis{str or None, optional (default=None)}) \textendash{} Name of the project metadata file.
If None, no metadata file is used.

\item {} 
\sphinxAtStartPar
\sphinxstylestrong{set\_logs} (\sphinxstyleemphasis{bool, optional (default=True)}) \textendash{} If True logger objects are created according to the parameters
specified in the configuration file

\end{itemize}

\end{description}\end{quote}

\end{fulllineitems}

\index{analyze\_keywords() (src.task\_manager.TaskManagerCMD method)@\spxentry{analyze\_keywords()}\spxextra{src.task\_manager.TaskManagerCMD method}}

\begin{fulllineitems}
\phantomsection\label{\detokenize{dc_task_manager:src.task_manager.TaskManagerCMD.analyze_keywords}}\pysiglinewithargsret{\sphinxbfcode{\sphinxupquote{analyze\_keywords}}}{}{}
\sphinxAtStartPar
Get a set of positive labels using keyword\sphinxhyphen{}based search

\end{fulllineitems}

\index{get\_labels\_by\_keywords() (src.task\_manager.TaskManagerCMD method)@\spxentry{get\_labels\_by\_keywords()}\spxextra{src.task\_manager.TaskManagerCMD method}}

\begin{fulllineitems}
\phantomsection\label{\detokenize{dc_task_manager:src.task_manager.TaskManagerCMD.get_labels_by_keywords}}\pysiglinewithargsret{\sphinxbfcode{\sphinxupquote{get\_labels\_by\_keywords}}}{}{}
\sphinxAtStartPar
Get a set of positive labels using keyword\sphinxhyphen{}based search

\end{fulllineitems}

\index{get\_labels\_by\_topics() (src.task\_manager.TaskManagerCMD method)@\spxentry{get\_labels\_by\_topics()}\spxextra{src.task\_manager.TaskManagerCMD method}}

\begin{fulllineitems}
\phantomsection\label{\detokenize{dc_task_manager:src.task_manager.TaskManagerCMD.get_labels_by_topics}}\pysiglinewithargsret{\sphinxbfcode{\sphinxupquote{get\_labels\_by\_topics}}}{}{}
\sphinxAtStartPar
Get a set of positive labels from a weighted list of topics

\end{fulllineitems}

\index{get\_labels\_by\_zeroshot() (src.task\_manager.TaskManagerCMD method)@\spxentry{get\_labels\_by\_zeroshot()}\spxextra{src.task\_manager.TaskManagerCMD method}}

\begin{fulllineitems}
\phantomsection\label{\detokenize{dc_task_manager:src.task_manager.TaskManagerCMD.get_labels_by_zeroshot}}\pysiglinewithargsret{\sphinxbfcode{\sphinxupquote{get\_labels\_by\_zeroshot}}}{}{}
\sphinxAtStartPar
Get a set of positive labels using keyword\sphinxhyphen{}based search

\end{fulllineitems}

\index{get\_labels\_from\_docs() (src.task\_manager.TaskManagerCMD method)@\spxentry{get\_labels\_from\_docs()}\spxextra{src.task\_manager.TaskManagerCMD method}}

\begin{fulllineitems}
\phantomsection\label{\detokenize{dc_task_manager:src.task_manager.TaskManagerCMD.get_labels_from_docs}}\pysiglinewithargsret{\sphinxbfcode{\sphinxupquote{get\_labels\_from\_docs}}}{\emph{\DUrole{n}{selected\_docs}}}{}
\sphinxAtStartPar
Requests feedback about the class of given documents.
\begin{quote}\begin{description}
\item[{Parameters}] \leavevmode
\sphinxAtStartPar
\sphinxstylestrong{selected\_docs} (\sphinxstyleemphasis{pands.DataFrame}) \textendash{} Selected documents

\item[{Returns}] \leavevmode
\sphinxAtStartPar
\sphinxstylestrong{labels} \textendash{} Labels for the given documents, in the same order than the
documents in the input dataframe

\item[{Return type}] \leavevmode
\sphinxAtStartPar
list of boolean

\end{description}\end{quote}

\end{fulllineitems}

\index{train\_PUmodel() (src.task\_manager.TaskManagerCMD method)@\spxentry{train\_PUmodel()}\spxextra{src.task\_manager.TaskManagerCMD method}}

\begin{fulllineitems}
\phantomsection\label{\detokenize{dc_task_manager:src.task_manager.TaskManagerCMD.train_PUmodel}}\pysiglinewithargsret{\sphinxbfcode{\sphinxupquote{train\_PUmodel}}}{}{}
\sphinxAtStartPar
Train a domain classifier

\end{fulllineitems}


\end{fulllineitems}

\index{TaskManagerGUI (class in src.task\_manager)@\spxentry{TaskManagerGUI}\spxextra{class in src.task\_manager}}

\begin{fulllineitems}
\phantomsection\label{\detokenize{dc_task_manager:src.task_manager.TaskManagerGUI}}\pysiglinewithargsret{\sphinxbfcode{\sphinxupquote{class\DUrole{w}{  }}}\sphinxcode{\sphinxupquote{src.task\_manager.}}\sphinxbfcode{\sphinxupquote{TaskManagerGUI}}}{\emph{\DUrole{n}{path2project}}, \emph{\DUrole{n}{path2source}\DUrole{o}{=}\DUrole{default_value}{None}}, \emph{\DUrole{n}{path2zeroshot}\DUrole{o}{=}\DUrole{default_value}{None}}, \emph{\DUrole{n}{config\_fname}\DUrole{o}{=}\DUrole{default_value}{\textquotesingle{}parameters.yaml\textquotesingle{}}}, \emph{\DUrole{n}{metadata\_fname}\DUrole{o}{=}\DUrole{default_value}{\textquotesingle{}metadata.yaml\textquotesingle{}}}, \emph{\DUrole{n}{set\_logs}\DUrole{o}{=}\DUrole{default_value}{True}}}{}
\sphinxAtStartPar
Bases: {\hyperref[\detokenize{dc_task_manager:src.task_manager.TaskManager}]{\sphinxcrossref{\sphinxcode{\sphinxupquote{src.task\_manager.TaskManager}}}}}

\sphinxAtStartPar
Provides extra functionality to the task manager, to be used by the
Graphical User Interface (GUI)
\index{get\_feedback() (src.task\_manager.TaskManagerGUI method)@\spxentry{get\_feedback()}\spxextra{src.task\_manager.TaskManagerGUI method}}

\begin{fulllineitems}
\phantomsection\label{\detokenize{dc_task_manager:src.task_manager.TaskManagerGUI.get_feedback}}\pysiglinewithargsret{\sphinxbfcode{\sphinxupquote{get\_feedback}}}{\emph{\DUrole{n}{idx}}, \emph{\DUrole{n}{labels}}}{}
\sphinxAtStartPar
Gets some labels from a user for a selected subset of documents
\subsubsection*{Notes}

\sphinxAtStartPar
In comparison to the corresponding parent method, STEPS 1 and 2 are
carried out directly through the GUI

\end{fulllineitems}

\index{get\_labels\_by\_keywords() (src.task\_manager.TaskManagerGUI method)@\spxentry{get\_labels\_by\_keywords()}\spxextra{src.task\_manager.TaskManagerGUI method}}

\begin{fulllineitems}
\phantomsection\label{\detokenize{dc_task_manager:src.task_manager.TaskManagerGUI.get_labels_by_keywords}}\pysiglinewithargsret{\sphinxbfcode{\sphinxupquote{get\_labels\_by\_keywords}}}{\emph{\DUrole{n}{keywords}}, \emph{\DUrole{n}{\_tag}}}{}
\sphinxAtStartPar
Get a set of positive labels using keyword\sphinxhyphen{}based search through the
MainWindow

\end{fulllineitems}

\index{get\_suggested\_keywords() (src.task\_manager.TaskManagerGUI method)@\spxentry{get\_suggested\_keywords()}\spxextra{src.task\_manager.TaskManagerGUI method}}

\begin{fulllineitems}
\phantomsection\label{\detokenize{dc_task_manager:src.task_manager.TaskManagerGUI.get_suggested_keywords}}\pysiglinewithargsret{\sphinxbfcode{\sphinxupquote{get\_suggested\_keywords}}}{}{}
\sphinxAtStartPar
Get the list of suggested keywords to showing it in the GUI.
\begin{quote}\begin{description}
\item[{Returns}] \leavevmode
\sphinxAtStartPar
\sphinxstylestrong{suggested\_keywords} \textendash{} List of suggested keywords

\item[{Return type}] \leavevmode
\sphinxAtStartPar
list of str

\end{description}\end{quote}

\end{fulllineitems}

\index{get\_topic\_words() (src.task\_manager.TaskManagerGUI method)@\spxentry{get\_topic\_words()}\spxextra{src.task\_manager.TaskManagerGUI method}}

\begin{fulllineitems}
\phantomsection\label{\detokenize{dc_task_manager:src.task_manager.TaskManagerGUI.get_topic_words}}\pysiglinewithargsret{\sphinxbfcode{\sphinxupquote{get\_topic\_words}}}{\emph{\DUrole{n}{n\_max}}, \emph{\DUrole{n}{s\_min}}}{}
\sphinxAtStartPar
Get a set of positive labels from a weighted list of topics

\end{fulllineitems}

\index{train\_PUmodel() (src.task\_manager.TaskManagerGUI method)@\spxentry{train\_PUmodel()}\spxextra{src.task\_manager.TaskManagerGUI method}}

\begin{fulllineitems}
\phantomsection\label{\detokenize{dc_task_manager:src.task_manager.TaskManagerGUI.train_PUmodel}}\pysiglinewithargsret{\sphinxbfcode{\sphinxupquote{train\_PUmodel}}}{\emph{\DUrole{n}{max\_imabalance}}, \emph{\DUrole{n}{nmax}}}{}
\sphinxAtStartPar
Train a domain classifier

\end{fulllineitems}


\end{fulllineitems}



\chapter{Data Manager}
\label{\detokenize{dc_data_manager:data-manager}}\label{\detokenize{dc_data_manager::doc}}\phantomsection\label{\detokenize{dc_data_manager:module-src.data_manager}}\index{module@\spxentry{module}!src.data\_manager@\spxentry{src.data\_manager}}\index{src.data\_manager@\spxentry{src.data\_manager}!module@\spxentry{module}}\index{DataManager (class in src.data\_manager)@\spxentry{DataManager}\spxextra{class in src.data\_manager}}

\begin{fulllineitems}
\phantomsection\label{\detokenize{dc_data_manager:src.data_manager.DataManager}}\pysiglinewithargsret{\sphinxbfcode{\sphinxupquote{class\DUrole{w}{  }}}\sphinxcode{\sphinxupquote{src.data\_manager.}}\sphinxbfcode{\sphinxupquote{DataManager}}}{\emph{\DUrole{n}{path2source}}, \emph{\DUrole{n}{path2labels}}, \emph{\DUrole{n}{path2datasets}}, \emph{\DUrole{n}{path2models}}, \emph{\DUrole{n}{path2embeddings}\DUrole{o}{=}\DUrole{default_value}{None}}}{}
\sphinxAtStartPar
Bases: \sphinxcode{\sphinxupquote{object}}

\sphinxAtStartPar
This class contains all read / write functionalities required by the
domain\_classification project.

\sphinxAtStartPar
It assumes that source and destination data will be stored in files.
\index{\_\_init\_\_() (src.data\_manager.DataManager method)@\spxentry{\_\_init\_\_()}\spxextra{src.data\_manager.DataManager method}}

\begin{fulllineitems}
\phantomsection\label{\detokenize{dc_data_manager:src.data_manager.DataManager.__init__}}\pysiglinewithargsret{\sphinxbfcode{\sphinxupquote{\_\_init\_\_}}}{\emph{\DUrole{n}{path2source}}, \emph{\DUrole{n}{path2labels}}, \emph{\DUrole{n}{path2datasets}}, \emph{\DUrole{n}{path2models}}, \emph{\DUrole{n}{path2embeddings}\DUrole{o}{=}\DUrole{default_value}{None}}}{}
\sphinxAtStartPar
Initializes the data manager object
\begin{quote}\begin{description}
\item[{Parameters}] \leavevmode\begin{itemize}
\item {} 
\sphinxAtStartPar
\sphinxstylestrong{path2source} (\sphinxstyleemphasis{str or pathlib.Path}) \textendash{} Path to the folder containing all external source data

\item {} 
\sphinxAtStartPar
\sphinxstylestrong{path2labels} (\sphinxstyleemphasis{str or pathlib.Path}) \textendash{} Path to the folder containing sets of labels

\item {} 
\sphinxAtStartPar
\sphinxstylestrong{path2datasets} (\sphinxstyleemphasis{str or pathlib.Path}) \textendash{} Path to the folder containing datasets

\item {} 
\sphinxAtStartPar
\sphinxstylestrong{path2embeddings} (\sphinxstyleemphasis{str or pathlib.Path}) \textendash{} Path to the folder containing the document embeddings

\end{itemize}

\end{description}\end{quote}

\end{fulllineitems}

\index{\_\_weakref\_\_ (src.data\_manager.DataManager attribute)@\spxentry{\_\_weakref\_\_}\spxextra{src.data\_manager.DataManager attribute}}

\begin{fulllineitems}
\phantomsection\label{\detokenize{dc_data_manager:src.data_manager.DataManager.__weakref__}}\pysigline{\sphinxbfcode{\sphinxupquote{\_\_weakref\_\_}}}
\sphinxAtStartPar
list of weak references to the object (if defined)

\end{fulllineitems}

\index{get\_corpus\_list() (src.data\_manager.DataManager method)@\spxentry{get\_corpus\_list()}\spxextra{src.data\_manager.DataManager method}}

\begin{fulllineitems}
\phantomsection\label{\detokenize{dc_data_manager:src.data_manager.DataManager.get_corpus_list}}\pysiglinewithargsret{\sphinxbfcode{\sphinxupquote{get\_corpus\_list}}}{}{}
\sphinxAtStartPar
Returns the list of available corpus

\end{fulllineitems}

\index{get\_dataset\_list() (src.data\_manager.DataManager method)@\spxentry{get\_dataset\_list()}\spxextra{src.data\_manager.DataManager method}}

\begin{fulllineitems}
\phantomsection\label{\detokenize{dc_data_manager:src.data_manager.DataManager.get_dataset_list}}\pysiglinewithargsret{\sphinxbfcode{\sphinxupquote{get\_dataset\_list}}}{}{}
\sphinxAtStartPar
Returns the list of available datasets

\end{fulllineitems}

\index{get\_keywords\_list() (src.data\_manager.DataManager method)@\spxentry{get\_keywords\_list()}\spxextra{src.data\_manager.DataManager method}}

\begin{fulllineitems}
\phantomsection\label{\detokenize{dc_data_manager:src.data_manager.DataManager.get_keywords_list}}\pysiglinewithargsret{\sphinxbfcode{\sphinxupquote{get\_keywords\_list}}}{}{}
\sphinxAtStartPar
Returns a list of IA\sphinxhyphen{}related keywords
\begin{quote}\begin{description}
\item[{Returns}] \leavevmode
\sphinxAtStartPar
\sphinxstylestrong{keywords} \textendash{} A list of keywords

\item[{Return type}] \leavevmode
\sphinxAtStartPar
list

\end{description}\end{quote}

\end{fulllineitems}

\index{get\_labelset\_list() (src.data\_manager.DataManager method)@\spxentry{get\_labelset\_list()}\spxextra{src.data\_manager.DataManager method}}

\begin{fulllineitems}
\phantomsection\label{\detokenize{dc_data_manager:src.data_manager.DataManager.get_labelset_list}}\pysiglinewithargsret{\sphinxbfcode{\sphinxupquote{get\_labelset\_list}}}{}{}
\sphinxAtStartPar
Returns the list of available labels

\end{fulllineitems}

\index{get\_model\_list() (src.data\_manager.DataManager method)@\spxentry{get\_model\_list()}\spxextra{src.data\_manager.DataManager method}}

\begin{fulllineitems}
\phantomsection\label{\detokenize{dc_data_manager:src.data_manager.DataManager.get_model_list}}\pysiglinewithargsret{\sphinxbfcode{\sphinxupquote{get\_model\_list}}}{}{}
\sphinxAtStartPar
Returns the list of available models

\end{fulllineitems}

\index{import\_labels() (src.data\_manager.DataManager method)@\spxentry{import\_labels()}\spxextra{src.data\_manager.DataManager method}}

\begin{fulllineitems}
\phantomsection\label{\detokenize{dc_data_manager:src.data_manager.DataManager.import_labels}}\pysiglinewithargsret{\sphinxbfcode{\sphinxupquote{import\_labels}}}{\emph{\DUrole{n}{ids\_corpus}\DUrole{o}{=}\DUrole{default_value}{None}}, \emph{\DUrole{n}{tag}\DUrole{o}{=}\DUrole{default_value}{\textquotesingle{}imported\textquotesingle{}}}}{}
\sphinxAtStartPar
Loads a subcorpus of positive labels from file.
\begin{quote}\begin{description}
\item[{Parameters}] \leavevmode\begin{itemize}
\item {} 
\sphinxAtStartPar
\sphinxstylestrong{ids\_corpus} (\sphinxstyleemphasis{list}) \textendash{} List of ids of the documents in the corpus. Only the labels with
ids in ids\_corpus are imported and saved into the output file.

\item {} 
\sphinxAtStartPar
\sphinxstylestrong{tag} (\sphinxstyleemphasis{str, optional (default=”imported”)}) \textendash{} Name for the category defined by the positive labels.

\end{itemize}

\item[{Returns}] \leavevmode
\sphinxAtStartPar
\sphinxstylestrong{df\_labels} \textendash{} Dataframe of labels, with two columns: id and class.
id identifies the document corresponding to the label.
class identifies the class. All documents are assumed to be class 1

\item[{Return type}] \leavevmode
\sphinxAtStartPar
pandas.DataFrame

\end{description}\end{quote}

\end{fulllineitems}

\index{load\_corpus() (src.data\_manager.DataManager method)@\spxentry{load\_corpus()}\spxextra{src.data\_manager.DataManager method}}

\begin{fulllineitems}
\phantomsection\label{\detokenize{dc_data_manager:src.data_manager.DataManager.load_corpus}}\pysiglinewithargsret{\sphinxbfcode{\sphinxupquote{load\_corpus}}}{\emph{\DUrole{n}{corpus\_name}}}{}
\sphinxAtStartPar
Loads a dataframe of documents from a given corpus.
\begin{quote}\begin{description}
\item[{Parameters}] \leavevmode
\sphinxAtStartPar
\sphinxstylestrong{corpus\_name} (\sphinxstyleemphasis{str}) \textendash{} Name of the corpus. It should be the name of a folder in
self.path2source

\end{description}\end{quote}

\end{fulllineitems}

\index{load\_dataset() (src.data\_manager.DataManager method)@\spxentry{load\_dataset()}\spxextra{src.data\_manager.DataManager method}}

\begin{fulllineitems}
\phantomsection\label{\detokenize{dc_data_manager:src.data_manager.DataManager.load_dataset}}\pysiglinewithargsret{\sphinxbfcode{\sphinxupquote{load\_dataset}}}{\emph{\DUrole{n}{tag}\DUrole{o}{=}\DUrole{default_value}{\textquotesingle{}\textquotesingle{}}}}{}
\sphinxAtStartPar
Loads a labeled dataset of documents in the format required by the
classifier modules
\begin{quote}\begin{description}
\item[{Parameters}] \leavevmode
\sphinxAtStartPar
\sphinxstylestrong{tag} (\sphinxstyleemphasis{str, optional (default=””)}) \textendash{} Name of the dataset

\end{description}\end{quote}

\end{fulllineitems}

\index{load\_labels() (src.data\_manager.DataManager method)@\spxentry{load\_labels()}\spxextra{src.data\_manager.DataManager method}}

\begin{fulllineitems}
\phantomsection\label{\detokenize{dc_data_manager:src.data_manager.DataManager.load_labels}}\pysiglinewithargsret{\sphinxbfcode{\sphinxupquote{load\_labels}}}{\emph{\DUrole{n}{tag}\DUrole{o}{=}\DUrole{default_value}{\textquotesingle{}\textquotesingle{}}}}{}
\sphinxAtStartPar
Loads a set or PU labels

\end{fulllineitems}

\index{load\_topics() (src.data\_manager.DataManager method)@\spxentry{load\_topics()}\spxextra{src.data\_manager.DataManager method}}

\begin{fulllineitems}
\phantomsection\label{\detokenize{dc_data_manager:src.data_manager.DataManager.load_topics}}\pysiglinewithargsret{\sphinxbfcode{\sphinxupquote{load\_topics}}}{}{}
\sphinxAtStartPar
Loads a topic matrix for a specific corpus

\end{fulllineitems}

\index{reset\_labels() (src.data\_manager.DataManager method)@\spxentry{reset\_labels()}\spxextra{src.data\_manager.DataManager method}}

\begin{fulllineitems}
\phantomsection\label{\detokenize{dc_data_manager:src.data_manager.DataManager.reset_labels}}\pysiglinewithargsret{\sphinxbfcode{\sphinxupquote{reset\_labels}}}{\emph{\DUrole{n}{tag}\DUrole{o}{=}\DUrole{default_value}{\textquotesingle{}\textquotesingle{}}}}{}
\sphinxAtStartPar
Delete all files related to a given class
\begin{quote}\begin{description}
\item[{Parameters}] \leavevmode
\sphinxAtStartPar
\sphinxstylestrong{tag} (\sphinxstyleemphasis{str, optional (default=””)}) \textendash{} Name of the class to be removed

\end{description}\end{quote}

\end{fulllineitems}

\index{save\_dataset() (src.data\_manager.DataManager method)@\spxentry{save\_dataset()}\spxextra{src.data\_manager.DataManager method}}

\begin{fulllineitems}
\phantomsection\label{\detokenize{dc_data_manager:src.data_manager.DataManager.save_dataset}}\pysiglinewithargsret{\sphinxbfcode{\sphinxupquote{save\_dataset}}}{\emph{\DUrole{n}{df\_dataset}}, \emph{\DUrole{n}{tag}\DUrole{o}{=}\DUrole{default_value}{\textquotesingle{}\textquotesingle{}}}, \emph{\DUrole{n}{save\_csv}\DUrole{o}{=}\DUrole{default_value}{False}}}{}
\sphinxAtStartPar
Save dataset in input dataframe in a feather file.
\begin{quote}\begin{description}
\item[{Parameters}] \leavevmode\begin{itemize}
\item {} 
\sphinxAtStartPar
\sphinxstylestrong{df\_dataset} (\sphinxstyleemphasis{pandas.DataFrame}) \textendash{} Dataset to save

\item {} 
\sphinxAtStartPar
\sphinxstylestrong{tag} (\sphinxstyleemphasis{str, optional (default=””)}) \textendash{} Optional string to add to the output file name.

\item {} 
\sphinxAtStartPar
\sphinxstylestrong{save\_csv} (\sphinxstyleemphasis{boolean, optional (default=False)}) \textendash{} If True, the dataset is saved in csv format too

\end{itemize}

\end{description}\end{quote}

\end{fulllineitems}


\end{fulllineitems}



\chapter{Query Manager}
\label{\detokenize{dc_query_manager:query-manager}}\label{\detokenize{dc_query_manager::doc}}\phantomsection\label{\detokenize{dc_query_manager:module-src.query_manager}}\index{module@\spxentry{module}!src.query\_manager@\spxentry{src.query\_manager}}\index{src.query\_manager@\spxentry{src.query\_manager}!module@\spxentry{module}}\index{QueryManager (class in src.query\_manager)@\spxentry{QueryManager}\spxextra{class in src.query\_manager}}

\begin{fulllineitems}
\phantomsection\label{\detokenize{dc_query_manager:src.query_manager.QueryManager}}\pysigline{\sphinxbfcode{\sphinxupquote{class\DUrole{w}{  }}}\sphinxcode{\sphinxupquote{src.query\_manager.}}\sphinxbfcode{\sphinxupquote{QueryManager}}}
\sphinxAtStartPar
Bases: \sphinxcode{\sphinxupquote{object}}

\sphinxAtStartPar
This class contains all user queries needed by the datamanager.
\index{\_\_init\_\_() (src.query\_manager.QueryManager method)@\spxentry{\_\_init\_\_()}\spxextra{src.query\_manager.QueryManager method}}

\begin{fulllineitems}
\phantomsection\label{\detokenize{dc_query_manager:src.query_manager.QueryManager.__init__}}\pysiglinewithargsret{\sphinxbfcode{\sphinxupquote{\_\_init\_\_}}}{}{}
\sphinxAtStartPar
Initializes the query manager object

\end{fulllineitems}

\index{\_\_weakref\_\_ (src.query\_manager.QueryManager attribute)@\spxentry{\_\_weakref\_\_}\spxextra{src.query\_manager.QueryManager attribute}}

\begin{fulllineitems}
\phantomsection\label{\detokenize{dc_query_manager:src.query_manager.QueryManager.__weakref__}}\pysigline{\sphinxbfcode{\sphinxupquote{\_\_weakref\_\_}}}
\sphinxAtStartPar
list of weak references to the object (if defined)

\end{fulllineitems}

\index{ask\_keywords() (src.query\_manager.QueryManager method)@\spxentry{ask\_keywords()}\spxextra{src.query\_manager.QueryManager method}}

\begin{fulllineitems}
\phantomsection\label{\detokenize{dc_query_manager:src.query_manager.QueryManager.ask_keywords}}\pysiglinewithargsret{\sphinxbfcode{\sphinxupquote{ask\_keywords}}}{\emph{\DUrole{n}{kw\_library}\DUrole{o}{=}\DUrole{default_value}{None}}}{}
\sphinxAtStartPar
Ask the user for a list of keywords.
\begin{quote}\begin{description}
\item[{Parameters}] \leavevmode
\sphinxAtStartPar
\sphinxstylestrong{kw\_library} (\sphinxstyleemphasis{list}) \textendash{} A list of possible keywords

\item[{Returns}] \leavevmode
\sphinxAtStartPar
\sphinxstylestrong{keywords} \textendash{} List of keywords

\item[{Return type}] \leavevmode
\sphinxAtStartPar
list of str

\end{description}\end{quote}

\end{fulllineitems}

\index{ask\_label() (src.query\_manager.QueryManager method)@\spxentry{ask\_label()}\spxextra{src.query\_manager.QueryManager method}}

\begin{fulllineitems}
\phantomsection\label{\detokenize{dc_query_manager:src.query_manager.QueryManager.ask_label}}\pysiglinewithargsret{\sphinxbfcode{\sphinxupquote{ask\_label}}}{}{}
\sphinxAtStartPar
Ask the user for a single binary label
\begin{quote}\begin{description}
\item[{Returns}] \leavevmode
\sphinxAtStartPar
\sphinxstylestrong{label} \textendash{} Binary value read from the standard input

\item[{Return type}] \leavevmode
\sphinxAtStartPar
int

\end{description}\end{quote}

\end{fulllineitems}

\index{ask\_label\_tag() (src.query\_manager.QueryManager method)@\spxentry{ask\_label\_tag()}\spxextra{src.query\_manager.QueryManager method}}

\begin{fulllineitems}
\phantomsection\label{\detokenize{dc_query_manager:src.query_manager.QueryManager.ask_label_tag}}\pysiglinewithargsret{\sphinxbfcode{\sphinxupquote{ask\_label\_tag}}}{}{}
\sphinxAtStartPar
Ask the user for a tag to compose the label file name.
\begin{quote}\begin{description}
\item[{Returns}] \leavevmode
\sphinxAtStartPar
\sphinxstylestrong{keywords} \textendash{} List of keywords

\item[{Return type}] \leavevmode
\sphinxAtStartPar
list of str

\end{description}\end{quote}

\end{fulllineitems}

\index{ask\_labels() (src.query\_manager.QueryManager method)@\spxentry{ask\_labels()}\spxextra{src.query\_manager.QueryManager method}}

\begin{fulllineitems}
\phantomsection\label{\detokenize{dc_query_manager:src.query_manager.QueryManager.ask_labels}}\pysiglinewithargsret{\sphinxbfcode{\sphinxupquote{ask\_labels}}}{}{}
\sphinxAtStartPar
Ask the user for a weighted list of labels related to some documents
\begin{quote}\begin{description}
\item[{Returns}] \leavevmode
\sphinxAtStartPar
\sphinxstylestrong{labels} \textendash{} List of labels

\item[{Return type}] \leavevmode
\sphinxAtStartPar
list of int

\end{description}\end{quote}

\end{fulllineitems}

\index{ask\_topics() (src.query\_manager.QueryManager method)@\spxentry{ask\_topics()}\spxextra{src.query\_manager.QueryManager method}}

\begin{fulllineitems}
\phantomsection\label{\detokenize{dc_query_manager:src.query_manager.QueryManager.ask_topics}}\pysiglinewithargsret{\sphinxbfcode{\sphinxupquote{ask\_topics}}}{\emph{\DUrole{n}{topic\_words}}}{}
\sphinxAtStartPar
Ask the user for a weighted list of topics
\begin{quote}\begin{description}
\item[{Parameters}] \leavevmode
\sphinxAtStartPar
\sphinxstylestrong{topic\_words} (\sphinxstyleemphasis{list of str}) \textendash{} List of the main words from each topic

\item[{Returns}] \leavevmode
\sphinxAtStartPar
\sphinxstylestrong{tw} \textendash{} Dictionary of topics: weights

\item[{Return type}] \leavevmode
\sphinxAtStartPar
dict

\end{description}\end{quote}

\end{fulllineitems}

\index{ask\_value() (src.query\_manager.QueryManager method)@\spxentry{ask\_value()}\spxextra{src.query\_manager.QueryManager method}}

\begin{fulllineitems}
\phantomsection\label{\detokenize{dc_query_manager:src.query_manager.QueryManager.ask_value}}\pysiglinewithargsret{\sphinxbfcode{\sphinxupquote{ask\_value}}}{\emph{\DUrole{n}{query=\textquotesingle{}Write value: \textquotesingle{}}}, \emph{\DUrole{n}{convert\_to=\textless{}class \textquotesingle{}str\textquotesingle{}\textgreater{}}}, \emph{\DUrole{n}{default=None}}}{}
\sphinxAtStartPar
Ask user for a value
\begin{quote}\begin{description}
\item[{Parameters}] \leavevmode\begin{itemize}
\item {} 
\sphinxAtStartPar
\sphinxstylestrong{query} (\sphinxstyleemphasis{str, optional (default value=”Write value: “)}) \textendash{} Text to print

\item {} 
\sphinxAtStartPar
\sphinxstylestrong{convert\_to} (\sphinxstyleemphasis{function, optional (default=str)}) \textendash{} A function to apply to the selected value. It can be used, for
instance, for type conversion.

\item {} 
\sphinxAtStartPar
\sphinxstylestrong{default} (\sphinxstyleemphasis{optional (default=None)}) \textendash{} Default value to return if an empty value is returned

\end{itemize}

\item[{Returns}] \leavevmode
\sphinxAtStartPar
The returned value is equal to conver\_to(x), where x is the string
introduced by the user (if any) or the default value.

\item[{Return type}] \leavevmode
\sphinxAtStartPar
value

\end{description}\end{quote}

\end{fulllineitems}

\index{confirm() (src.query\_manager.QueryManager method)@\spxentry{confirm()}\spxextra{src.query\_manager.QueryManager method}}

\begin{fulllineitems}
\phantomsection\label{\detokenize{dc_query_manager:src.query_manager.QueryManager.confirm}}\pysiglinewithargsret{\sphinxbfcode{\sphinxupquote{confirm}}}{}{}
\sphinxAtStartPar
Ask the user for confirmation
\begin{quote}\begin{description}
\item[{Return type}] \leavevmode
\sphinxAtStartPar
True if the user inputs ‘y’, False otherwise

\end{description}\end{quote}

\end{fulllineitems}


\end{fulllineitems}



\chapter{Classifier}
\label{\detokenize{dc_classifier:classifier}}\label{\detokenize{dc_classifier::doc}}\phantomsection\label{\detokenize{dc_classifier:module-src.domain_classifier.classifier}}\index{module@\spxentry{module}!src.domain\_classifier.classifier@\spxentry{src.domain\_classifier.classifier}}\index{src.domain\_classifier.classifier@\spxentry{src.domain\_classifier.classifier}!module@\spxentry{module}}\index{CorpusClassifier (class in src.domain\_classifier.classifier)@\spxentry{CorpusClassifier}\spxextra{class in src.domain\_classifier.classifier}}

\begin{fulllineitems}
\phantomsection\label{\detokenize{dc_classifier:src.domain_classifier.classifier.CorpusClassifier}}\pysiglinewithargsret{\sphinxbfcode{\sphinxupquote{class\DUrole{w}{  }}}\sphinxcode{\sphinxupquote{src.domain\_classifier.classifier.}}\sphinxbfcode{\sphinxupquote{CorpusClassifier}}}{\emph{\DUrole{n}{df\_dataset}}, \emph{\DUrole{n}{path2transformers}\DUrole{o}{=}\DUrole{default_value}{\textquotesingle{}.\textquotesingle{}}}, \emph{\DUrole{n}{use\_cuda}\DUrole{o}{=}\DUrole{default_value}{True}}}{}
\sphinxAtStartPar
Bases: \sphinxcode{\sphinxupquote{object}}

\sphinxAtStartPar
A container of corpus classification methods
\index{AL\_sample() (src.domain\_classifier.classifier.CorpusClassifier method)@\spxentry{AL\_sample()}\spxextra{src.domain\_classifier.classifier.CorpusClassifier method}}

\begin{fulllineitems}
\phantomsection\label{\detokenize{dc_classifier:src.domain_classifier.classifier.CorpusClassifier.AL_sample}}\pysiglinewithargsret{\sphinxbfcode{\sphinxupquote{AL\_sample}}}{\emph{\DUrole{n}{n\_samples}\DUrole{o}{=}\DUrole{default_value}{5}}}{}
\sphinxAtStartPar
Returns a given number of samples for active learning (AL)
\begin{quote}\begin{description}
\item[{Parameters}] \leavevmode
\sphinxAtStartPar
\sphinxstylestrong{n\_samples} (\sphinxstyleemphasis{int, optional (default=5)}) \textendash{} Number of samples to return

\item[{Returns}] \leavevmode
\sphinxAtStartPar
\sphinxstylestrong{df\_out} \textendash{} Selected samples

\item[{Return type}] \leavevmode
\sphinxAtStartPar
pandas.dataFrame

\end{description}\end{quote}

\end{fulllineitems}

\index{\_\_init\_\_() (src.domain\_classifier.classifier.CorpusClassifier method)@\spxentry{\_\_init\_\_()}\spxextra{src.domain\_classifier.classifier.CorpusClassifier method}}

\begin{fulllineitems}
\phantomsection\label{\detokenize{dc_classifier:src.domain_classifier.classifier.CorpusClassifier.__init__}}\pysiglinewithargsret{\sphinxbfcode{\sphinxupquote{\_\_init\_\_}}}{\emph{\DUrole{n}{df\_dataset}}, \emph{\DUrole{n}{path2transformers}\DUrole{o}{=}\DUrole{default_value}{\textquotesingle{}.\textquotesingle{}}}, \emph{\DUrole{n}{use\_cuda}\DUrole{o}{=}\DUrole{default_value}{True}}}{}
\sphinxAtStartPar
Initializes a preprocessor object
\begin{quote}\begin{description}
\item[{Parameters}] \leavevmode\begin{itemize}
\item {} 
\sphinxAtStartPar
\sphinxstylestrong{df\_dataset} (\sphinxstyleemphasis{pandas.DataFrame}) \textendash{} Dataset with text and labels. It must contain at least two columns
with names “text” and “labels”, with the input and the target
labels for classification.

\item {} 
\sphinxAtStartPar
\sphinxstylestrong{path2transformers} (\sphinxstyleemphasis{pathlib.Path or str, optional (default=”.”)}) \textendash{} Path to the folder that will store all files produced by the
simpletransformers library.
Default value is “.”.

\item {} 
\sphinxAtStartPar
\sphinxstylestrong{use\_cuda} (\sphinxstyleemphasis{boolean, optional (default=True)}) \textendash{} If true, GPU will be used, if available.

\end{itemize}

\end{description}\end{quote}
\subsubsection*{Notes}

\sphinxAtStartPar
Be aware that the simpletransformers library produces several folders,
with some large files. You might like to use a value of
path2transformers other than ‘.’.

\end{fulllineitems}

\index{\_\_weakref\_\_ (src.domain\_classifier.classifier.CorpusClassifier attribute)@\spxentry{\_\_weakref\_\_}\spxextra{src.domain\_classifier.classifier.CorpusClassifier attribute}}

\begin{fulllineitems}
\phantomsection\label{\detokenize{dc_classifier:src.domain_classifier.classifier.CorpusClassifier.__weakref__}}\pysigline{\sphinxbfcode{\sphinxupquote{\_\_weakref\_\_}}}
\sphinxAtStartPar
list of weak references to the object (if defined)

\end{fulllineitems}

\index{annotate() (src.domain\_classifier.classifier.CorpusClassifier method)@\spxentry{annotate()}\spxextra{src.domain\_classifier.classifier.CorpusClassifier method}}

\begin{fulllineitems}
\phantomsection\label{\detokenize{dc_classifier:src.domain_classifier.classifier.CorpusClassifier.annotate}}\pysiglinewithargsret{\sphinxbfcode{\sphinxupquote{annotate}}}{\emph{\DUrole{n}{idx}}, \emph{\DUrole{n}{labels}}, \emph{\DUrole{n}{col}\DUrole{o}{=}\DUrole{default_value}{\textquotesingle{}annotations\textquotesingle{}}}}{}
\sphinxAtStartPar
Annotate the given labels in the given positions
\begin{quote}\begin{description}
\item[{Parameters}] \leavevmode\begin{itemize}
\item {} 
\sphinxAtStartPar
\sphinxstylestrong{idx} (\sphinxstyleemphasis{list of int}) \textendash{} Rows to locate the labels.

\item {} 
\sphinxAtStartPar
\sphinxstylestrong{labels} (\sphinxstyleemphasis{list of int}) \textendash{} Labels to annotate

\item {} 
\sphinxAtStartPar
\sphinxstylestrong{col} (\sphinxstyleemphasis{str, optional (default = ‘annotations’)}) \textendash{} Column in the dataframe where the labels will be annotated. If it
does not exist, it is created.

\end{itemize}

\end{description}\end{quote}

\end{fulllineitems}

\index{eval\_model() (src.domain\_classifier.classifier.CorpusClassifier method)@\spxentry{eval\_model()}\spxextra{src.domain\_classifier.classifier.CorpusClassifier method}}

\begin{fulllineitems}
\phantomsection\label{\detokenize{dc_classifier:src.domain_classifier.classifier.CorpusClassifier.eval_model}}\pysiglinewithargsret{\sphinxbfcode{\sphinxupquote{eval\_model}}}{\emph{\DUrole{n}{tag\_score}\DUrole{o}{=}\DUrole{default_value}{\textquotesingle{}score\textquotesingle{}}}}{}
\sphinxAtStartPar
Compute predictions of the classification model over the input dataset
and compute performance metrics.
\begin{quote}\begin{description}
\item[{Parameters}] \leavevmode
\sphinxAtStartPar
\sphinxstylestrong{tag\_score} (\sphinxstyleemphasis{str}) \textendash{} Prefix of the score names.
The scores will be save in the columns of self.df\_dataset
containing these scores.

\end{description}\end{quote}
\subsubsection*{Notes}

\sphinxAtStartPar
The use of simpletransformers follows the example code in
\sphinxurl{https://towardsdatascience.com/simple-transformers-introducing-the}\sphinxhyphen{}
easiest\sphinxhyphen{}bert\sphinxhyphen{}roberta\sphinxhyphen{}xlnet\sphinxhyphen{}and\sphinxhyphen{}xlm\sphinxhyphen{}library\sphinxhyphen{}58bf8c59b2a3

\end{fulllineitems}

\index{load\_model() (src.domain\_classifier.classifier.CorpusClassifier method)@\spxentry{load\_model()}\spxextra{src.domain\_classifier.classifier.CorpusClassifier method}}

\begin{fulllineitems}
\phantomsection\label{\detokenize{dc_classifier:src.domain_classifier.classifier.CorpusClassifier.load_model}}\pysiglinewithargsret{\sphinxbfcode{\sphinxupquote{load\_model}}}{}{}
\sphinxAtStartPar
Loads an existing classification model
\begin{quote}\begin{description}
\item[{Return type}] \leavevmode
\sphinxAtStartPar
The loaded model is stored in attribute self.model

\end{description}\end{quote}

\end{fulllineitems}

\index{load\_model\_config() (src.domain\_classifier.classifier.CorpusClassifier method)@\spxentry{load\_model\_config()}\spxextra{src.domain\_classifier.classifier.CorpusClassifier method}}

\begin{fulllineitems}
\phantomsection\label{\detokenize{dc_classifier:src.domain_classifier.classifier.CorpusClassifier.load_model_config}}\pysiglinewithargsret{\sphinxbfcode{\sphinxupquote{load\_model\_config}}}{}{}
\sphinxAtStartPar
Load configuration for model.

\sphinxAtStartPar
If there is no previous configuration, copy it from simpletransformers
ClassificationModel and save it.

\end{fulllineitems}

\index{retrain\_model() (src.domain\_classifier.classifier.CorpusClassifier method)@\spxentry{retrain\_model()}\spxextra{src.domain\_classifier.classifier.CorpusClassifier method}}

\begin{fulllineitems}
\phantomsection\label{\detokenize{dc_classifier:src.domain_classifier.classifier.CorpusClassifier.retrain_model}}\pysiglinewithargsret{\sphinxbfcode{\sphinxupquote{retrain\_model}}}{}{}
\sphinxAtStartPar
Re\sphinxhyphen{}train the classifier model using annotations

\end{fulllineitems}

\index{train\_model() (src.domain\_classifier.classifier.CorpusClassifier method)@\spxentry{train\_model()}\spxextra{src.domain\_classifier.classifier.CorpusClassifier method}}

\begin{fulllineitems}
\phantomsection\label{\detokenize{dc_classifier:src.domain_classifier.classifier.CorpusClassifier.train_model}}\pysiglinewithargsret{\sphinxbfcode{\sphinxupquote{train\_model}}}{\emph{\DUrole{n}{epochs}\DUrole{o}{=}\DUrole{default_value}{3}}, \emph{\DUrole{n}{evaluate}\DUrole{o}{=}\DUrole{default_value}{True}}}{}
\sphinxAtStartPar
Train binary text classification model based on transformers
\subsubsection*{Notes}

\sphinxAtStartPar
The use of simpletransformers follows the example code in
\sphinxurl{https://towardsdatascience.com/simple-transformers-introducing-the}\sphinxhyphen{}
easiest\sphinxhyphen{}bert\sphinxhyphen{}roberta\sphinxhyphen{}xlnet\sphinxhyphen{}and\sphinxhyphen{}xlm\sphinxhyphen{}library\sphinxhyphen{}58bf8c59b2a3

\end{fulllineitems}

\index{train\_test\_split() (src.domain\_classifier.classifier.CorpusClassifier method)@\spxentry{train\_test\_split()}\spxextra{src.domain\_classifier.classifier.CorpusClassifier method}}

\begin{fulllineitems}
\phantomsection\label{\detokenize{dc_classifier:src.domain_classifier.classifier.CorpusClassifier.train_test_split}}\pysiglinewithargsret{\sphinxbfcode{\sphinxupquote{train\_test\_split}}}{\emph{\DUrole{n}{max\_imbalance}\DUrole{o}{=}\DUrole{default_value}{None}}, \emph{\DUrole{n}{nmax}\DUrole{o}{=}\DUrole{default_value}{None}}, \emph{\DUrole{n}{train\_size}\DUrole{o}{=}\DUrole{default_value}{0.6}}, \emph{\DUrole{n}{random\_state}\DUrole{o}{=}\DUrole{default_value}{None}}}{}
\sphinxAtStartPar
Split dataframe dataset into train an test datasets, undersampling
the negative class
\begin{quote}\begin{description}
\item[{Parameters}] \leavevmode\begin{itemize}
\item {} 
\sphinxAtStartPar
\sphinxstylestrong{max\_imbalance} (\sphinxstyleemphasis{int or float or None, optional (default=None)}) \textendash{} Maximum ratio negative vs positive samples. If the ratio in
df\_dataset is higher, the negative class is subsampled
If None, the original proportions are preserved

\item {} 
\sphinxAtStartPar
\sphinxstylestrong{nmax} (\sphinxstyleemphasis{int or None (defautl=None)}) \textendash{} Maximum size of the whole (train+test) dataset

\item {} 
\sphinxAtStartPar
\sphinxstylestrong{train\_size} (\sphinxstyleemphasis{float or int (default=0.6)}) \textendash{} Size of the training set.
If float in {[}0.0, 1.0{]}, proportion of the dataset to include in the
train split.
If int, absolute number of train samples.

\item {} 
\sphinxAtStartPar
\sphinxstylestrong{random\_state} (\sphinxstyleemphasis{int or None (default=None)}) \textendash{} Controls the shuffling applied to the data before splitting.
Pass an int for reproducible output across multiple function calls.

\end{itemize}

\item[{Returns}] \leavevmode
\sphinxAtStartPar
\begin{itemize}
\item {} 
\sphinxAtStartPar
\sphinxstyleemphasis{No variables are returned. The dataset dataframe in self.df\_dataset is}

\item {} 
\sphinxAtStartPar
\sphinxstyleemphasis{updated with a new columm ‘train\_test’ taking values} \textendash{} 0: if row is selected for training
1: if row is selected for test
\sphinxhyphen{}1: otherwise

\end{itemize}


\end{description}\end{quote}

\end{fulllineitems}


\end{fulllineitems}



\chapter{Custom Model}
\label{\detokenize{dc_custom_model:custom-model}}\label{\detokenize{dc_custom_model::doc}}\phantomsection\label{\detokenize{dc_custom_model:module-src.domain_classifier.custom_model}}\index{module@\spxentry{module}!src.domain\_classifier.custom\_model@\spxentry{src.domain\_classifier.custom\_model}}\index{src.domain\_classifier.custom\_model@\spxentry{src.domain\_classifier.custom\_model}!module@\spxentry{module}}
\sphinxAtStartPar
A custom classifier based on the RobertaModel class from transformers.

\sphinxAtStartPar
Created on February 2022

\sphinxAtStartPar
@author: José Antonio Espinosa
\index{CustomClassificationHead (class in src.domain\_classifier.custom\_model)@\spxentry{CustomClassificationHead}\spxextra{class in src.domain\_classifier.custom\_model}}

\begin{fulllineitems}
\phantomsection\label{\detokenize{dc_custom_model:src.domain_classifier.custom_model.CustomClassificationHead}}\pysiglinewithargsret{\sphinxbfcode{\sphinxupquote{class\DUrole{w}{  }}}\sphinxcode{\sphinxupquote{src.domain\_classifier.custom\_model.}}\sphinxbfcode{\sphinxupquote{CustomClassificationHead}}}{\emph{\DUrole{n}{classifier\_dropout}\DUrole{o}{=}\DUrole{default_value}{0.1}}, \emph{\DUrole{n}{hidden\_dropout\_prob}\DUrole{o}{=}\DUrole{default_value}{0.1}}, \emph{\DUrole{n}{hidden\_size}\DUrole{o}{=}\DUrole{default_value}{768}}}{}
\sphinxAtStartPar
Bases: \sphinxcode{\sphinxupquote{torch.nn.modules.module.Module}}

\sphinxAtStartPar
Copy of class RobertaClassificationHead
(from transformers.models.roberta.modeling\_roberta)
Head for sentence\sphinxhyphen{}level classification tasks.
\index{\_\_init\_\_() (src.domain\_classifier.custom\_model.CustomClassificationHead method)@\spxentry{\_\_init\_\_()}\spxextra{src.domain\_classifier.custom\_model.CustomClassificationHead method}}

\begin{fulllineitems}
\phantomsection\label{\detokenize{dc_custom_model:src.domain_classifier.custom_model.CustomClassificationHead.__init__}}\pysiglinewithargsret{\sphinxbfcode{\sphinxupquote{\_\_init\_\_}}}{\emph{\DUrole{n}{classifier\_dropout}\DUrole{o}{=}\DUrole{default_value}{0.1}}, \emph{\DUrole{n}{hidden\_dropout\_prob}\DUrole{o}{=}\DUrole{default_value}{0.1}}, \emph{\DUrole{n}{hidden\_size}\DUrole{o}{=}\DUrole{default_value}{768}}}{}
\sphinxAtStartPar
Initializes internal Module state, shared by both nn.Module and ScriptModule.

\end{fulllineitems}

\index{forward() (src.domain\_classifier.custom\_model.CustomClassificationHead method)@\spxentry{forward()}\spxextra{src.domain\_classifier.custom\_model.CustomClassificationHead method}}

\begin{fulllineitems}
\phantomsection\label{\detokenize{dc_custom_model:src.domain_classifier.custom_model.CustomClassificationHead.forward}}\pysiglinewithargsret{\sphinxbfcode{\sphinxupquote{forward}}}{\emph{\DUrole{n}{features}\DUrole{p}{:}\DUrole{w}{  }\DUrole{n}{torch.Tensor}}}{}\begin{quote}\begin{description}
\item[{Parameters}] \leavevmode
\sphinxAtStartPar
\sphinxstylestrong{features} (\sphinxstyleemphasis{Tensor}) \textendash{} The encoded text

\end{description}\end{quote}

\end{fulllineitems}


\end{fulllineitems}

\index{CustomDataset (class in src.domain\_classifier.custom\_model)@\spxentry{CustomDataset}\spxextra{class in src.domain\_classifier.custom\_model}}

\begin{fulllineitems}
\phantomsection\label{\detokenize{dc_custom_model:src.domain_classifier.custom_model.CustomDataset}}\pysiglinewithargsret{\sphinxbfcode{\sphinxupquote{class\DUrole{w}{  }}}\sphinxcode{\sphinxupquote{src.domain\_classifier.custom\_model.}}\sphinxbfcode{\sphinxupquote{CustomDataset}}}{\emph{\DUrole{n}{df}}}{}
\sphinxAtStartPar
Bases: \sphinxcode{\sphinxupquote{torch.utils.data.dataset.Dataset}}
\index{\_\_init\_\_() (src.domain\_classifier.custom\_model.CustomDataset method)@\spxentry{\_\_init\_\_()}\spxextra{src.domain\_classifier.custom\_model.CustomDataset method}}

\begin{fulllineitems}
\phantomsection\label{\detokenize{dc_custom_model:src.domain_classifier.custom_model.CustomDataset.__init__}}\pysiglinewithargsret{\sphinxbfcode{\sphinxupquote{\_\_init\_\_}}}{\emph{\DUrole{n}{df}}}{}
\end{fulllineitems}


\end{fulllineitems}

\index{CustomEncoderLayer (class in src.domain\_classifier.custom\_model)@\spxentry{CustomEncoderLayer}\spxextra{class in src.domain\_classifier.custom\_model}}

\begin{fulllineitems}
\phantomsection\label{\detokenize{dc_custom_model:src.domain_classifier.custom_model.CustomEncoderLayer}}\pysiglinewithargsret{\sphinxbfcode{\sphinxupquote{class\DUrole{w}{  }}}\sphinxcode{\sphinxupquote{src.domain\_classifier.custom\_model.}}\sphinxbfcode{\sphinxupquote{CustomEncoderLayer}}}{\emph{\DUrole{n}{hidden\_act}\DUrole{o}{=}\DUrole{default_value}{\textquotesingle{}gelu\textquotesingle{}}}, \emph{\DUrole{n}{hidden\_size}\DUrole{o}{=}\DUrole{default_value}{768}}, \emph{\DUrole{n}{intermediate\_size}\DUrole{o}{=}\DUrole{default_value}{3072}}, \emph{\DUrole{n}{layer\_norm\_eps}\DUrole{o}{=}\DUrole{default_value}{1e\sphinxhyphen{}05}}, \emph{\DUrole{n}{num\_attention\_heads}\DUrole{o}{=}\DUrole{default_value}{12}}, \emph{\DUrole{n}{num\_hidden\_layers}\DUrole{o}{=}\DUrole{default_value}{1}}}{}
\sphinxAtStartPar
Bases: \sphinxcode{\sphinxupquote{torch.nn.modules.module.Module}}

\sphinxAtStartPar
Custom encoder layer of transformer for classification.
\index{\_\_init\_\_() (src.domain\_classifier.custom\_model.CustomEncoderLayer method)@\spxentry{\_\_init\_\_()}\spxextra{src.domain\_classifier.custom\_model.CustomEncoderLayer method}}

\begin{fulllineitems}
\phantomsection\label{\detokenize{dc_custom_model:src.domain_classifier.custom_model.CustomEncoderLayer.__init__}}\pysiglinewithargsret{\sphinxbfcode{\sphinxupquote{\_\_init\_\_}}}{\emph{\DUrole{n}{hidden\_act}\DUrole{o}{=}\DUrole{default_value}{\textquotesingle{}gelu\textquotesingle{}}}, \emph{\DUrole{n}{hidden\_size}\DUrole{o}{=}\DUrole{default_value}{768}}, \emph{\DUrole{n}{intermediate\_size}\DUrole{o}{=}\DUrole{default_value}{3072}}, \emph{\DUrole{n}{layer\_norm\_eps}\DUrole{o}{=}\DUrole{default_value}{1e\sphinxhyphen{}05}}, \emph{\DUrole{n}{num\_attention\_heads}\DUrole{o}{=}\DUrole{default_value}{12}}, \emph{\DUrole{n}{num\_hidden\_layers}\DUrole{o}{=}\DUrole{default_value}{1}}}{}
\sphinxAtStartPar
Initializes internal Module state, shared by both nn.Module and ScriptModule.

\end{fulllineitems}

\index{forward() (src.domain\_classifier.custom\_model.CustomEncoderLayer method)@\spxentry{forward()}\spxextra{src.domain\_classifier.custom\_model.CustomEncoderLayer method}}

\begin{fulllineitems}
\phantomsection\label{\detokenize{dc_custom_model:src.domain_classifier.custom_model.CustomEncoderLayer.forward}}\pysiglinewithargsret{\sphinxbfcode{\sphinxupquote{forward}}}{\emph{\DUrole{n}{features}\DUrole{p}{:}\DUrole{w}{  }\DUrole{n}{torch.Tensor}}, \emph{\DUrole{n}{mask}\DUrole{p}{:}\DUrole{w}{  }\DUrole{n}{torch.Tensor}}}{}\begin{quote}\begin{description}
\item[{Parameters}] \leavevmode\begin{itemize}
\item {} 
\sphinxAtStartPar
\sphinxstylestrong{features} (\sphinxstyleemphasis{Tensor}) \textendash{} The sequence to the encoder

\item {} 
\sphinxAtStartPar
\sphinxstylestrong{mask} (\sphinxstyleemphasis{Tensor}) \textendash{} The mask for the src keys per batch

\end{itemize}

\end{description}\end{quote}

\end{fulllineitems}


\end{fulllineitems}

\index{CustomModel (class in src.domain\_classifier.custom\_model)@\spxentry{CustomModel}\spxextra{class in src.domain\_classifier.custom\_model}}

\begin{fulllineitems}
\phantomsection\label{\detokenize{dc_custom_model:src.domain_classifier.custom_model.CustomModel}}\pysiglinewithargsret{\sphinxbfcode{\sphinxupquote{class\DUrole{w}{  }}}\sphinxcode{\sphinxupquote{src.domain\_classifier.custom\_model.}}\sphinxbfcode{\sphinxupquote{CustomModel}}}{\emph{\DUrole{n}{config}}, \emph{\DUrole{n}{path\_model}}}{}
\sphinxAtStartPar
Bases: \sphinxcode{\sphinxupquote{torch.nn.modules.module.Module}}

\sphinxAtStartPar
Copy of class RobertaClassificationHead
(from transformers.models.roberta.modeling\_roberta)
Head for sentence\sphinxhyphen{}level classification tasks.
\index{\_\_init\_\_() (src.domain\_classifier.custom\_model.CustomModel method)@\spxentry{\_\_init\_\_()}\spxextra{src.domain\_classifier.custom\_model.CustomModel method}}

\begin{fulllineitems}
\phantomsection\label{\detokenize{dc_custom_model:src.domain_classifier.custom_model.CustomModel.__init__}}\pysiglinewithargsret{\sphinxbfcode{\sphinxupquote{\_\_init\_\_}}}{\emph{\DUrole{n}{config}}, \emph{\DUrole{n}{path\_model}}}{}
\sphinxAtStartPar
Initializes internal Module state, shared by both nn.Module and ScriptModule.

\end{fulllineitems}

\index{create\_data\_loader() (src.domain\_classifier.custom\_model.CustomModel method)@\spxentry{create\_data\_loader()}\spxextra{src.domain\_classifier.custom\_model.CustomModel method}}

\begin{fulllineitems}
\phantomsection\label{\detokenize{dc_custom_model:src.domain_classifier.custom_model.CustomModel.create_data_loader}}\pysiglinewithargsret{\sphinxbfcode{\sphinxupquote{create\_data\_loader}}}{\emph{\DUrole{n}{df}}, \emph{\DUrole{n}{batch\_size}\DUrole{o}{=}\DUrole{default_value}{8}}}{}
\sphinxAtStartPar
Creates a DataLoader from a DataFrame to train/eval model

\end{fulllineitems}

\index{eval\_model() (src.domain\_classifier.custom\_model.CustomModel method)@\spxentry{eval\_model()}\spxextra{src.domain\_classifier.custom\_model.CustomModel method}}

\begin{fulllineitems}
\phantomsection\label{\detokenize{dc_custom_model:src.domain_classifier.custom_model.CustomModel.eval_model}}\pysiglinewithargsret{\sphinxbfcode{\sphinxupquote{eval\_model}}}{\emph{\DUrole{n}{df\_eval}}, \emph{\DUrole{n}{device}\DUrole{o}{=}\DUrole{default_value}{\textquotesingle{}cuda\textquotesingle{}}}}{}
\sphinxAtStartPar
Evaluate trained model

\end{fulllineitems}

\index{forward() (src.domain\_classifier.custom\_model.CustomModel method)@\spxentry{forward()}\spxextra{src.domain\_classifier.custom\_model.CustomModel method}}

\begin{fulllineitems}
\phantomsection\label{\detokenize{dc_custom_model:src.domain_classifier.custom_model.CustomModel.forward}}\pysiglinewithargsret{\sphinxbfcode{\sphinxupquote{forward}}}{\emph{\DUrole{n}{features}\DUrole{p}{:}\DUrole{w}{  }\DUrole{n}{torch.Tensor}}, \emph{\DUrole{n}{mask}\DUrole{p}{:}\DUrole{w}{  }\DUrole{n}{torch.Tensor}}}{}\begin{quote}\begin{description}
\item[{Parameters}] \leavevmode\begin{itemize}
\item {} 
\sphinxAtStartPar
\sphinxstylestrong{features} (\sphinxstyleemphasis{Tensor}) \textendash{} The sequence to the encoder

\item {} 
\sphinxAtStartPar
\sphinxstylestrong{mask} (\sphinxstyleemphasis{Tensor}) \textendash{} The mask for the src keys per batch

\end{itemize}

\end{description}\end{quote}

\end{fulllineitems}

\index{load\_embeddings() (src.domain\_classifier.custom\_model.CustomModel method)@\spxentry{load\_embeddings()}\spxextra{src.domain\_classifier.custom\_model.CustomModel method}}

\begin{fulllineitems}
\phantomsection\label{\detokenize{dc_custom_model:src.domain_classifier.custom_model.CustomModel.load_embeddings}}\pysiglinewithargsret{\sphinxbfcode{\sphinxupquote{load\_embeddings}}}{}{}
\sphinxAtStartPar
Load embeddings layer.

\sphinxAtStartPar
If there is no previous configuration, copy it from simpletransformers
ClassificationModel and save it.

\end{fulllineitems}

\index{train\_model() (src.domain\_classifier.custom\_model.CustomModel method)@\spxentry{train\_model()}\spxextra{src.domain\_classifier.custom\_model.CustomModel method}}

\begin{fulllineitems}
\phantomsection\label{\detokenize{dc_custom_model:src.domain_classifier.custom_model.CustomModel.train_model}}\pysiglinewithargsret{\sphinxbfcode{\sphinxupquote{train\_model}}}{\emph{\DUrole{n}{df\_train}}, \emph{\DUrole{n}{device}\DUrole{o}{=}\DUrole{default_value}{\textquotesingle{}cuda\textquotesingle{}}}}{}
\sphinxAtStartPar
Train the model
\begin{quote}\begin{description}
\item[{Parameters}] \leavevmode\begin{itemize}
\item {} 
\sphinxAtStartPar
\sphinxstylestrong{df\_train} (\sphinxstyleemphasis{DataFrame}) \textendash{} Training dataframe

\item {} 
\sphinxAtStartPar
\sphinxstylestrong{epochs} (\sphinxstyleemphasis{int}) \textendash{} Number of epochs to train model

\end{itemize}

\end{description}\end{quote}

\end{fulllineitems}


\end{fulllineitems}



\chapter{Preprocessor}
\label{\detokenize{dc_preprocessor:preprocessor}}\label{\detokenize{dc_preprocessor::doc}}\phantomsection\label{\detokenize{dc_preprocessor:module-src.domain_classifier.preprocessor}}\index{module@\spxentry{module}!src.domain\_classifier.preprocessor@\spxentry{src.domain\_classifier.preprocessor}}\index{src.domain\_classifier.preprocessor@\spxentry{src.domain\_classifier.preprocessor}!module@\spxentry{module}}\index{CorpusDFProcessor (class in src.domain\_classifier.preprocessor)@\spxentry{CorpusDFProcessor}\spxextra{class in src.domain\_classifier.preprocessor}}

\begin{fulllineitems}
\phantomsection\label{\detokenize{dc_preprocessor:src.domain_classifier.preprocessor.CorpusDFProcessor}}\pysiglinewithargsret{\sphinxbfcode{\sphinxupquote{class\DUrole{w}{  }}}\sphinxcode{\sphinxupquote{src.domain\_classifier.preprocessor.}}\sphinxbfcode{\sphinxupquote{CorpusDFProcessor}}}{\emph{\DUrole{n}{df\_corpus}}, \emph{\DUrole{n}{path2embeddings}\DUrole{o}{=}\DUrole{default_value}{None}}, \emph{\DUrole{n}{path2zeroshot}\DUrole{o}{=}\DUrole{default_value}{None}}}{}
\sphinxAtStartPar
Bases: \sphinxcode{\sphinxupquote{object}}

\sphinxAtStartPar
A container of corpus processing methods.
It assumes that a corpus is given by a dataframe of documents.

\sphinxAtStartPar
Each dataframe must contain three columns:
id: document identifiers
title: document titles
description: body of the document text
\index{\_\_init\_\_() (src.domain\_classifier.preprocessor.CorpusDFProcessor method)@\spxentry{\_\_init\_\_()}\spxextra{src.domain\_classifier.preprocessor.CorpusDFProcessor method}}

\begin{fulllineitems}
\phantomsection\label{\detokenize{dc_preprocessor:src.domain_classifier.preprocessor.CorpusDFProcessor.__init__}}\pysiglinewithargsret{\sphinxbfcode{\sphinxupquote{\_\_init\_\_}}}{\emph{\DUrole{n}{df\_corpus}}, \emph{\DUrole{n}{path2embeddings}\DUrole{o}{=}\DUrole{default_value}{None}}, \emph{\DUrole{n}{path2zeroshot}\DUrole{o}{=}\DUrole{default_value}{None}}}{}
\sphinxAtStartPar
Initializes a preprocessor object
\begin{quote}\begin{description}
\item[{Parameters}] \leavevmode\begin{itemize}
\item {} 
\sphinxAtStartPar
\sphinxstylestrong{df\_corpus} (\sphinxstyleemphasis{pandas.dataFrame}) \textendash{} Input corpus.

\item {} 
\sphinxAtStartPar
\sphinxstylestrong{path2embeddings} (\sphinxstyleemphasis{str or pathlib.Path or None, optional (default=None)}) \textendash{} Path to the folder containing the document embeddings.
If None, no embeddings will be used. Document scores will be based
in word counts

\item {} 
\sphinxAtStartPar
\sphinxstylestrong{path2zeroshot} (\sphinxstyleemphasis{str or pathlib.Path or None, optional (default=None)}) \textendash{} Path to the folder containing the pretrained zero\sphinxhyphen{}shot model
If None, zero\sphinxhyphen{}shot classification will not be available.

\end{itemize}

\end{description}\end{quote}

\end{fulllineitems}

\index{\_\_weakref\_\_ (src.domain\_classifier.preprocessor.CorpusDFProcessor attribute)@\spxentry{\_\_weakref\_\_}\spxextra{src.domain\_classifier.preprocessor.CorpusDFProcessor attribute}}

\begin{fulllineitems}
\phantomsection\label{\detokenize{dc_preprocessor:src.domain_classifier.preprocessor.CorpusDFProcessor.__weakref__}}\pysigline{\sphinxbfcode{\sphinxupquote{\_\_weakref\_\_}}}
\sphinxAtStartPar
list of weak references to the object (if defined)

\end{fulllineitems}

\index{compute\_keyword\_stats() (src.domain\_classifier.preprocessor.CorpusDFProcessor method)@\spxentry{compute\_keyword\_stats()}\spxextra{src.domain\_classifier.preprocessor.CorpusDFProcessor method}}

\begin{fulllineitems}
\phantomsection\label{\detokenize{dc_preprocessor:src.domain_classifier.preprocessor.CorpusDFProcessor.compute_keyword_stats}}\pysiglinewithargsret{\sphinxbfcode{\sphinxupquote{compute\_keyword\_stats}}}{\emph{\DUrole{n}{keywords}}, \emph{\DUrole{n}{wt}\DUrole{o}{=}\DUrole{default_value}{2}}}{}
\sphinxAtStartPar
Computes keyword statistics
\begin{quote}\begin{description}
\item[{Parameters}] \leavevmode\begin{itemize}
\item {} 
\sphinxAtStartPar
\sphinxstylestrong{corpus} (\sphinxstyleemphasis{dataframe}) \textendash{} Dataframe of corpus.

\item {} 
\sphinxAtStartPar
\sphinxstylestrong{keywords} (\sphinxstyleemphasis{list of str}) \textendash{} List of keywords

\end{itemize}

\item[{Returns}] \leavevmode
\sphinxAtStartPar
\begin{itemize}
\item {} 
\sphinxAtStartPar
\sphinxstylestrong{df\_stats} (\sphinxstyleemphasis{dict}) \textendash{} Dictionary of document frequencies per keyword
df\_stats{[}k{]} is the number of docs containing keyword k

\item {} 
\sphinxAtStartPar
\sphinxstylestrong{kf\_stats} (\sphinxstyleemphasis{dict}) \textendash{} Dictionary of keyword frequencies
df\_stats{[}k{]} is the number of times keyword k appers in the corpus

\item {} 
\sphinxAtStartPar
\sphinxstylestrong{wt} (\sphinxstyleemphasis{float, optional (default=2)}) \textendash{} Weighting factor for the title components. Keyword matches with
title words are weighted by this factor

\end{itemize}


\end{description}\end{quote}

\end{fulllineitems}

\index{filter\_by\_keywords() (src.domain\_classifier.preprocessor.CorpusDFProcessor method)@\spxentry{filter\_by\_keywords()}\spxextra{src.domain\_classifier.preprocessor.CorpusDFProcessor method}}

\begin{fulllineitems}
\phantomsection\label{\detokenize{dc_preprocessor:src.domain_classifier.preprocessor.CorpusDFProcessor.filter_by_keywords}}\pysiglinewithargsret{\sphinxbfcode{\sphinxupquote{filter\_by\_keywords}}}{\emph{\DUrole{n}{keywords}}, \emph{\DUrole{n}{wt}\DUrole{o}{=}\DUrole{default_value}{2}}, \emph{\DUrole{n}{n\_max}\DUrole{o}{=}\DUrole{default_value}{1e+100}}, \emph{\DUrole{n}{s\_min}\DUrole{o}{=}\DUrole{default_value}{0}}}{}
\sphinxAtStartPar
Select documents with a significant presence of a given set of keywords
\begin{quote}\begin{description}
\item[{Parameters}] \leavevmode\begin{itemize}
\item {} 
\sphinxAtStartPar
\sphinxstylestrong{keywords} (\sphinxstyleemphasis{list of str}) \textendash{} List of keywords

\item {} 
\sphinxAtStartPar
\sphinxstylestrong{wt} (\sphinxstyleemphasis{float, optional (default=2)}) \textendash{} Weighting factor for the title components. Keyword matches with
title words are weighted by this factor. Not used if
self.path2embeddings is None

\item {} 
\sphinxAtStartPar
\sphinxstylestrong{n\_max} (\sphinxstyleemphasis{int or None, optional (defaul=1e100)}) \textendash{} Maximum number of elements in the output list. The default is
a huge number that, in practice, means there is no loimit

\item {} 
\sphinxAtStartPar
\sphinxstylestrong{s\_min} (\sphinxstyleemphasis{float, optional (default=0)}) \textendash{} Minimum score. Only elements strictly above s\_min are selected

\end{itemize}

\item[{Returns}] \leavevmode
\sphinxAtStartPar
\sphinxstylestrong{ids} \textendash{} List of ids of the selected documents

\item[{Return type}] \leavevmode
\sphinxAtStartPar
list

\end{description}\end{quote}

\end{fulllineitems}

\index{filter\_by\_topics() (src.domain\_classifier.preprocessor.CorpusDFProcessor method)@\spxentry{filter\_by\_topics()}\spxextra{src.domain\_classifier.preprocessor.CorpusDFProcessor method}}

\begin{fulllineitems}
\phantomsection\label{\detokenize{dc_preprocessor:src.domain_classifier.preprocessor.CorpusDFProcessor.filter_by_topics}}\pysiglinewithargsret{\sphinxbfcode{\sphinxupquote{filter\_by\_topics}}}{\emph{\DUrole{n}{T}}, \emph{\DUrole{n}{doc\_ids}}, \emph{\DUrole{n}{topic\_weights}}, \emph{\DUrole{n}{n\_max}\DUrole{o}{=}\DUrole{default_value}{1e+100}}, \emph{\DUrole{n}{s\_min}\DUrole{o}{=}\DUrole{default_value}{0}}}{}
\sphinxAtStartPar
Select documents with a significant presence of a given set of keywords
\begin{quote}\begin{description}
\item[{Parameters}] \leavevmode\begin{itemize}
\item {} 
\sphinxAtStartPar
\sphinxstylestrong{T} (\sphinxstyleemphasis{numpy.ndarray or scipy.sparse}) \textendash{} Topic matrix.

\item {} 
\sphinxAtStartPar
\sphinxstylestrong{doc\_ids} (\sphinxstyleemphasis{array\sphinxhyphen{}like}) \textendash{} Ids of the documents in the topic matrix. doc\_ids{[}i{]} = ‘123’ means
that document with id ‘123’ has topic vector T{[}i{]}

\item {} 
\sphinxAtStartPar
\sphinxstylestrong{topic\_weights} (\sphinxstyleemphasis{dict}) \textendash{} Dictionary \{t\_i: w\_i\}, where t\_i is a topic index and w\_i is the
weight of the topic

\item {} 
\sphinxAtStartPar
\sphinxstylestrong{n\_max} (\sphinxstyleemphasis{int or None, optional (defaul=1e100)}) \textendash{} Maximum number of elements in the output list. The default is
a huge number that, in practice, means there is no loimit

\item {} 
\sphinxAtStartPar
\sphinxstylestrong{s\_min} (\sphinxstyleemphasis{float, optional (default=0)}) \textendash{} Minimum score. Only elements strictly above s\_min are selected

\end{itemize}

\item[{Returns}] \leavevmode
\sphinxAtStartPar
\sphinxstylestrong{ids} \textendash{} List of ids of the selected documents

\item[{Return type}] \leavevmode
\sphinxAtStartPar
list

\end{description}\end{quote}

\end{fulllineitems}

\index{get\_top\_scores() (src.domain\_classifier.preprocessor.CorpusDFProcessor method)@\spxentry{get\_top\_scores()}\spxextra{src.domain\_classifier.preprocessor.CorpusDFProcessor method}}

\begin{fulllineitems}
\phantomsection\label{\detokenize{dc_preprocessor:src.domain_classifier.preprocessor.CorpusDFProcessor.get_top_scores}}\pysiglinewithargsret{\sphinxbfcode{\sphinxupquote{get\_top\_scores}}}{\emph{\DUrole{n}{scores}}, \emph{\DUrole{n}{n\_max}\DUrole{o}{=}\DUrole{default_value}{1e+100}}, \emph{\DUrole{n}{s\_min}\DUrole{o}{=}\DUrole{default_value}{0}}}{}
\sphinxAtStartPar
Select documents from the corpus whose score is strictly above a lower
bound
\begin{quote}\begin{description}
\item[{Parameters}] \leavevmode\begin{itemize}
\item {} 
\sphinxAtStartPar
\sphinxstylestrong{scores} (\sphinxstyleemphasis{array\sphinxhyphen{}like of float}) \textendash{} List of scores. It must be the same size than the number of docs
in the corpus

\item {} 
\sphinxAtStartPar
\sphinxstylestrong{n\_max} (\sphinxstyleemphasis{int or None, optional (defaul=1e100)}) \textendash{} Maximum number of elements in the output list. The default is
a huge number that, in practice, means there is no loimit

\item {} 
\sphinxAtStartPar
\sphinxstylestrong{s\_min} (\sphinxstyleemphasis{float, optional (default=0)}) \textendash{} Minimum score. Only elements strictly above s\_min are selected

\end{itemize}

\end{description}\end{quote}

\end{fulllineitems}

\index{make\_PU\_dataset() (src.domain\_classifier.preprocessor.CorpusDFProcessor method)@\spxentry{make\_PU\_dataset()}\spxextra{src.domain\_classifier.preprocessor.CorpusDFProcessor method}}

\begin{fulllineitems}
\phantomsection\label{\detokenize{dc_preprocessor:src.domain_classifier.preprocessor.CorpusDFProcessor.make_PU_dataset}}\pysiglinewithargsret{\sphinxbfcode{\sphinxupquote{make\_PU\_dataset}}}{\emph{\DUrole{n}{df\_labels}}}{}
\sphinxAtStartPar
Returns the labeled dataframe in the format required by the
CorpusClassifier class
\begin{quote}\begin{description}
\item[{Parameters}] \leavevmode\begin{itemize}
\item {} 
\sphinxAtStartPar
\sphinxstylestrong{df\_corpus} (\sphinxstyleemphasis{pandas.DataFrame}) \textendash{} Text corpus, with at least three columns: id, title and description

\item {} 
\sphinxAtStartPar
\sphinxstylestrong{df\_labels} (\sphinxstyleemphasis{pandas.DataFrame}) \textendash{} Dataframe of positive labels. It should contain column id. All
labels are assumed to be positive

\end{itemize}

\item[{Returns}] \leavevmode
\sphinxAtStartPar
\sphinxstylestrong{df\_dataset} \textendash{} A pandas dataframe with three columns: id, text and labels.

\item[{Return type}] \leavevmode
\sphinxAtStartPar
pandas.DataFrame

\end{description}\end{quote}

\end{fulllineitems}

\index{make\_pos\_labels\_df() (src.domain\_classifier.preprocessor.CorpusDFProcessor method)@\spxentry{make\_pos\_labels\_df()}\spxextra{src.domain\_classifier.preprocessor.CorpusDFProcessor method}}

\begin{fulllineitems}
\phantomsection\label{\detokenize{dc_preprocessor:src.domain_classifier.preprocessor.CorpusDFProcessor.make_pos_labels_df}}\pysiglinewithargsret{\sphinxbfcode{\sphinxupquote{make\_pos\_labels\_df}}}{\emph{\DUrole{n}{ids}}}{}
\sphinxAtStartPar
Returns a dataframe with the given ids and a single, all\sphinxhyphen{}ones column
\begin{quote}\begin{description}
\item[{Parameters}] \leavevmode
\sphinxAtStartPar
\sphinxstylestrong{ids} (\sphinxstyleemphasis{array\sphinxhyphen{}like}) \textendash{} Values for the column ‘ids’

\item[{Returns}] \leavevmode
\sphinxAtStartPar
\sphinxstylestrong{df\_labels} \textendash{} A dataframe with two columns: ‘id’ and ‘class’. All values in
class column are equal to one.

\item[{Return type}] \leavevmode
\sphinxAtStartPar
pandas.DataFrame

\end{description}\end{quote}

\end{fulllineitems}

\index{remove\_docs\_from\_topics() (src.domain\_classifier.preprocessor.CorpusDFProcessor method)@\spxentry{remove\_docs\_from\_topics()}\spxextra{src.domain\_classifier.preprocessor.CorpusDFProcessor method}}

\begin{fulllineitems}
\phantomsection\label{\detokenize{dc_preprocessor:src.domain_classifier.preprocessor.CorpusDFProcessor.remove_docs_from_topics}}\pysiglinewithargsret{\sphinxbfcode{\sphinxupquote{remove\_docs\_from\_topics}}}{\emph{\DUrole{n}{T}}, \emph{\DUrole{n}{df\_metadata}}, \emph{\DUrole{n}{col\_id}\DUrole{o}{=}\DUrole{default_value}{\textquotesingle{}id\textquotesingle{}}}}{}
\sphinxAtStartPar
Removes, from a given topic\sphinxhyphen{}document matrix and its corresponding
metadata dataframe, all documents that do not belong to the corpus
\begin{quote}\begin{description}
\item[{Parameters}] \leavevmode\begin{itemize}
\item {} 
\sphinxAtStartPar
\sphinxstylestrong{T} (\sphinxstyleemphasis{numpy.ndarray or scipy.sparse}) \textendash{} Topic matrix (one column per topic)

\item {} 
\sphinxAtStartPar
\sphinxstylestrong{df\_metadata} (\sphinxstyleemphasis{pandas.DataFrame}) \textendash{} Dataframe of metadata. It must include a column with document ids

\item {} 
\sphinxAtStartPar
\sphinxstylestrong{col\_id} (\sphinxstyleemphasis{str, optional (default=’id’)}) \textendash{} Name of the column containing the document ids in df\_metadata

\end{itemize}

\item[{Returns}] \leavevmode
\sphinxAtStartPar
\begin{itemize}
\item {} 
\sphinxAtStartPar
\sphinxstylestrong{T\_out} (\sphinxstyleemphasis{numpy.ndarray or scipy.sparse}) \textendash{} Reduced topic matrix (after document removal)

\item {} 
\sphinxAtStartPar
\sphinxstylestrong{df\_out} (\sphinxstyleemphasis{pands.DataFrame}) \textendash{} Metadata dataframe, after document removal

\end{itemize}


\end{description}\end{quote}

\end{fulllineitems}

\index{score\_by\_keyword\_count() (src.domain\_classifier.preprocessor.CorpusDFProcessor method)@\spxentry{score\_by\_keyword\_count()}\spxextra{src.domain\_classifier.preprocessor.CorpusDFProcessor method}}

\begin{fulllineitems}
\phantomsection\label{\detokenize{dc_preprocessor:src.domain_classifier.preprocessor.CorpusDFProcessor.score_by_keyword_count}}\pysiglinewithargsret{\sphinxbfcode{\sphinxupquote{score\_by\_keyword\_count}}}{\emph{\DUrole{n}{keywords}}, \emph{\DUrole{n}{wt}\DUrole{o}{=}\DUrole{default_value}{2}}}{}
\sphinxAtStartPar
Computes a score for every document in a given pandas dataframe
according to the frequency of appearing some given keywords
\begin{quote}\begin{description}
\item[{Parameters}] \leavevmode\begin{itemize}
\item {} 
\sphinxAtStartPar
\sphinxstylestrong{corpus} (\sphinxstyleemphasis{dataframe}) \textendash{} Dataframe of corpus.

\item {} 
\sphinxAtStartPar
\sphinxstylestrong{keywords} (\sphinxstyleemphasis{list of str}) \textendash{} List of keywords

\item {} 
\sphinxAtStartPar
\sphinxstylestrong{wt} (\sphinxstyleemphasis{float, optional (default=2)}) \textendash{} Weighting factor for the title components. Keyword matches with
title words are weighted by this factor

\end{itemize}

\item[{Returns}] \leavevmode
\sphinxAtStartPar
\sphinxstylestrong{score} \textendash{} List of scores, one per documents in corpus

\item[{Return type}] \leavevmode
\sphinxAtStartPar
list of float

\end{description}\end{quote}

\end{fulllineitems}

\index{score\_by\_keywords() (src.domain\_classifier.preprocessor.CorpusDFProcessor method)@\spxentry{score\_by\_keywords()}\spxextra{src.domain\_classifier.preprocessor.CorpusDFProcessor method}}

\begin{fulllineitems}
\phantomsection\label{\detokenize{dc_preprocessor:src.domain_classifier.preprocessor.CorpusDFProcessor.score_by_keywords}}\pysiglinewithargsret{\sphinxbfcode{\sphinxupquote{score\_by\_keywords}}}{\emph{\DUrole{n}{keywords}}, \emph{\DUrole{n}{wt}\DUrole{o}{=}\DUrole{default_value}{2}}}{}
\sphinxAtStartPar
Computes a score for every document in a given pandas dataframe
according to the frequency of appearing some given keywords
\begin{quote}\begin{description}
\item[{Parameters}] \leavevmode\begin{itemize}
\item {} 
\sphinxAtStartPar
\sphinxstylestrong{keywords} (\sphinxstyleemphasis{list of str}) \textendash{} List of keywords

\item {} 
\sphinxAtStartPar
\sphinxstylestrong{wt} (\sphinxstyleemphasis{float, optional (default=2)}) \textendash{} Weighting factor for the title components. Keyword matches with
title words are weighted by this factor
This input argument is used if self.path2embeddings is None only

\end{itemize}

\item[{Returns}] \leavevmode
\sphinxAtStartPar
\sphinxstylestrong{score} \textendash{} List of scores, one per documents in corpus

\item[{Return type}] \leavevmode
\sphinxAtStartPar
list of float

\end{description}\end{quote}

\end{fulllineitems}

\index{score\_by\_topics() (src.domain\_classifier.preprocessor.CorpusDFProcessor method)@\spxentry{score\_by\_topics()}\spxextra{src.domain\_classifier.preprocessor.CorpusDFProcessor method}}

\begin{fulllineitems}
\phantomsection\label{\detokenize{dc_preprocessor:src.domain_classifier.preprocessor.CorpusDFProcessor.score_by_topics}}\pysiglinewithargsret{\sphinxbfcode{\sphinxupquote{score\_by\_topics}}}{\emph{\DUrole{n}{T}}, \emph{\DUrole{n}{doc\_ids}}, \emph{\DUrole{n}{topic\_weights}}}{}
\sphinxAtStartPar
Computes a score for every document in a given pandas dataframe
according to the relevance of a weighted list of topics
\begin{quote}\begin{description}
\item[{Parameters}] \leavevmode\begin{itemize}
\item {} 
\sphinxAtStartPar
\sphinxstylestrong{T} (\sphinxstyleemphasis{numpy.ndarray or scipy.sparse}) \textendash{} Topic matrix (one column per topic)

\item {} 
\sphinxAtStartPar
\sphinxstylestrong{doc\_ids} (\sphinxstyleemphasis{array\sphinxhyphen{}like}) \textendash{} Ids of the documents in the topic matrix. doc\_ids{[}i{]} = ‘123’ means
that document with id ‘123’ has topic vector T{[}i{]}

\item {} 
\sphinxAtStartPar
\sphinxstylestrong{topic\_weights} (\sphinxstyleemphasis{dict}) \textendash{} Dictionary \{t\_i: w\_i\}, where t\_i is a topic index and w\_i is the
weight of the topic

\end{itemize}

\item[{Returns}] \leavevmode
\sphinxAtStartPar
\sphinxstylestrong{score} \textendash{} List of scores, one per documents in corpus

\item[{Return type}] \leavevmode
\sphinxAtStartPar
list of float

\end{description}\end{quote}

\end{fulllineitems}

\index{score\_by\_zeroshot() (src.domain\_classifier.preprocessor.CorpusDFProcessor method)@\spxentry{score\_by\_zeroshot()}\spxextra{src.domain\_classifier.preprocessor.CorpusDFProcessor method}}

\begin{fulllineitems}
\phantomsection\label{\detokenize{dc_preprocessor:src.domain_classifier.preprocessor.CorpusDFProcessor.score_by_zeroshot}}\pysiglinewithargsret{\sphinxbfcode{\sphinxupquote{score\_by\_zeroshot}}}{\emph{\DUrole{n}{keyword}}}{}
\sphinxAtStartPar
Computes a score for every document in a given pandas dataframe
according to the relevance of a given keyword according to a pretrained
zero\sphinxhyphen{}shot classifier
\begin{quote}\begin{description}
\item[{Parameters}] \leavevmode
\sphinxAtStartPar
\sphinxstylestrong{keyword} (\sphinxstyleemphasis{str}) \textendash{} Keywords defining the target category

\item[{Returns}] \leavevmode
\sphinxAtStartPar
\sphinxstylestrong{score} \textendash{} List of scores, one per documents in corpus

\item[{Return type}] \leavevmode
\sphinxAtStartPar
list of float

\end{description}\end{quote}

\end{fulllineitems}


\end{fulllineitems}

\index{CorpusProcessor (class in src.domain\_classifier.preprocessor)@\spxentry{CorpusProcessor}\spxextra{class in src.domain\_classifier.preprocessor}}

\begin{fulllineitems}
\phantomsection\label{\detokenize{dc_preprocessor:src.domain_classifier.preprocessor.CorpusProcessor}}\pysiglinewithargsret{\sphinxbfcode{\sphinxupquote{class\DUrole{w}{  }}}\sphinxcode{\sphinxupquote{src.domain\_classifier.preprocessor.}}\sphinxbfcode{\sphinxupquote{CorpusProcessor}}}{\emph{\DUrole{n}{path2embeddings}\DUrole{o}{=}\DUrole{default_value}{None}}, \emph{\DUrole{n}{path2zeroshot}\DUrole{o}{=}\DUrole{default_value}{None}}}{}
\sphinxAtStartPar
Bases: \sphinxcode{\sphinxupquote{object}}

\sphinxAtStartPar
A container of corpus preprocessing methods
It provides basic processing methods to a corpus of text documents
The input corpus must be given by a list of strings (or a pandas series
of strings)
\index{\_\_init\_\_() (src.domain\_classifier.preprocessor.CorpusProcessor method)@\spxentry{\_\_init\_\_()}\spxextra{src.domain\_classifier.preprocessor.CorpusProcessor method}}

\begin{fulllineitems}
\phantomsection\label{\detokenize{dc_preprocessor:src.domain_classifier.preprocessor.CorpusProcessor.__init__}}\pysiglinewithargsret{\sphinxbfcode{\sphinxupquote{\_\_init\_\_}}}{\emph{\DUrole{n}{path2embeddings}\DUrole{o}{=}\DUrole{default_value}{None}}, \emph{\DUrole{n}{path2zeroshot}\DUrole{o}{=}\DUrole{default_value}{None}}}{}
\sphinxAtStartPar
Initializes a preprocessor object
\begin{quote}\begin{description}
\item[{Parameters}] \leavevmode\begin{itemize}
\item {} 
\sphinxAtStartPar
\sphinxstylestrong{path2embeddings} (\sphinxstyleemphasis{str or pathlib.Path or None, optional (default=None)}) \textendash{} Path to the folder containing the document embeddings.
If None, no embeddings will be used. Document scores will be based
in word counts

\item {} 
\sphinxAtStartPar
\sphinxstylestrong{path2zeroshot} (\sphinxstyleemphasis{str or pathlib.Path or None, optional (default=None)}) \textendash{} Path to the folder containing the pretrained zero\sphinxhyphen{}shot model
If None, zero\sphinxhyphen{}shot classification will not be available.

\end{itemize}

\end{description}\end{quote}

\end{fulllineitems}

\index{\_\_weakref\_\_ (src.domain\_classifier.preprocessor.CorpusProcessor attribute)@\spxentry{\_\_weakref\_\_}\spxextra{src.domain\_classifier.preprocessor.CorpusProcessor attribute}}

\begin{fulllineitems}
\phantomsection\label{\detokenize{dc_preprocessor:src.domain_classifier.preprocessor.CorpusProcessor.__weakref__}}\pysigline{\sphinxbfcode{\sphinxupquote{\_\_weakref\_\_}}}
\sphinxAtStartPar
list of weak references to the object (if defined)

\end{fulllineitems}

\index{compute\_keyword\_stats() (src.domain\_classifier.preprocessor.CorpusProcessor method)@\spxentry{compute\_keyword\_stats()}\spxextra{src.domain\_classifier.preprocessor.CorpusProcessor method}}

\begin{fulllineitems}
\phantomsection\label{\detokenize{dc_preprocessor:src.domain_classifier.preprocessor.CorpusProcessor.compute_keyword_stats}}\pysiglinewithargsret{\sphinxbfcode{\sphinxupquote{compute\_keyword\_stats}}}{\emph{\DUrole{n}{corpus}}, \emph{\DUrole{n}{keywords}}}{}
\sphinxAtStartPar
Computes keyword statistics
\begin{quote}\begin{description}
\item[{Parameters}] \leavevmode\begin{itemize}
\item {} 
\sphinxAtStartPar
\sphinxstylestrong{corpus} (\sphinxstyleemphasis{list (or pandas.Series) of str}) \textendash{} Input corpus.

\item {} 
\sphinxAtStartPar
\sphinxstylestrong{keywords} (\sphinxstyleemphasis{list of str}) \textendash{} List of keywords

\end{itemize}

\item[{Returns}] \leavevmode
\sphinxAtStartPar
\begin{itemize}
\item {} 
\sphinxAtStartPar
\sphinxstylestrong{df\_stats} (\sphinxstyleemphasis{dict}) \textendash{} Dictionary of document frequencies per keyword
df\_stats{[}k{]} is the number of docs containing keyword k

\item {} 
\sphinxAtStartPar
\sphinxstylestrong{kf\_stats} (\sphinxstyleemphasis{dict}) \textendash{} Dictionary of keyword frequencies
df\_stats{[}k{]} is the number of times keyword k appers in the corpus

\end{itemize}


\end{description}\end{quote}

\end{fulllineitems}

\index{get\_top\_scores() (src.domain\_classifier.preprocessor.CorpusProcessor method)@\spxentry{get\_top\_scores()}\spxextra{src.domain\_classifier.preprocessor.CorpusProcessor method}}

\begin{fulllineitems}
\phantomsection\label{\detokenize{dc_preprocessor:src.domain_classifier.preprocessor.CorpusProcessor.get_top_scores}}\pysiglinewithargsret{\sphinxbfcode{\sphinxupquote{get\_top\_scores}}}{\emph{\DUrole{n}{scores}}, \emph{\DUrole{n}{n\_max}\DUrole{o}{=}\DUrole{default_value}{1e+100}}, \emph{\DUrole{n}{s\_min}\DUrole{o}{=}\DUrole{default_value}{0}}}{}
\sphinxAtStartPar
Select the elements from a given list of numbers that fulfill some
conditions
\begin{quote}\begin{description}
\item[{Parameters}] \leavevmode\begin{itemize}
\item {} 
\sphinxAtStartPar
\sphinxstylestrong{n\_max} (\sphinxstyleemphasis{int or None, optional (defaul=1e100)}) \textendash{} Maximum number of elements in the output list. The default is
a huge number that, in practice, means there is no loimit

\item {} 
\sphinxAtStartPar
\sphinxstylestrong{s\_min} (\sphinxstyleemphasis{float, optional (default=0)}) \textendash{} Minimum score. Only elements strictly above s\_min are selected

\end{itemize}

\end{description}\end{quote}

\end{fulllineitems}

\index{score\_docs\_by\_keyword\_count() (src.domain\_classifier.preprocessor.CorpusProcessor method)@\spxentry{score\_docs\_by\_keyword\_count()}\spxextra{src.domain\_classifier.preprocessor.CorpusProcessor method}}

\begin{fulllineitems}
\phantomsection\label{\detokenize{dc_preprocessor:src.domain_classifier.preprocessor.CorpusProcessor.score_docs_by_keyword_count}}\pysiglinewithargsret{\sphinxbfcode{\sphinxupquote{score\_docs\_by\_keyword\_count}}}{\emph{\DUrole{n}{corpus}}, \emph{\DUrole{n}{keywords}}}{}
\sphinxAtStartPar
Computes a score for every document in a given pandas dataframe
according to the frequency of appearing some given keywords
\begin{quote}\begin{description}
\item[{Parameters}] \leavevmode\begin{itemize}
\item {} 
\sphinxAtStartPar
\sphinxstylestrong{corpus} (\sphinxstyleemphasis{list (or pandas.Series) of str}) \textendash{} Input corpus.

\item {} 
\sphinxAtStartPar
\sphinxstylestrong{keywords} (\sphinxstyleemphasis{list of str}) \textendash{} List of keywords

\end{itemize}

\item[{Returns}] \leavevmode
\sphinxAtStartPar
\sphinxstylestrong{score} \textendash{} List of scores, one per document in corpus

\item[{Return type}] \leavevmode
\sphinxAtStartPar
list of float

\end{description}\end{quote}

\end{fulllineitems}

\index{score\_docs\_by\_keywords() (src.domain\_classifier.preprocessor.CorpusProcessor method)@\spxentry{score\_docs\_by\_keywords()}\spxextra{src.domain\_classifier.preprocessor.CorpusProcessor method}}

\begin{fulllineitems}
\phantomsection\label{\detokenize{dc_preprocessor:src.domain_classifier.preprocessor.CorpusProcessor.score_docs_by_keywords}}\pysiglinewithargsret{\sphinxbfcode{\sphinxupquote{score\_docs\_by\_keywords}}}{\emph{\DUrole{n}{corpus}}, \emph{\DUrole{n}{keywords}}}{}
\sphinxAtStartPar
Computes a score for every document in a given pandas dataframe
according to the frequency of appearing some given keywords
\begin{quote}\begin{description}
\item[{Parameters}] \leavevmode\begin{itemize}
\item {} 
\sphinxAtStartPar
\sphinxstylestrong{corpus} (\sphinxstyleemphasis{list (or pandas.Series) of str}) \textendash{} Input corpus.

\item {} 
\sphinxAtStartPar
\sphinxstylestrong{keywords} (\sphinxstyleemphasis{list of str}) \textendash{} List of keywords

\end{itemize}

\item[{Returns}] \leavevmode
\sphinxAtStartPar
\sphinxstylestrong{score} \textendash{} List of scores, one per document in corpus

\item[{Return type}] \leavevmode
\sphinxAtStartPar
list of float

\end{description}\end{quote}

\end{fulllineitems}

\index{score\_docs\_by\_zeroshot() (src.domain\_classifier.preprocessor.CorpusProcessor method)@\spxentry{score\_docs\_by\_zeroshot()}\spxextra{src.domain\_classifier.preprocessor.CorpusProcessor method}}

\begin{fulllineitems}
\phantomsection\label{\detokenize{dc_preprocessor:src.domain_classifier.preprocessor.CorpusProcessor.score_docs_by_zeroshot}}\pysiglinewithargsret{\sphinxbfcode{\sphinxupquote{score\_docs\_by\_zeroshot}}}{\emph{\DUrole{n}{corpus}}, \emph{\DUrole{n}{keyword}}}{}
\sphinxAtStartPar
Computes a score for every document in a given pandas dataframe
according to a given keyword and a pre\sphinxhyphen{}trained zero\sphinxhyphen{}shot classifier
\begin{quote}\begin{description}
\item[{Parameters}] \leavevmode\begin{itemize}
\item {} 
\sphinxAtStartPar
\sphinxstylestrong{corpus} (\sphinxstyleemphasis{list (or pandas.Series) of str}) \textendash{} Input corpus.

\item {} 
\sphinxAtStartPar
\sphinxstylestrong{keyword} (\sphinxstyleemphasis{str}) \textendash{} Keyword defining the target category

\end{itemize}

\item[{Returns}] \leavevmode
\sphinxAtStartPar
\sphinxstylestrong{score} \textendash{} List of scores, one per document in corpus

\item[{Return type}] \leavevmode
\sphinxAtStartPar
list of float

\end{description}\end{quote}

\end{fulllineitems}


\end{fulllineitems}



\chapter{Main Window}
\label{\detokenize{gui_main_window:main-window}}\label{\detokenize{gui_main_window::doc}}\phantomsection\label{\detokenize{gui_main_window:module-src.graphical_user_interface.main_window}}\index{module@\spxentry{module}!src.graphical\_user\_interface.main\_window@\spxentry{src.graphical\_user\_interface.main\_window}}\index{src.graphical\_user\_interface.main\_window@\spxentry{src.graphical\_user\_interface.main\_window}!module@\spxentry{module}}
\sphinxAtStartPar
@author: lcalv
\index{MainWindow (class in src.graphical\_user\_interface.main\_window)@\spxentry{MainWindow}\spxextra{class in src.graphical\_user\_interface.main\_window}}

\begin{fulllineitems}
\phantomsection\label{\detokenize{gui_main_window:src.graphical_user_interface.main_window.MainWindow}}\pysiglinewithargsret{\sphinxbfcode{\sphinxupquote{class\DUrole{w}{  }}}\sphinxcode{\sphinxupquote{src.graphical\_user\_interface.main\_window.}}\sphinxbfcode{\sphinxupquote{MainWindow}}}{\emph{\DUrole{n}{project\_folder}}, \emph{\DUrole{n}{source\_folder}}, \emph{\DUrole{n}{tm}}, \emph{\DUrole{n}{widget}}, \emph{\DUrole{n}{stdout}}, \emph{\DUrole{n}{stderr}}}{}
\sphinxAtStartPar
Bases: \sphinxcode{\sphinxupquote{PyQt5.QtWidgets.QMainWindow}}

\sphinxAtStartPar
Class representing the main window of the application.
\index{\_\_init\_\_() (src.graphical\_user\_interface.main\_window.MainWindow method)@\spxentry{\_\_init\_\_()}\spxextra{src.graphical\_user\_interface.main\_window.MainWindow method}}

\begin{fulllineitems}
\phantomsection\label{\detokenize{gui_main_window:src.graphical_user_interface.main_window.MainWindow.__init__}}\pysiglinewithargsret{\sphinxbfcode{\sphinxupquote{\_\_init\_\_}}}{\emph{\DUrole{n}{project\_folder}}, \emph{\DUrole{n}{source\_folder}}, \emph{\DUrole{n}{tm}}, \emph{\DUrole{n}{widget}}, \emph{\DUrole{n}{stdout}}, \emph{\DUrole{n}{stderr}}}{}
\sphinxAtStartPar
Initializes the application’s main window based on the parameters received
from the application’s starting window.
\begin{quote}\begin{description}
\item[{Parameters}] \leavevmode\begin{itemize}
\item {} 
\sphinxAtStartPar
\sphinxstylestrong{project\_folder} (\sphinxstyleemphasis{pathlib.Path}) \textendash{} Path to the application project

\item {} 
\sphinxAtStartPar
\sphinxstylestrong{source\_folder} (\sphinxstyleemphasis{pathlib.Path}) \textendash{} Path to the folder containing the data sources

\item {} 
\sphinxAtStartPar
\sphinxstylestrong{tm} (\sphinxstyleemphasis{TaskManager}) \textendash{} TaskManager object associated with the project

\item {} 
\sphinxAtStartPar
\sphinxstylestrong{widget} (\sphinxstyleemphasis{QtWidgets.QStackedWidget}) \textendash{} Window to which the application’s main window is attached to

\item {} 
\sphinxAtStartPar
\sphinxstylestrong{stdout} (\sphinxstyleemphasis{sys.stdout}) \textendash{} Output file object

\item {} 
\sphinxAtStartPar
\sphinxstylestrong{stderr} (\sphinxstyleemphasis{sys.stderr}) \textendash{} Standard Error file object

\end{itemize}

\end{description}\end{quote}

\end{fulllineitems}

\index{append\_text\_evaluate() (src.graphical\_user\_interface.main\_window.MainWindow method)@\spxentry{append\_text\_evaluate()}\spxextra{src.graphical\_user\_interface.main\_window.MainWindow method}}

\begin{fulllineitems}
\phantomsection\label{\detokenize{gui_main_window:src.graphical_user_interface.main_window.MainWindow.append_text_evaluate}}\pysiglinewithargsret{\sphinxbfcode{\sphinxupquote{append\_text\_evaluate}}}{\emph{\DUrole{n}{text}}}{}
\sphinxAtStartPar
Method to redirect the stdout and stderr in the “text\_edit\_results\_eval\_pu\_classifier”
while the evaluation of a PU model is being performed.

\end{fulllineitems}

\index{append\_text\_retrain\_reval() (src.graphical\_user\_interface.main\_window.MainWindow method)@\spxentry{append\_text\_retrain\_reval()}\spxextra{src.graphical\_user\_interface.main\_window.MainWindow method}}

\begin{fulllineitems}
\phantomsection\label{\detokenize{gui_main_window:src.graphical_user_interface.main_window.MainWindow.append_text_retrain_reval}}\pysiglinewithargsret{\sphinxbfcode{\sphinxupquote{append\_text\_retrain\_reval}}}{\emph{\DUrole{n}{text}}}{}
\sphinxAtStartPar
Method to redirect the stdout and stderr in the “text\_edit\_results\_reval\_retrain\_pu\_model”
while the retraining of a PU model is being performed.

\end{fulllineitems}

\index{append\_text\_train() (src.graphical\_user\_interface.main\_window.MainWindow method)@\spxentry{append\_text\_train()}\spxextra{src.graphical\_user\_interface.main\_window.MainWindow method}}

\begin{fulllineitems}
\phantomsection\label{\detokenize{gui_main_window:src.graphical_user_interface.main_window.MainWindow.append_text_train}}\pysiglinewithargsret{\sphinxbfcode{\sphinxupquote{append\_text\_train}}}{\emph{\DUrole{n}{text}}}{}
\sphinxAtStartPar
Method to redirect the stdout and stderr in the “text\_logs\_train\_pu\_model”
while the training of a PU model is being performed.

\end{fulllineitems}

\index{center() (src.graphical\_user\_interface.main\_window.MainWindow method)@\spxentry{center()}\spxextra{src.graphical\_user\_interface.main\_window.MainWindow method}}

\begin{fulllineitems}
\phantomsection\label{\detokenize{gui_main_window:src.graphical_user_interface.main_window.MainWindow.center}}\pysiglinewithargsret{\sphinxbfcode{\sphinxupquote{center}}}{}{}
\sphinxAtStartPar
Centers the window at the middle of the screen at which the application is being executed.

\end{fulllineitems}

\index{clicked\_change\_predicted\_class() (src.graphical\_user\_interface.main\_window.MainWindow method)@\spxentry{clicked\_change\_predicted\_class()}\spxextra{src.graphical\_user\_interface.main\_window.MainWindow method}}

\begin{fulllineitems}
\phantomsection\label{\detokenize{gui_main_window:src.graphical_user_interface.main_window.MainWindow.clicked_change_predicted_class}}\pysiglinewithargsret{\sphinxbfcode{\sphinxupquote{clicked\_change\_predicted\_class}}}{\emph{\DUrole{n}{checkbox}}}{}
\sphinxAtStartPar
Method to control the checking or unchecking of the QCheckboxes that represented
the predicted class that the user has associated to each of the documents to annotate.

\end{fulllineitems}

\index{clicked\_evaluate\_PU\_model() (src.graphical\_user\_interface.main\_window.MainWindow method)@\spxentry{clicked\_evaluate\_PU\_model()}\spxextra{src.graphical\_user\_interface.main\_window.MainWindow method}}

\begin{fulllineitems}
\phantomsection\label{\detokenize{gui_main_window:src.graphical_user_interface.main_window.MainWindow.clicked_evaluate_PU_model}}\pysiglinewithargsret{\sphinxbfcode{\sphinxupquote{clicked\_evaluate\_PU\_model}}}{}{}
\sphinxAtStartPar
Method that controls the actions that are carried out when the button “eval\_pu\_classifier\_push\_button” is
clicked by the user.

\end{fulllineitems}

\index{clicked\_get\_labels() (src.graphical\_user\_interface.main\_window.MainWindow method)@\spxentry{clicked\_get\_labels()}\spxextra{src.graphical\_user\_interface.main\_window.MainWindow method}}

\begin{fulllineitems}
\phantomsection\label{\detokenize{gui_main_window:src.graphical_user_interface.main_window.MainWindow.clicked_get_labels}}\pysiglinewithargsret{\sphinxbfcode{\sphinxupquote{clicked\_get\_labels}}}{}{}
\sphinxAtStartPar
Method for performing the getting of the labels according to the
method selected for it.

\end{fulllineitems}

\index{clicked\_get\_labels\_option() (src.graphical\_user\_interface.main\_window.MainWindow method)@\spxentry{clicked\_get\_labels\_option()}\spxextra{src.graphical\_user\_interface.main\_window.MainWindow method}}

\begin{fulllineitems}
\phantomsection\label{\detokenize{gui_main_window:src.graphical_user_interface.main_window.MainWindow.clicked_get_labels_option}}\pysiglinewithargsret{\sphinxbfcode{\sphinxupquote{clicked\_get\_labels\_option}}}{}{}
\sphinxAtStartPar
Method to control the functionality associated with the selection of
each of the QRadioButtons associated with the labels’ getting.
Only one QRadioButton can be selected at a time.

\end{fulllineitems}

\index{clicked\_give\_feedback() (src.graphical\_user\_interface.main\_window.MainWindow method)@\spxentry{clicked\_give\_feedback()}\spxextra{src.graphical\_user\_interface.main\_window.MainWindow method}}

\begin{fulllineitems}
\phantomsection\label{\detokenize{gui_main_window:src.graphical_user_interface.main_window.MainWindow.clicked_give_feedback}}\pysiglinewithargsret{\sphinxbfcode{\sphinxupquote{clicked\_give\_feedback}}}{}{}
\sphinxAtStartPar
Method that controls the actions that are carried out when the button “give\_feedback\_user\_push\_button” is
clicked by the user.

\end{fulllineitems}

\index{clicked\_load\_corpus() (src.graphical\_user\_interface.main\_window.MainWindow method)@\spxentry{clicked\_load\_corpus()}\spxextra{src.graphical\_user\_interface.main\_window.MainWindow method}}

\begin{fulllineitems}
\phantomsection\label{\detokenize{gui_main_window:src.graphical_user_interface.main_window.MainWindow.clicked_load_corpus}}\pysiglinewithargsret{\sphinxbfcode{\sphinxupquote{clicked\_load\_corpus}}}{}{}
\sphinxAtStartPar
Method to control the selection of a new corpus by double\sphinxhyphen{}clicking
one of the items of the corpus list within the selected source folder,
as well as its loading as dataframe into the TaskManager object.
Important is that the corpus cannot be changed inside the same project,
so if a corpus was used before me must keep the same one.

\end{fulllineitems}

\index{clicked\_load\_labels() (src.graphical\_user\_interface.main\_window.MainWindow method)@\spxentry{clicked\_load\_labels()}\spxextra{src.graphical\_user\_interface.main\_window.MainWindow method}}

\begin{fulllineitems}
\phantomsection\label{\detokenize{gui_main_window:src.graphical_user_interface.main_window.MainWindow.clicked_load_labels}}\pysiglinewithargsret{\sphinxbfcode{\sphinxupquote{clicked\_load\_labels}}}{}{}
\sphinxAtStartPar
Method for controlling the loading of the labels into the session.
It is equivalent to the “\_get\_labelset\_list” method from the
TaskManager class

\end{fulllineitems}

\index{clicked\_reevaluate\_model() (src.graphical\_user\_interface.main\_window.MainWindow method)@\spxentry{clicked\_reevaluate\_model()}\spxextra{src.graphical\_user\_interface.main\_window.MainWindow method}}

\begin{fulllineitems}
\phantomsection\label{\detokenize{gui_main_window:src.graphical_user_interface.main_window.MainWindow.clicked_reevaluate_model}}\pysiglinewithargsret{\sphinxbfcode{\sphinxupquote{clicked\_reevaluate\_model}}}{}{}
\sphinxAtStartPar
Method that controls the actions that are carried out when the button “retrain\_pu\_model\_push\_button” is
clicked by the user.

\end{fulllineitems}

\index{clicked\_reset\_labels() (src.graphical\_user\_interface.main\_window.MainWindow method)@\spxentry{clicked\_reset\_labels()}\spxextra{src.graphical\_user\_interface.main\_window.MainWindow method}}

\begin{fulllineitems}
\phantomsection\label{\detokenize{gui_main_window:src.graphical_user_interface.main_window.MainWindow.clicked_reset_labels}}\pysiglinewithargsret{\sphinxbfcode{\sphinxupquote{clicked\_reset\_labels}}}{}{}
\sphinxAtStartPar
Method for controlling the resetting of the current session’s labels.

\end{fulllineitems}

\index{clicked\_retrain\_model() (src.graphical\_user\_interface.main\_window.MainWindow method)@\spxentry{clicked\_retrain\_model()}\spxextra{src.graphical\_user\_interface.main\_window.MainWindow method}}

\begin{fulllineitems}
\phantomsection\label{\detokenize{gui_main_window:src.graphical_user_interface.main_window.MainWindow.clicked_retrain_model}}\pysiglinewithargsret{\sphinxbfcode{\sphinxupquote{clicked\_retrain\_model}}}{}{}
\sphinxAtStartPar
Method that controls the actions that are carried out when the button “retrain\_pu\_model\_push\_button” is
clicked by the user.

\end{fulllineitems}

\index{clicked\_train\_PU\_model() (src.graphical\_user\_interface.main\_window.MainWindow method)@\spxentry{clicked\_train\_PU\_model()}\spxextra{src.graphical\_user\_interface.main\_window.MainWindow method}}

\begin{fulllineitems}
\phantomsection\label{\detokenize{gui_main_window:src.graphical_user_interface.main_window.MainWindow.clicked_train_PU_model}}\pysiglinewithargsret{\sphinxbfcode{\sphinxupquote{clicked\_train\_PU\_model}}}{}{}
\sphinxAtStartPar
Method that controls the actions that are carried out when the button “train\_pu\_model\_push\_button” is
clicked by the user.

\end{fulllineitems}

\index{clicked\_update\_ndocs\_al() (src.graphical\_user\_interface.main\_window.MainWindow method)@\spxentry{clicked\_update\_ndocs\_al()}\spxextra{src.graphical\_user\_interface.main\_window.MainWindow method}}

\begin{fulllineitems}
\phantomsection\label{\detokenize{gui_main_window:src.graphical_user_interface.main_window.MainWindow.clicked_update_ndocs_al}}\pysiglinewithargsret{\sphinxbfcode{\sphinxupquote{clicked\_update\_ndocs\_al}}}{}{}
\sphinxAtStartPar
Updates the AL parameter that is going to be used for the resampling of the documents to be labelled based
on the value specified by the user and shows the id, title, abstract and predicted class (if available) of each
of the documents in a QTextEdit widget.

\end{fulllineitems}

\index{do\_after\_evaluate\_pu\_model() (src.graphical\_user\_interface.main\_window.MainWindow method)@\spxentry{do\_after\_evaluate\_pu\_model()}\spxextra{src.graphical\_user\_interface.main\_window.MainWindow method}}

\begin{fulllineitems}
\phantomsection\label{\detokenize{gui_main_window:src.graphical_user_interface.main_window.MainWindow.do_after_evaluate_pu_model}}\pysiglinewithargsret{\sphinxbfcode{\sphinxupquote{do\_after\_evaluate\_pu\_model}}}{}{}
\sphinxAtStartPar
Method to be executed after the evaluation of the PU model has been
completed.

\end{fulllineitems}

\index{do\_after\_give\_feedback() (src.graphical\_user\_interface.main\_window.MainWindow method)@\spxentry{do\_after\_give\_feedback()}\spxextra{src.graphical\_user\_interface.main\_window.MainWindow method}}

\begin{fulllineitems}
\phantomsection\label{\detokenize{gui_main_window:src.graphical_user_interface.main_window.MainWindow.do_after_give_feedback}}\pysiglinewithargsret{\sphinxbfcode{\sphinxupquote{do\_after\_give\_feedback}}}{}{}
\sphinxAtStartPar
Method to be executed after annotating the labels given by the user in their corresponding positions

\end{fulllineitems}

\index{do\_after\_import\_labels() (src.graphical\_user\_interface.main\_window.MainWindow method)@\spxentry{do\_after\_import\_labels()}\spxextra{src.graphical\_user\_interface.main\_window.MainWindow method}}

\begin{fulllineitems}
\phantomsection\label{\detokenize{gui_main_window:src.graphical_user_interface.main_window.MainWindow.do_after_import_labels}}\pysiglinewithargsret{\sphinxbfcode{\sphinxupquote{do\_after\_import\_labels}}}{}{}
\sphinxAtStartPar
Function to be executed after the labels’ importing has been completed.

\end{fulllineitems}

\index{do\_after\_load\_corpus() (src.graphical\_user\_interface.main\_window.MainWindow method)@\spxentry{do\_after\_load\_corpus()}\spxextra{src.graphical\_user\_interface.main\_window.MainWindow method}}

\begin{fulllineitems}
\phantomsection\label{\detokenize{gui_main_window:src.graphical_user_interface.main_window.MainWindow.do_after_load_corpus}}\pysiglinewithargsret{\sphinxbfcode{\sphinxupquote{do\_after\_load\_corpus}}}{}{}
\sphinxAtStartPar
Method to be executed after the loading of the corpus has been
completed.

\end{fulllineitems}

\index{do\_after\_reevaluate\_model() (src.graphical\_user\_interface.main\_window.MainWindow method)@\spxentry{do\_after\_reevaluate\_model()}\spxextra{src.graphical\_user\_interface.main\_window.MainWindow method}}

\begin{fulllineitems}
\phantomsection\label{\detokenize{gui_main_window:src.graphical_user_interface.main_window.MainWindow.do_after_reevaluate_model}}\pysiglinewithargsret{\sphinxbfcode{\sphinxupquote{do\_after\_reevaluate\_model}}}{}{}
\sphinxAtStartPar
Method to be executed once the reevaluation of the model based on the feedback of the user has been completed.

\end{fulllineitems}

\index{do\_after\_retrain\_model() (src.graphical\_user\_interface.main\_window.MainWindow method)@\spxentry{do\_after\_retrain\_model()}\spxextra{src.graphical\_user\_interface.main\_window.MainWindow method}}

\begin{fulllineitems}
\phantomsection\label{\detokenize{gui_main_window:src.graphical_user_interface.main_window.MainWindow.do_after_retrain_model}}\pysiglinewithargsret{\sphinxbfcode{\sphinxupquote{do\_after\_retrain\_model}}}{}{}
\sphinxAtStartPar
Method to be executed once the retraining of the model based on the feedback of the user has been completed.

\end{fulllineitems}

\index{do\_after\_train\_classifier() (src.graphical\_user\_interface.main\_window.MainWindow method)@\spxentry{do\_after\_train\_classifier()}\spxextra{src.graphical\_user\_interface.main\_window.MainWindow method}}

\begin{fulllineitems}
\phantomsection\label{\detokenize{gui_main_window:src.graphical_user_interface.main_window.MainWindow.do_after_train_classifier}}\pysiglinewithargsret{\sphinxbfcode{\sphinxupquote{do\_after\_train\_classifier}}}{}{}
\sphinxAtStartPar
Method to be executed after the training of the classifier has been
completed.

\end{fulllineitems}

\index{execute\_evaluate\_pu\_model() (src.graphical\_user\_interface.main\_window.MainWindow method)@\spxentry{execute\_evaluate\_pu\_model()}\spxextra{src.graphical\_user\_interface.main\_window.MainWindow method}}

\begin{fulllineitems}
\phantomsection\label{\detokenize{gui_main_window:src.graphical_user_interface.main_window.MainWindow.execute_evaluate_pu_model}}\pysiglinewithargsret{\sphinxbfcode{\sphinxupquote{execute\_evaluate\_pu\_model}}}{}{}
\sphinxAtStartPar
Method to control the execution of the evaluation of a classifier on a
secondary thread while the MainWindow execution is maintained in the
main thread.

\end{fulllineitems}

\index{execute\_give\_feedback() (src.graphical\_user\_interface.main\_window.MainWindow method)@\spxentry{execute\_give\_feedback()}\spxextra{src.graphical\_user\_interface.main\_window.MainWindow method}}

\begin{fulllineitems}
\phantomsection\label{\detokenize{gui_main_window:src.graphical_user_interface.main_window.MainWindow.execute_give_feedback}}\pysiglinewithargsret{\sphinxbfcode{\sphinxupquote{execute\_give\_feedback}}}{}{}
\sphinxAtStartPar
Method to control the annotation of a selected subset of documents based on the labels introduced by the
user on a secondary thread while the MainWindow execution is maintained in the main thread.

\end{fulllineitems}

\index{execute\_import\_labels() (src.graphical\_user\_interface.main\_window.MainWindow method)@\spxentry{execute\_import\_labels()}\spxextra{src.graphical\_user\_interface.main\_window.MainWindow method}}

\begin{fulllineitems}
\phantomsection\label{\detokenize{gui_main_window:src.graphical_user_interface.main_window.MainWindow.execute_import_labels}}\pysiglinewithargsret{\sphinxbfcode{\sphinxupquote{execute\_import\_labels}}}{}{}
\sphinxAtStartPar
Imports the labels by invoking the corresponding method in the Task Manager
object associated with the GUI.

\end{fulllineitems}

\index{execute\_load\_corpus() (src.graphical\_user\_interface.main\_window.MainWindow method)@\spxentry{execute\_load\_corpus()}\spxextra{src.graphical\_user\_interface.main\_window.MainWindow method}}

\begin{fulllineitems}
\phantomsection\label{\detokenize{gui_main_window:src.graphical_user_interface.main_window.MainWindow.execute_load_corpus}}\pysiglinewithargsret{\sphinxbfcode{\sphinxupquote{execute\_load\_corpus}}}{}{}
\sphinxAtStartPar
Method to control the execution of the loading of a corpus on a
secondary thread while the MainWindow execution is maintained in the
main thread.

\end{fulllineitems}

\index{execute\_reevaluate\_model() (src.graphical\_user\_interface.main\_window.MainWindow method)@\spxentry{execute\_reevaluate\_model()}\spxextra{src.graphical\_user\_interface.main\_window.MainWindow method}}

\begin{fulllineitems}
\phantomsection\label{\detokenize{gui_main_window:src.graphical_user_interface.main_window.MainWindow.execute_reevaluate_model}}\pysiglinewithargsret{\sphinxbfcode{\sphinxupquote{execute\_reevaluate\_model}}}{}{}
\sphinxAtStartPar
Method to control the execution of the reevaluation of a classifier on a
secondary thread while the MainWindow execution is maintained in the
main thread.

\end{fulllineitems}

\index{execute\_retrain\_model() (src.graphical\_user\_interface.main\_window.MainWindow method)@\spxentry{execute\_retrain\_model()}\spxextra{src.graphical\_user\_interface.main\_window.MainWindow method}}

\begin{fulllineitems}
\phantomsection\label{\detokenize{gui_main_window:src.graphical_user_interface.main_window.MainWindow.execute_retrain_model}}\pysiglinewithargsret{\sphinxbfcode{\sphinxupquote{execute\_retrain\_model}}}{}{}
\sphinxAtStartPar
Method to control the execution of the retraining of a classifier on a
secondary thread while the MainWindow execution is maintained in the
main thread.

\end{fulllineitems}

\index{execute\_train\_classifier() (src.graphical\_user\_interface.main\_window.MainWindow method)@\spxentry{execute\_train\_classifier()}\spxextra{src.graphical\_user\_interface.main\_window.MainWindow method}}

\begin{fulllineitems}
\phantomsection\label{\detokenize{gui_main_window:src.graphical_user_interface.main_window.MainWindow.execute_train_classifier}}\pysiglinewithargsret{\sphinxbfcode{\sphinxupquote{execute\_train\_classifier}}}{}{}
\sphinxAtStartPar
Method to control the execution of the training of a classifier on a
secondary thread while the MainWindow execution is maintained in the
main thread.

\end{fulllineitems}

\index{init\_feedback\_elements() (src.graphical\_user\_interface.main\_window.MainWindow method)@\spxentry{init\_feedback\_elements()}\spxextra{src.graphical\_user\_interface.main\_window.MainWindow method}}

\begin{fulllineitems}
\phantomsection\label{\detokenize{gui_main_window:src.graphical_user_interface.main_window.MainWindow.init_feedback_elements}}\pysiglinewithargsret{\sphinxbfcode{\sphinxupquote{init\_feedback\_elements}}}{}{}
\sphinxAtStartPar
Method for showing the documents to be annotated in the FEEDBACK TAB.
The number of documents that is shown depends on the value assigned to
self.n\_docs\_al; the remaining widgets that exist for showing documents
until Constants.MAX\_N\_DOCS are hided while they are not used.
The widgets that represent the document are displayed as empty spaces
until the conditions for the annotation of the documents are met, i.e.
a corpus and a set of labels have been selected

\end{fulllineitems}

\index{init\_ndocs\_al() (src.graphical\_user\_interface.main\_window.MainWindow method)@\spxentry{init\_ndocs\_al()}\spxextra{src.graphical\_user\_interface.main\_window.MainWindow method}}

\begin{fulllineitems}
\phantomsection\label{\detokenize{gui_main_window:src.graphical_user_interface.main_window.MainWindow.init_ndocs_al}}\pysiglinewithargsret{\sphinxbfcode{\sphinxupquote{init\_ndocs\_al}}}{}{}
\sphinxAtStartPar
Initializes the AL parameter in the text edit within the third tab of the main GUI
window, i.e. n\_docs. The default configuration of this parameter is read from the
configuration file ‘/config/parameters.default.yaml’.

\end{fulllineitems}

\index{init\_params\_train\_pu\_model() (src.graphical\_user\_interface.main\_window.MainWindow method)@\spxentry{init\_params\_train\_pu\_model()}\spxextra{src.graphical\_user\_interface.main\_window.MainWindow method}}

\begin{fulllineitems}
\phantomsection\label{\detokenize{gui_main_window:src.graphical_user_interface.main_window.MainWindow.init_params_train_pu_model}}\pysiglinewithargsret{\sphinxbfcode{\sphinxupquote{init\_params\_train\_pu\_model}}}{}{}
\sphinxAtStartPar
Initializes the classifier parameters in the parameters’ table within the second tab of the main GUI
window, i.e. max\_imbalance and nmax. The default configuration of these parameters is read from the
configuration file ‘/config/parameters.default.yaml’.

\end{fulllineitems}

\index{init\_ui() (src.graphical\_user\_interface.main\_window.MainWindow method)@\spxentry{init\_ui()}\spxextra{src.graphical\_user\_interface.main\_window.MainWindow method}}

\begin{fulllineitems}
\phantomsection\label{\detokenize{gui_main_window:src.graphical_user_interface.main_window.MainWindow.init_ui}}\pysiglinewithargsret{\sphinxbfcode{\sphinxupquote{init\_ui}}}{}{}
\sphinxAtStartPar
Configures the elements of the GUI window that are not configured in the UI, i.e. icon of the application,
the application’s title, and the position of the window at its opening.

\end{fulllineitems}

\index{reset\_params\_train\_pu\_model() (src.graphical\_user\_interface.main\_window.MainWindow method)@\spxentry{reset\_params\_train\_pu\_model()}\spxextra{src.graphical\_user\_interface.main\_window.MainWindow method}}

\begin{fulllineitems}
\phantomsection\label{\detokenize{gui_main_window:src.graphical_user_interface.main_window.MainWindow.reset_params_train_pu_model}}\pysiglinewithargsret{\sphinxbfcode{\sphinxupquote{reset\_params\_train\_pu\_model}}}{}{}
\sphinxAtStartPar
Resets the PU model training parameters to its default value based on the values
that were read initially from the configuration file

\end{fulllineitems}

\index{show\_corpora() (src.graphical\_user\_interface.main\_window.MainWindow method)@\spxentry{show\_corpora()}\spxextra{src.graphical\_user\_interface.main\_window.MainWindow method}}

\begin{fulllineitems}
\phantomsection\label{\detokenize{gui_main_window:src.graphical_user_interface.main_window.MainWindow.show_corpora}}\pysiglinewithargsret{\sphinxbfcode{\sphinxupquote{show\_corpora}}}{}{}
\sphinxAtStartPar
List all corpora contained in the source folder selected by the user.

\end{fulllineitems}

\index{show\_labels() (src.graphical\_user\_interface.main\_window.MainWindow method)@\spxentry{show\_labels()}\spxextra{src.graphical\_user\_interface.main\_window.MainWindow method}}

\begin{fulllineitems}
\phantomsection\label{\detokenize{gui_main_window:src.graphical_user_interface.main_window.MainWindow.show_labels}}\pysiglinewithargsret{\sphinxbfcode{\sphinxupquote{show\_labels}}}{}{}
\sphinxAtStartPar
Method for showing the labels associated with the selected corpus.

\end{fulllineitems}

\index{show\_sampled\_docs\_for\_labeling() (src.graphical\_user\_interface.main\_window.MainWindow method)@\spxentry{show\_sampled\_docs\_for\_labeling()}\spxextra{src.graphical\_user\_interface.main\_window.MainWindow method}}

\begin{fulllineitems}
\phantomsection\label{\detokenize{gui_main_window:src.graphical_user_interface.main_window.MainWindow.show_sampled_docs_for_labeling}}\pysiglinewithargsret{\sphinxbfcode{\sphinxupquote{show\_sampled\_docs\_for\_labeling}}}{}{}
\sphinxAtStartPar
Visualizes the documents from which the user is going to give feedback for the updating of a model.

\end{fulllineitems}

\index{update\_params\_train\_pu\_model() (src.graphical\_user\_interface.main\_window.MainWindow method)@\spxentry{update\_params\_train\_pu\_model()}\spxextra{src.graphical\_user\_interface.main\_window.MainWindow method}}

\begin{fulllineitems}
\phantomsection\label{\detokenize{gui_main_window:src.graphical_user_interface.main_window.MainWindow.update_params_train_pu_model}}\pysiglinewithargsret{\sphinxbfcode{\sphinxupquote{update\_params\_train\_pu\_model}}}{}{}
\sphinxAtStartPar
Updates the classifier parameters that are going to be used for the training of the PU model based on the
values read from the table within the second tab of the main GUI
window that have been specified by the user.

\end{fulllineitems}


\end{fulllineitems}



\chapter{Analyze Keywords Window}
\label{\detokenize{gui_analyze_keywords_window:analyze-keywords-window}}\label{\detokenize{gui_analyze_keywords_window::doc}}\phantomsection\label{\detokenize{gui_analyze_keywords_window:module-src.graphical_user_interface.analyze_keywords_window}}\index{module@\spxentry{module}!src.graphical\_user\_interface.analyze\_keywords\_window@\spxentry{src.graphical\_user\_interface.analyze\_keywords\_window}}\index{src.graphical\_user\_interface.analyze\_keywords\_window@\spxentry{src.graphical\_user\_interface.analyze\_keywords\_window}!module@\spxentry{module}}
\sphinxAtStartPar
@author: lcalv
\index{AnalyzeKeywordsWindow (class in src.graphical\_user\_interface.analyze\_keywords\_window)@\spxentry{AnalyzeKeywordsWindow}\spxextra{class in src.graphical\_user\_interface.analyze\_keywords\_window}}

\begin{fulllineitems}
\phantomsection\label{\detokenize{gui_analyze_keywords_window:src.graphical_user_interface.analyze_keywords_window.AnalyzeKeywordsWindow}}\pysiglinewithargsret{\sphinxbfcode{\sphinxupquote{class\DUrole{w}{  }}}\sphinxcode{\sphinxupquote{src.graphical\_user\_interface.analyze\_keywords\_window.}}\sphinxbfcode{\sphinxupquote{AnalyzeKeywordsWindow}}}{\emph{\DUrole{n}{tm}}}{}
\sphinxAtStartPar
Bases: \sphinxcode{\sphinxupquote{PyQt5.QtWidgets.QDialog}}

\sphinxAtStartPar
Class representing the window that is used for the analysis of the presence of
selected keywords in the corpus
\index{\_\_init\_\_() (src.graphical\_user\_interface.analyze\_keywords\_window.AnalyzeKeywordsWindow method)@\spxentry{\_\_init\_\_()}\spxextra{src.graphical\_user\_interface.analyze\_keywords\_window.AnalyzeKeywordsWindow method}}

\begin{fulllineitems}
\phantomsection\label{\detokenize{gui_analyze_keywords_window:src.graphical_user_interface.analyze_keywords_window.AnalyzeKeywordsWindow.__init__}}\pysiglinewithargsret{\sphinxbfcode{\sphinxupquote{\_\_init\_\_}}}{\emph{\DUrole{n}{tm}}}{}
\sphinxAtStartPar
Initializes a “AnalyzeKeywordsWindow” window.
\begin{quote}\begin{description}
\item[{Parameters}] \leavevmode
\sphinxAtStartPar
\sphinxstylestrong{tm} (\sphinxstyleemphasis{TaskManager}) \textendash{} TaskManager object associated with the project

\end{description}\end{quote}

\end{fulllineitems}

\index{center() (src.graphical\_user\_interface.analyze\_keywords\_window.AnalyzeKeywordsWindow method)@\spxentry{center()}\spxextra{src.graphical\_user\_interface.analyze\_keywords\_window.AnalyzeKeywordsWindow method}}

\begin{fulllineitems}
\phantomsection\label{\detokenize{gui_analyze_keywords_window:src.graphical_user_interface.analyze_keywords_window.AnalyzeKeywordsWindow.center}}\pysiglinewithargsret{\sphinxbfcode{\sphinxupquote{center}}}{}{}
\sphinxAtStartPar
Centers the window at the middle of the screen at which the application is being executed.

\end{fulllineitems}

\index{do\_analysis() (src.graphical\_user\_interface.analyze\_keywords\_window.AnalyzeKeywordsWindow method)@\spxentry{do\_analysis()}\spxextra{src.graphical\_user\_interface.analyze\_keywords\_window.AnalyzeKeywordsWindow method}}

\begin{fulllineitems}
\phantomsection\label{\detokenize{gui_analyze_keywords_window:src.graphical_user_interface.analyze_keywords_window.AnalyzeKeywordsWindow.do_analysis}}\pysiglinewithargsret{\sphinxbfcode{\sphinxupquote{do\_analysis}}}{}{}
\sphinxAtStartPar
Performs the analysis of the keywords based by showing the “Sorted document scores”,
“Document frequencies” and “Keyword frequencies” graphs.

\end{fulllineitems}

\index{initUI() (src.graphical\_user\_interface.analyze\_keywords\_window.AnalyzeKeywordsWindow method)@\spxentry{initUI()}\spxextra{src.graphical\_user\_interface.analyze\_keywords\_window.AnalyzeKeywordsWindow method}}

\begin{fulllineitems}
\phantomsection\label{\detokenize{gui_analyze_keywords_window:src.graphical_user_interface.analyze_keywords_window.AnalyzeKeywordsWindow.initUI}}\pysiglinewithargsret{\sphinxbfcode{\sphinxupquote{initUI}}}{}{}
\sphinxAtStartPar
Configures the elements of the GUI window that are not configured in the UI,
i.e. icon of the application, the application’s title, and the position of the window at its opening.

\end{fulllineitems}


\end{fulllineitems}



\chapter{Constants}
\label{\detokenize{gui_constants:constants}}\label{\detokenize{gui_constants::doc}}\phantomsection\label{\detokenize{gui_constants:module-src.graphical_user_interface.constants}}\index{module@\spxentry{module}!src.graphical\_user\_interface.constants@\spxentry{src.graphical\_user\_interface.constants}}\index{src.graphical\_user\_interface.constants@\spxentry{src.graphical\_user\_interface.constants}!module@\spxentry{module}}
\sphinxAtStartPar
@author: lcalv
\index{Constants (class in src.graphical\_user\_interface.constants)@\spxentry{Constants}\spxextra{class in src.graphical\_user\_interface.constants}}

\begin{fulllineitems}
\phantomsection\label{\detokenize{gui_constants:src.graphical_user_interface.constants.Constants}}\pysigline{\sphinxbfcode{\sphinxupquote{class\DUrole{w}{  }}}\sphinxcode{\sphinxupquote{src.graphical\_user\_interface.constants.}}\sphinxbfcode{\sphinxupquote{Constants}}}
\sphinxAtStartPar
Bases: \sphinxcode{\sphinxupquote{object}}

\sphinxAtStartPar
Class containing a series of constants for GUI configuration.
\index{\_\_weakref\_\_ (src.graphical\_user\_interface.constants.Constants attribute)@\spxentry{\_\_weakref\_\_}\spxextra{src.graphical\_user\_interface.constants.Constants attribute}}

\begin{fulllineitems}
\phantomsection\label{\detokenize{gui_constants:src.graphical_user_interface.constants.Constants.__weakref__}}\pysigline{\sphinxbfcode{\sphinxupquote{\_\_weakref\_\_}}}
\sphinxAtStartPar
list of weak references to the object (if defined)

\end{fulllineitems}


\end{fulllineitems}



\chapter{Get Keywords Window}
\label{\detokenize{gui_get_keywords_window:get-keywords-window}}\label{\detokenize{gui_get_keywords_window::doc}}\phantomsection\label{\detokenize{gui_get_keywords_window:module-src.graphical_user_interface.get_keywords_window}}\index{module@\spxentry{module}!src.graphical\_user\_interface.get\_keywords\_window@\spxentry{src.graphical\_user\_interface.get\_keywords\_window}}\index{src.graphical\_user\_interface.get\_keywords\_window@\spxentry{src.graphical\_user\_interface.get\_keywords\_window}!module@\spxentry{module}}
\sphinxAtStartPar
@author: lcalv
\index{GetKeywordsWindow (class in src.graphical\_user\_interface.get\_keywords\_window)@\spxentry{GetKeywordsWindow}\spxextra{class in src.graphical\_user\_interface.get\_keywords\_window}}

\begin{fulllineitems}
\phantomsection\label{\detokenize{gui_get_keywords_window:src.graphical_user_interface.get_keywords_window.GetKeywordsWindow}}\pysiglinewithargsret{\sphinxbfcode{\sphinxupquote{class\DUrole{w}{  }}}\sphinxcode{\sphinxupquote{src.graphical\_user\_interface.get\_keywords\_window.}}\sphinxbfcode{\sphinxupquote{GetKeywordsWindow}}}{\emph{\DUrole{n}{tm}}}{}
\sphinxAtStartPar
Bases: \sphinxcode{\sphinxupquote{PyQt5.QtWidgets.QDialog}}

\sphinxAtStartPar
Class representing the window that is used for the attainment of a subcorpus
from a given list of keywords, this list being selected by the user.
\index{\_\_init\_\_() (src.graphical\_user\_interface.get\_keywords\_window.GetKeywordsWindow method)@\spxentry{\_\_init\_\_()}\spxextra{src.graphical\_user\_interface.get\_keywords\_window.GetKeywordsWindow method}}

\begin{fulllineitems}
\phantomsection\label{\detokenize{gui_get_keywords_window:src.graphical_user_interface.get_keywords_window.GetKeywordsWindow.__init__}}\pysiglinewithargsret{\sphinxbfcode{\sphinxupquote{\_\_init\_\_}}}{\emph{\DUrole{n}{tm}}}{}
\sphinxAtStartPar
Initializes a “GetKeywordsWindow” window.
\begin{quote}\begin{description}
\item[{Parameters}] \leavevmode
\sphinxAtStartPar
\sphinxstylestrong{tm} (\sphinxstyleemphasis{TaskManager}) \textendash{} TaskManager object associated with the project

\end{description}\end{quote}

\end{fulllineitems}

\index{center() (src.graphical\_user\_interface.get\_keywords\_window.GetKeywordsWindow method)@\spxentry{center()}\spxextra{src.graphical\_user\_interface.get\_keywords\_window.GetKeywordsWindow method}}

\begin{fulllineitems}
\phantomsection\label{\detokenize{gui_get_keywords_window:src.graphical_user_interface.get_keywords_window.GetKeywordsWindow.center}}\pysiglinewithargsret{\sphinxbfcode{\sphinxupquote{center}}}{}{}
\sphinxAtStartPar
Centers the window at the middle of the screen at which the application is being executed.

\end{fulllineitems}

\index{clicked\_select\_keywords() (src.graphical\_user\_interface.get\_keywords\_window.GetKeywordsWindow method)@\spxentry{clicked\_select\_keywords()}\spxextra{src.graphical\_user\_interface.get\_keywords\_window.GetKeywordsWindow method}}

\begin{fulllineitems}
\phantomsection\label{\detokenize{gui_get_keywords_window:src.graphical_user_interface.get_keywords_window.GetKeywordsWindow.clicked_select_keywords}}\pysiglinewithargsret{\sphinxbfcode{\sphinxupquote{clicked\_select\_keywords}}}{}{}
\sphinxAtStartPar
Method to control the actions that are carried out at the time the “Select keywords” button of the “Get
keywords window” is pressed by the user.

\end{fulllineitems}

\index{init\_params() (src.graphical\_user\_interface.get\_keywords\_window.GetKeywordsWindow method)@\spxentry{init\_params()}\spxextra{src.graphical\_user\_interface.get\_keywords\_window.GetKeywordsWindow method}}

\begin{fulllineitems}
\phantomsection\label{\detokenize{gui_get_keywords_window:src.graphical_user_interface.get_keywords_window.GetKeywordsWindow.init_params}}\pysiglinewithargsret{\sphinxbfcode{\sphinxupquote{init\_params}}}{}{}
\sphinxAtStartPar
Initializes the keywords parameters in the parameters’ table within the GUI’s “Get keywords” window,
i.e. wt, n\_max, and s\_min. The default configuration of these parameters is read from the configuration file
‘/config/parameters.default.yaml’.

\end{fulllineitems}

\index{init\_ui() (src.graphical\_user\_interface.get\_keywords\_window.GetKeywordsWindow method)@\spxentry{init\_ui()}\spxextra{src.graphical\_user\_interface.get\_keywords\_window.GetKeywordsWindow method}}

\begin{fulllineitems}
\phantomsection\label{\detokenize{gui_get_keywords_window:src.graphical_user_interface.get_keywords_window.GetKeywordsWindow.init_ui}}\pysiglinewithargsret{\sphinxbfcode{\sphinxupquote{init\_ui}}}{}{}
\sphinxAtStartPar
Configures the elements of the GUI window that are not configured in the UI, i.e. icon of the application,
the application’s title, and the position of the window at its opening.

\end{fulllineitems}

\index{show\_suggested\_keywords() (src.graphical\_user\_interface.get\_keywords\_window.GetKeywordsWindow method)@\spxentry{show\_suggested\_keywords()}\spxextra{src.graphical\_user\_interface.get\_keywords\_window.GetKeywordsWindow method}}

\begin{fulllineitems}
\phantomsection\label{\detokenize{gui_get_keywords_window:src.graphical_user_interface.get_keywords_window.GetKeywordsWindow.show_suggested_keywords}}\pysiglinewithargsret{\sphinxbfcode{\sphinxupquote{show\_suggested\_keywords}}}{}{}
\sphinxAtStartPar
Displays the corresponding keywords based on the configuration parameters selected by the user on the top
QTextEdit “text\_edit\_show\_keywords”.

\end{fulllineitems}

\index{update\_params() (src.graphical\_user\_interface.get\_keywords\_window.GetKeywordsWindow method)@\spxentry{update\_params()}\spxextra{src.graphical\_user\_interface.get\_keywords\_window.GetKeywordsWindow method}}

\begin{fulllineitems}
\phantomsection\label{\detokenize{gui_get_keywords_window:src.graphical_user_interface.get_keywords_window.GetKeywordsWindow.update_params}}\pysiglinewithargsret{\sphinxbfcode{\sphinxupquote{update\_params}}}{}{}
\sphinxAtStartPar
Updates the keywords parameters that are going to be used in the getting of the keywords based on the
values read from the table within the GUI’s “Get keywords” window that have been specified by the user.

\end{fulllineitems}


\end{fulllineitems}



\chapter{Get Topics List Window}
\label{\detokenize{gui_get_topics_list_window:get-topics-list-window}}\label{\detokenize{gui_get_topics_list_window::doc}}\phantomsection\label{\detokenize{gui_get_topics_list_window:module-src.graphical_user_interface.get_topics_list_window}}\index{module@\spxentry{module}!src.graphical\_user\_interface.get\_topics\_list\_window@\spxentry{src.graphical\_user\_interface.get\_topics\_list\_window}}\index{src.graphical\_user\_interface.get\_topics\_list\_window@\spxentry{src.graphical\_user\_interface.get\_topics\_list\_window}!module@\spxentry{module}}
\sphinxAtStartPar
@author: lcalv
\index{GetTopicsListWindow (class in src.graphical\_user\_interface.get\_topics\_list\_window)@\spxentry{GetTopicsListWindow}\spxextra{class in src.graphical\_user\_interface.get\_topics\_list\_window}}

\begin{fulllineitems}
\phantomsection\label{\detokenize{gui_get_topics_list_window:src.graphical_user_interface.get_topics_list_window.GetTopicsListWindow}}\pysiglinewithargsret{\sphinxbfcode{\sphinxupquote{class\DUrole{w}{  }}}\sphinxcode{\sphinxupquote{src.graphical\_user\_interface.get\_topics\_list\_window.}}\sphinxbfcode{\sphinxupquote{GetTopicsListWindow}}}{\emph{\DUrole{n}{tm}}}{}
\sphinxAtStartPar
Bases: \sphinxcode{\sphinxupquote{PyQt5.QtWidgets.QDialog}}

\sphinxAtStartPar
Class representing the window in charge of getting a subcorpus from a given
list of topics, such a list being specified by the user.
\index{\_\_init\_\_() (src.graphical\_user\_interface.get\_topics\_list\_window.GetTopicsListWindow method)@\spxentry{\_\_init\_\_()}\spxextra{src.graphical\_user\_interface.get\_topics\_list\_window.GetTopicsListWindow method}}

\begin{fulllineitems}
\phantomsection\label{\detokenize{gui_get_topics_list_window:src.graphical_user_interface.get_topics_list_window.GetTopicsListWindow.__init__}}\pysiglinewithargsret{\sphinxbfcode{\sphinxupquote{\_\_init\_\_}}}{\emph{\DUrole{n}{tm}}}{}
\sphinxAtStartPar
Initializes a “GetTopicsListWindow” window.
\begin{quote}\begin{description}
\item[{Parameters}] \leavevmode
\sphinxAtStartPar
\sphinxstylestrong{tm} (\sphinxstyleemphasis{TaskManager}) \textendash{} TaskManager object associated with the project

\end{description}\end{quote}

\end{fulllineitems}

\index{center() (src.graphical\_user\_interface.get\_topics\_list\_window.GetTopicsListWindow method)@\spxentry{center()}\spxextra{src.graphical\_user\_interface.get\_topics\_list\_window.GetTopicsListWindow method}}

\begin{fulllineitems}
\phantomsection\label{\detokenize{gui_get_topics_list_window:src.graphical_user_interface.get_topics_list_window.GetTopicsListWindow.center}}\pysiglinewithargsret{\sphinxbfcode{\sphinxupquote{center}}}{}{}
\sphinxAtStartPar
Centers the window at the middle of the screen at which the application is being executed.

\end{fulllineitems}

\index{clicked\_get\_topic\_list() (src.graphical\_user\_interface.get\_topics\_list\_window.GetTopicsListWindow method)@\spxentry{clicked\_get\_topic\_list()}\spxextra{src.graphical\_user\_interface.get\_topics\_list\_window.GetTopicsListWindow method}}

\begin{fulllineitems}
\phantomsection\label{\detokenize{gui_get_topics_list_window:src.graphical_user_interface.get_topics_list_window.GetTopicsListWindow.clicked_get_topic_list}}\pysiglinewithargsret{\sphinxbfcode{\sphinxupquote{clicked\_get\_topic\_list}}}{}{}
\sphinxAtStartPar
Method to control the actions that are carried out at the time the “Select weighted topic list” button of
the “Get topics window” is pressed by the user.

\end{fulllineitems}

\index{initUI() (src.graphical\_user\_interface.get\_topics\_list\_window.GetTopicsListWindow method)@\spxentry{initUI()}\spxextra{src.graphical\_user\_interface.get\_topics\_list\_window.GetTopicsListWindow method}}

\begin{fulllineitems}
\phantomsection\label{\detokenize{gui_get_topics_list_window:src.graphical_user_interface.get_topics_list_window.GetTopicsListWindow.initUI}}\pysiglinewithargsret{\sphinxbfcode{\sphinxupquote{initUI}}}{}{}
\sphinxAtStartPar
Configures the elements of the GUI window that are not configured in the UI, i.e. icon of the application,
the application’s title, and the position of the window at its opening.

\end{fulllineitems}

\index{init\_params() (src.graphical\_user\_interface.get\_topics\_list\_window.GetTopicsListWindow method)@\spxentry{init\_params()}\spxextra{src.graphical\_user\_interface.get\_topics\_list\_window.GetTopicsListWindow method}}

\begin{fulllineitems}
\phantomsection\label{\detokenize{gui_get_topics_list_window:src.graphical_user_interface.get_topics_list_window.GetTopicsListWindow.init_params}}\pysiglinewithargsret{\sphinxbfcode{\sphinxupquote{init\_params}}}{}{}
\sphinxAtStartPar
Initializes the topics parameters in the parameters’ table within the GUI’s “Get topic list” window,
i.e. n\_max and s\_min. The default configuration of these parameters is read from the configuration file
‘/config/parameters.default.yaml’.

\end{fulllineitems}

\index{show\_topics() (src.graphical\_user\_interface.get\_topics\_list\_window.GetTopicsListWindow method)@\spxentry{show\_topics()}\spxextra{src.graphical\_user\_interface.get\_topics\_list\_window.GetTopicsListWindow method}}

\begin{fulllineitems}
\phantomsection\label{\detokenize{gui_get_topics_list_window:src.graphical_user_interface.get_topics_list_window.GetTopicsListWindow.show_topics}}\pysiglinewithargsret{\sphinxbfcode{\sphinxupquote{show\_topics}}}{}{}
\sphinxAtStartPar
Configures the “table\_widget\_topic\_list” and “table\_widget\_topics\_weight” tables to have the appropriate
number of columns and rows based on the available topics, and fills out the “table\_widget\_topic\_list” table
with the id and corresponding chemical description of each of the topics.

\end{fulllineitems}

\index{update\_params() (src.graphical\_user\_interface.get\_topics\_list\_window.GetTopicsListWindow method)@\spxentry{update\_params()}\spxextra{src.graphical\_user\_interface.get\_topics\_list\_window.GetTopicsListWindow method}}

\begin{fulllineitems}
\phantomsection\label{\detokenize{gui_get_topics_list_window:src.graphical_user_interface.get_topics_list_window.GetTopicsListWindow.update_params}}\pysiglinewithargsret{\sphinxbfcode{\sphinxupquote{update\_params}}}{}{}
\sphinxAtStartPar
Updates the topics parameters that are going to be used in the getting of the keywords based on the values
read from the table within the GUI’s “Get topic list” window that have been specified by the user.

\end{fulllineitems}

\index{updated\_topic\_weighted\_list() (src.graphical\_user\_interface.get\_topics\_list\_window.GetTopicsListWindow method)@\spxentry{updated\_topic\_weighted\_list()}\spxextra{src.graphical\_user\_interface.get\_topics\_list\_window.GetTopicsListWindow method}}

\begin{fulllineitems}
\phantomsection\label{\detokenize{gui_get_topics_list_window:src.graphical_user_interface.get_topics_list_window.GetTopicsListWindow.updated_topic_weighted_list}}\pysiglinewithargsret{\sphinxbfcode{\sphinxupquote{updated\_topic\_weighted\_list}}}{}{}
\sphinxAtStartPar
Generates the topic weighted list based on the weights that the user has introduced on the
“table\_widget\_topics\_weight” table

\end{fulllineitems}


\end{fulllineitems}



\chapter{Messages}
\label{\detokenize{gui_messages:messages}}\label{\detokenize{gui_messages::doc}}\phantomsection\label{\detokenize{gui_messages:module-src.graphical_user_interface.messages}}\index{module@\spxentry{module}!src.graphical\_user\_interface.messages@\spxentry{src.graphical\_user\_interface.messages}}\index{src.graphical\_user\_interface.messages@\spxentry{src.graphical\_user\_interface.messages}!module@\spxentry{module}}
\sphinxAtStartPar
@author: lcalv
\index{Messages (class in src.graphical\_user\_interface.messages)@\spxentry{Messages}\spxextra{class in src.graphical\_user\_interface.messages}}

\begin{fulllineitems}
\phantomsection\label{\detokenize{gui_messages:src.graphical_user_interface.messages.Messages}}\pysigline{\sphinxbfcode{\sphinxupquote{class\DUrole{w}{  }}}\sphinxcode{\sphinxupquote{src.graphical\_user\_interface.messages.}}\sphinxbfcode{\sphinxupquote{Messages}}}
\sphinxAtStartPar
Bases: \sphinxcode{\sphinxupquote{object}}

\sphinxAtStartPar
Class containing the majority of the messages utilized in the GUI to
facilitate the readability of the code.
\index{\_\_weakref\_\_ (src.graphical\_user\_interface.messages.Messages attribute)@\spxentry{\_\_weakref\_\_}\spxextra{src.graphical\_user\_interface.messages.Messages attribute}}

\begin{fulllineitems}
\phantomsection\label{\detokenize{gui_messages:src.graphical_user_interface.messages.Messages.__weakref__}}\pysigline{\sphinxbfcode{\sphinxupquote{\_\_weakref\_\_}}}
\sphinxAtStartPar
list of weak references to the object (if defined)

\end{fulllineitems}


\end{fulllineitems}



\chapter{Output Wrapper}
\label{\detokenize{gui_output_wrapper:output-wrapper}}\label{\detokenize{gui_output_wrapper::doc}}\phantomsection\label{\detokenize{gui_output_wrapper:module-src.graphical_user_interface.output_wrapper}}\index{module@\spxentry{module}!src.graphical\_user\_interface.output\_wrapper@\spxentry{src.graphical\_user\_interface.output\_wrapper}}\index{src.graphical\_user\_interface.output\_wrapper@\spxentry{src.graphical\_user\_interface.output\_wrapper}!module@\spxentry{module}}
\sphinxAtStartPar
@author: lcalv
\index{OutputWrapper (class in src.graphical\_user\_interface.output\_wrapper)@\spxentry{OutputWrapper}\spxextra{class in src.graphical\_user\_interface.output\_wrapper}}

\begin{fulllineitems}
\phantomsection\label{\detokenize{gui_output_wrapper:src.graphical_user_interface.output_wrapper.OutputWrapper}}\pysiglinewithargsret{\sphinxbfcode{\sphinxupquote{class\DUrole{w}{  }}}\sphinxcode{\sphinxupquote{src.graphical\_user\_interface.output\_wrapper.}}\sphinxbfcode{\sphinxupquote{OutputWrapper}}}{\emph{\DUrole{n}{parent}}, \emph{\DUrole{n}{stdout}\DUrole{o}{=}\DUrole{default_value}{True}}}{}
\sphinxAtStartPar
Bases: \sphinxcode{\sphinxupquote{PyQt5.QtCore.QObject}}

\sphinxAtStartPar
Module that overrides the “sys.stderr” and “sys.stdout” with a wrapper object
that emits a signal whenever output is written. In order to account for other
modules that need “sys.stdout” / “sys.stderr” (such as the logging module) use
the wrapped versions wherever necessary, the instance of the OutputWrapper are
created before the TaskManager object.

\sphinxAtStartPar
It has been created based on the analogous
class provided by:
\sphinxurl{https://stackoverflow.com/questions/19855288/duplicate-stdout-stderr-in-qtextedit-widget}
\index{\_\_getattr\_\_() (src.graphical\_user\_interface.output\_wrapper.OutputWrapper method)@\spxentry{\_\_getattr\_\_()}\spxextra{src.graphical\_user\_interface.output\_wrapper.OutputWrapper method}}

\begin{fulllineitems}
\phantomsection\label{\detokenize{gui_output_wrapper:src.graphical_user_interface.output_wrapper.OutputWrapper.__getattr__}}\pysiglinewithargsret{\sphinxbfcode{\sphinxupquote{\_\_getattr\_\_}}}{\emph{\DUrole{n}{self}}, \emph{\DUrole{n}{str}}}{{ $\rightarrow$ object}}
\end{fulllineitems}

\index{\_\_init\_\_() (src.graphical\_user\_interface.output\_wrapper.OutputWrapper method)@\spxentry{\_\_init\_\_()}\spxextra{src.graphical\_user\_interface.output\_wrapper.OutputWrapper method}}

\begin{fulllineitems}
\phantomsection\label{\detokenize{gui_output_wrapper:src.graphical_user_interface.output_wrapper.OutputWrapper.__init__}}\pysiglinewithargsret{\sphinxbfcode{\sphinxupquote{\_\_init\_\_}}}{\emph{\DUrole{n}{parent}}, \emph{\DUrole{n}{stdout}\DUrole{o}{=}\DUrole{default_value}{True}}}{}
\end{fulllineitems}


\end{fulllineitems}



\chapter{Util}
\label{\detokenize{gui_util:util}}\label{\detokenize{gui_util::doc}}\phantomsection\label{\detokenize{gui_util:module-src.graphical_user_interface.util}}\index{module@\spxentry{module}!src.graphical\_user\_interface.util@\spxentry{src.graphical\_user\_interface.util}}\index{src.graphical\_user\_interface.util@\spxentry{src.graphical\_user\_interface.util}!module@\spxentry{module}}
\sphinxAtStartPar
@author: lcalv
\index{change\_background\_color\_text\_edit() (in module src.graphical\_user\_interface.util)@\spxentry{change\_background\_color\_text\_edit()}\spxextra{in module src.graphical\_user\_interface.util}}

\begin{fulllineitems}
\phantomsection\label{\detokenize{gui_util:src.graphical_user_interface.util.change_background_color_text_edit}}\pysiglinewithargsret{\sphinxcode{\sphinxupquote{src.graphical\_user\_interface.util.}}\sphinxbfcode{\sphinxupquote{change\_background\_color\_text\_edit}}}{\emph{\DUrole{n}{text\_edit}}, \emph{\DUrole{n}{prediction}}}{}
\sphinxAtStartPar
Method to that changes the border color of the QTextEdit associated with the documents to be annotated, based on
the predicted class selected by the user.
\begin{quote}\begin{description}
\item[{Parameters}] \leavevmode\begin{itemize}
\item {} 
\sphinxAtStartPar
\sphinxstylestrong{text\_edit} (\sphinxstyleemphasis{QTextEdit}) \textendash{} QTextEdit (document) whose border color is going to be updated based on the

\item {} 
\sphinxAtStartPar
\sphinxstylestrong{prediction} (\sphinxstyleemphasis{int (0 or 1)}) \textendash{} Predicted class specified by the user. If prediction == 1, the document’s associated QTextEdit’s border
color is set to \#6A7288; if prediction == 0, it is set to \#DB8678.

\end{itemize}

\end{description}\end{quote}

\end{fulllineitems}

\index{execute\_in\_thread() (in module src.graphical\_user\_interface.util)@\spxentry{execute\_in\_thread()}\spxextra{in module src.graphical\_user\_interface.util}}

\begin{fulllineitems}
\phantomsection\label{\detokenize{gui_util:src.graphical_user_interface.util.execute_in_thread}}\pysiglinewithargsret{\sphinxcode{\sphinxupquote{src.graphical\_user\_interface.util.}}\sphinxbfcode{\sphinxupquote{execute\_in\_thread}}}{\emph{\DUrole{n}{gui}}, \emph{\DUrole{n}{function}}, \emph{\DUrole{n}{function\_output}}, \emph{\DUrole{n}{progress\_bar}}}{}
\sphinxAtStartPar
Method to execute a function in the secondary thread while showing
a progress bar at the time the function is being executed if a progress bar object is provided.
When finished, it forces the execution of the method to be
executed after the function executing in a thread is completed.
Based on the functions provided in the manual available at:
\sphinxurl{https://www.pythonguis.com/tutorials/multithreading-pyqt-applications-qthreadpool/}
\begin{quote}\begin{description}
\item[{Parameters}] \leavevmode\begin{itemize}
\item {} 
\sphinxAtStartPar
\sphinxstylestrong{function} (\sphinxstyleemphasis{UDF}) \textendash{} Function to be executed in thread

\item {} 
\sphinxAtStartPar
\sphinxstylestrong{function\_output} (\sphinxstyleemphasis{UDF}) \textendash{} Function to be executed at the end of the thread

\item {} 
\sphinxAtStartPar
\sphinxstylestrong{progress\_bar} (\sphinxstyleemphasis{QProgressBar}) \textendash{} If a QProgressBar object is provided, it shows a progress bar in the
main thread while the main task is being carried out in a secondary thread

\end{itemize}

\end{description}\end{quote}

\end{fulllineitems}

\index{signal\_accept() (in module src.graphical\_user\_interface.util)@\spxentry{signal\_accept()}\spxextra{in module src.graphical\_user\_interface.util}}

\begin{fulllineitems}
\phantomsection\label{\detokenize{gui_util:src.graphical_user_interface.util.signal_accept}}\pysiglinewithargsret{\sphinxcode{\sphinxupquote{src.graphical\_user\_interface.util.}}\sphinxbfcode{\sphinxupquote{signal\_accept}}}{\emph{\DUrole{n}{progress\_bar}}}{}
\sphinxAtStartPar
Makes the progress bar passed as an argument visible and configures it for
an event whose duration is unknown by setting both its minimum and maximum
both to 0, thus the bar shows a busy indicator instead of a percentage of steps.
\begin{quote}\begin{description}
\item[{Parameters}] \leavevmode
\sphinxAtStartPar
\sphinxstylestrong{progress\_bar} (\sphinxstyleemphasis{QProgressBar}) \textendash{} Progress bar object in which the progress is going to be displayed.

\end{description}\end{quote}

\end{fulllineitems}

\index{toggle\_menu() (in module src.graphical\_user\_interface.util)@\spxentry{toggle\_menu()}\spxextra{in module src.graphical\_user\_interface.util}}

\begin{fulllineitems}
\phantomsection\label{\detokenize{gui_util:src.graphical_user_interface.util.toggle_menu}}\pysiglinewithargsret{\sphinxcode{\sphinxupquote{src.graphical\_user\_interface.util.}}\sphinxbfcode{\sphinxupquote{toggle\_menu}}}{\emph{\DUrole{n}{gui}}, \emph{\DUrole{n}{max\_width}}}{}
\sphinxAtStartPar
Method to control the movement of the Toggle menu located on the
left. When collapsed, only the icon for each of the options is shown;
when expanded, both icons and name indicating the description of the
functionality are shown.
Based on the code available at:
\sphinxurl{https://github.com/Wanderson-Magalhaes/Toggle\_Burguer\_Menu\_Python\_PySide2/blob/master/ui\_functions.py}
\begin{quote}\begin{description}
\item[{Parameters}] \leavevmode\begin{itemize}
\item {} 
\sphinxAtStartPar
\sphinxstylestrong{gui} (\sphinxstyleemphasis{MainWindow}) \textendash{} MainWindow object to which the toggle menu will be appended.

\item {} 
\sphinxAtStartPar
\sphinxstylestrong{maxWidth} (\sphinxstyleemphasis{int}) \textendash{} Maximum width to which the toggle menu is going to be expanded.

\end{itemize}

\end{description}\end{quote}

\end{fulllineitems}



\chapter{Worker Signals}
\label{\detokenize{gui_worker_signals:worker-signals}}\label{\detokenize{gui_worker_signals::doc}}\phantomsection\label{\detokenize{gui_worker_signals:module-src.graphical_user_interface.worker_signals}}\index{module@\spxentry{module}!src.graphical\_user\_interface.worker\_signals@\spxentry{src.graphical\_user\_interface.worker\_signals}}\index{src.graphical\_user\_interface.worker\_signals@\spxentry{src.graphical\_user\_interface.worker\_signals}!module@\spxentry{module}}
\sphinxAtStartPar
Created on Tue Mar  2 13:19:34 2021
@author: lcalv
\index{WorkerSignals (class in src.graphical\_user\_interface.worker\_signals)@\spxentry{WorkerSignals}\spxextra{class in src.graphical\_user\_interface.worker\_signals}}

\begin{fulllineitems}
\phantomsection\label{\detokenize{gui_worker_signals:src.graphical_user_interface.worker_signals.WorkerSignals}}\pysigline{\sphinxbfcode{\sphinxupquote{class\DUrole{w}{  }}}\sphinxcode{\sphinxupquote{src.graphical\_user\_interface.worker\_signals.}}\sphinxbfcode{\sphinxupquote{WorkerSignals}}}
\sphinxAtStartPar
Bases: \sphinxcode{\sphinxupquote{PyQt5.QtCore.QObject}}

\sphinxAtStartPar
Module that defines the signals that are available from a running
worker thread, the supported signals being “finished” (there is no more data to process),
“error”, “result” (object data returned from processing) and “progress” (a
numerical indicator of the progress that has been achieved at a particular moment).
It has been created based on the analogous class provided by:
\sphinxurl{https://www.pythonguis.com/tutorials/multithreading-pyqt-applications-qthreadpool/}

\end{fulllineitems}



\chapter{Worker}
\label{\detokenize{gui_worker:worker}}\label{\detokenize{gui_worker::doc}}\phantomsection\label{\detokenize{gui_worker:module-src.graphical_user_interface.worker}}\index{module@\spxentry{module}!src.graphical\_user\_interface.worker@\spxentry{src.graphical\_user\_interface.worker}}\index{src.graphical\_user\_interface.worker@\spxentry{src.graphical\_user\_interface.worker}!module@\spxentry{module}}
\sphinxAtStartPar
Created on Tue Mar  2 13:19:34 2021
@author: lcalv
\index{Worker (class in src.graphical\_user\_interface.worker)@\spxentry{Worker}\spxextra{class in src.graphical\_user\_interface.worker}}

\begin{fulllineitems}
\phantomsection\label{\detokenize{gui_worker:src.graphical_user_interface.worker.Worker}}\pysiglinewithargsret{\sphinxbfcode{\sphinxupquote{class\DUrole{w}{  }}}\sphinxcode{\sphinxupquote{src.graphical\_user\_interface.worker.}}\sphinxbfcode{\sphinxupquote{Worker}}}{\emph{\DUrole{n}{fn}}, \emph{\DUrole{o}{*}\DUrole{n}{args}}, \emph{\DUrole{o}{**}\DUrole{n}{kwargs}}}{}
\sphinxAtStartPar
Bases: \sphinxcode{\sphinxupquote{PyQt5.QtCore.QRunnable}}

\sphinxAtStartPar
Module that inherits from QRunnable and is used to handler worker
thread setup, signals and wrap\sphinxhyphen{}up. It has been created based on the analogous
class provided by:
\sphinxurl{https://www.pythonguis.com/tutorials/multithreading-pyqt-applications-qthreadpool/}.
\index{\_\_init\_\_() (src.graphical\_user\_interface.worker.Worker method)@\spxentry{\_\_init\_\_()}\spxextra{src.graphical\_user\_interface.worker.Worker method}}

\begin{fulllineitems}
\phantomsection\label{\detokenize{gui_worker:src.graphical_user_interface.worker.Worker.__init__}}\pysiglinewithargsret{\sphinxbfcode{\sphinxupquote{\_\_init\_\_}}}{\emph{\DUrole{n}{fn}}, \emph{\DUrole{o}{*}\DUrole{n}{args}}, \emph{\DUrole{o}{**}\DUrole{n}{kwargs}}}{}
\sphinxAtStartPar
Initializes the application’s main window based on the parameters received
from the application’s starting window.
\begin{quote}\begin{description}
\item[{Parameters}] \leavevmode\begin{itemize}
\item {} 
\sphinxAtStartPar
\sphinxstylestrong{callback} (\sphinxstyleemphasis{UDF}) \textendash{} The function callback to run on this worker thread.
Supplied args and kwargs will be passed through to the runner.

\item {} 
\sphinxAtStartPar
\sphinxstylestrong{callback} (\sphinxstyleemphasis{UDF}) \textendash{} Function

\item {} 
\sphinxAtStartPar
\sphinxstylestrong{args} (\sphinxstyleemphasis{list}) \textendash{} Arguments to pass to the callback function

\item {} 
\sphinxAtStartPar
\sphinxstylestrong{kwargs} (\sphinxstyleemphasis{dict}) \textendash{} Keywords to pass to the callback function

\end{itemize}

\end{description}\end{quote}

\end{fulllineitems}

\index{run() (src.graphical\_user\_interface.worker.Worker method)@\spxentry{run()}\spxextra{src.graphical\_user\_interface.worker.Worker method}}

\begin{fulllineitems}
\phantomsection\label{\detokenize{gui_worker:src.graphical_user_interface.worker.Worker.run}}\pysiglinewithargsret{\sphinxbfcode{\sphinxupquote{run}}}{}{}
\sphinxAtStartPar
Initialises the runner function with passed args, kwargs.

\end{fulllineitems}


\end{fulllineitems}



\chapter{Indices and tables}
\label{\detokenize{index:indices-and-tables}}\begin{itemize}
\item {} 
\sphinxAtStartPar
\DUrole{xref,std,std-ref}{genindex}

\item {} 
\sphinxAtStartPar
\DUrole{xref,std,std-ref}{modindex}

\item {} 
\sphinxAtStartPar
\DUrole{xref,std,std-ref}{search}

\end{itemize}


\renewcommand{\indexname}{Python Module Index}
\begin{sphinxtheindex}
\let\bigletter\sphinxstyleindexlettergroup
\bigletter{s}
\item\relax\sphinxstyleindexentry{src.base\_taskmanager}\sphinxstyleindexpageref{dc_base_taskmanager:\detokenize{module-src.base_taskmanager}}
\item\relax\sphinxstyleindexentry{src.data\_manager}\sphinxstyleindexpageref{dc_data_manager:\detokenize{module-src.data_manager}}
\item\relax\sphinxstyleindexentry{src.domain\_classifier.classifier}\sphinxstyleindexpageref{dc_classifier:\detokenize{module-src.domain_classifier.classifier}}
\item\relax\sphinxstyleindexentry{src.domain\_classifier.custom\_model}\sphinxstyleindexpageref{dc_custom_model:\detokenize{module-src.domain_classifier.custom_model}}
\item\relax\sphinxstyleindexentry{src.domain\_classifier.preprocessor}\sphinxstyleindexpageref{dc_preprocessor:\detokenize{module-src.domain_classifier.preprocessor}}
\item\relax\sphinxstyleindexentry{src.graphical\_user\_interface.analyze\_keywords\_window}\sphinxstyleindexpageref{gui_analyze_keywords_window:\detokenize{module-src.graphical_user_interface.analyze_keywords_window}}
\item\relax\sphinxstyleindexentry{src.graphical\_user\_interface.constants}\sphinxstyleindexpageref{gui_constants:\detokenize{module-src.graphical_user_interface.constants}}
\item\relax\sphinxstyleindexentry{src.graphical\_user\_interface.get\_keywords\_window}\sphinxstyleindexpageref{gui_get_keywords_window:\detokenize{module-src.graphical_user_interface.get_keywords_window}}
\item\relax\sphinxstyleindexentry{src.graphical\_user\_interface.get\_topics\_list\_window}\sphinxstyleindexpageref{gui_get_topics_list_window:\detokenize{module-src.graphical_user_interface.get_topics_list_window}}
\item\relax\sphinxstyleindexentry{src.graphical\_user\_interface.main\_window}\sphinxstyleindexpageref{gui_main_window:\detokenize{module-src.graphical_user_interface.main_window}}
\item\relax\sphinxstyleindexentry{src.graphical\_user\_interface.messages}\sphinxstyleindexpageref{gui_messages:\detokenize{module-src.graphical_user_interface.messages}}
\item\relax\sphinxstyleindexentry{src.graphical\_user\_interface.output\_wrapper}\sphinxstyleindexpageref{gui_output_wrapper:\detokenize{module-src.graphical_user_interface.output_wrapper}}
\item\relax\sphinxstyleindexentry{src.graphical\_user\_interface.util}\sphinxstyleindexpageref{gui_util:\detokenize{module-src.graphical_user_interface.util}}
\item\relax\sphinxstyleindexentry{src.graphical\_user\_interface.worker}\sphinxstyleindexpageref{gui_worker:\detokenize{module-src.graphical_user_interface.worker}}
\item\relax\sphinxstyleindexentry{src.graphical\_user\_interface.worker\_signals}\sphinxstyleindexpageref{gui_worker_signals:\detokenize{module-src.graphical_user_interface.worker_signals}}
\item\relax\sphinxstyleindexentry{src.menu\_navigator.menu\_navigator}\sphinxstyleindexpageref{mn_menu_navigator:\detokenize{module-src.menu_navigator.menu_navigator}}
\item\relax\sphinxstyleindexentry{src.query\_manager}\sphinxstyleindexpageref{dc_query_manager:\detokenize{module-src.query_manager}}
\item\relax\sphinxstyleindexentry{src.task\_manager}\sphinxstyleindexpageref{dc_task_manager:\detokenize{module-src.task_manager}}
\end{sphinxtheindex}

\renewcommand{\indexname}{Index}
\printindex
\end{document}